%\documentclass[smaller,handout]{beamer}
\def\bmode{0} % Mode 0 for presentation, mode 1 for a handout with notes, mode 2 for handout without notes
\if 0\bmode
\documentclass[smaller]{beamer}
\else \if 1\bmode
\immediate\write18{pdflatex -jobname=\jobname-Notes-Handout\space\jobname}
\documentclass[smaller,handout]{beamer}
\usepackage{handoutWithNotes}
\pgfpagesuselayout{2 on 1 with notes}[letterpaper, landscape, border shrink=4mm]
\else \if 2\bmode
\immediate\write18{pdflatex -jobname=\jobname-Handout\space\jobname}
\documentclass[smaller,handout]{beamer}
\fi
\fi
\fi

% \documentclass[smaller,handout
% ]{beamer}
%\usepackage{etex}
%\newcommand{\num}{6{} }

% \usetheme[
%   outer/progressbar=foot,
%   outer/numbering=counter,
%  block=fillFF
% ]{metropolis}

%\useoutertheme{metropolis}

\usetheme{Madrid}
\useoutertheme[subsection=false]{miniframes} % Alternatively: miniframes, infolines, split
\useinnertheme{circles}
\usecolortheme{seahorse}

% \usepackage[backend=biber,style=authoryear,maxcitenames=2,maxbibnames=99,safeinputenc,url=false,
% eprint=false]{biblatex}
% \addbibresource{bib/references.bib}
% \AtEveryCitekey{\iffootnote{{\tiny}\tiny}{\tiny}}

%\usepackage{pgfpages}
%\setbeameroption{hide notes} % Only slides
%\setbeameroption{show only notes} % Only notes
%\setbeameroption{hide notes} % Only notes
%\setbeameroption{show notes on second screen=right} % Both

% \usepackage[sfdefault]{Fira Sans}

% \setsansfont[BoldFont={Fira Sans}]{Fira Sans Light}
% \setmonofont{Fira Mono}

%\usepackage{fira}
%\setsansfont{Fira}
%\setmonofont{Fira Mono}
% To give a presentation with the Skim reader (http://skim-app.sourceforge.net) on OSX so
% that you see the notes on your laptop and the slides on the projector, do the following:
% 
% 1. Generate just the presentation (hide notes) and save to slides.pdf
% 2. Generate onlt the notes (show only nodes) and save to notes.pdf
% 3. With Skim open both slides.pdf and notes.pdf
% 4. Click on slides.pdf to bring it to front.
% 5. In Skim, under "View -> Presentation Option -> Synhcronized Noted Document"
%    select notes.pdf.
% 6. Now as you move around in slides.pdf the notes.pdf file will follow you.
% 7. Arrange windows so that notes.pdf is in full screen mode on your laptop
%    and slides.pdf is in presentation mode on the projector.

% Give a slight yellow tint to the notes page
%\setbeamertemplate{note page}{\pagecolor{yellow!5}\insertnote}\usepackage{palatino}


%\usetheme{metropolis}
%\usecolortheme{beaver}
\usepackage{xcolor}
\definecolor{darkcandyapplered}{HTML}{A40000}
\definecolor{lightcandyapplered}{HTML}{e74c3c}

%\setbeamercolor{title}{fg=darkcandyapplered}
%\setbeamercolor{frametitle}{bg=darkcandyapplered!80!black!90!white}
%\setbeamertemplate{frametitle}{\bf\insertframetitle}
%\setbeamercolor{footnote mark}{fg=darkcandyapplered}
%\setbeamercolor{footnote}{fg=darkcandyapplered!70}
%\Raggedbottom
%\setbeamerfont{page number in head/foot}{size=\tiny}
%\usepackage[tracking]{microtype}


\setbeamertemplate{frametitle}{%
    \nointerlineskip%
    \begin{beamercolorbox}[wd=\paperwidth,ht=2.0ex,dp=0.6ex]{frametitle}
        \hspace*{1ex}\insertframetitle%
    \end{beamercolorbox}%
}



\setbeamerfont{caption}{size=\footnotesize}
\setbeamercolor{caption name}{fg=darkcandyapplered}


%\usepackage[sc,osf]{mathpazo}   % With old-style figures and real smallcaps.
%\linespread{1.025}              % Palatino leads a little more leading

% Euler for math and numbers
%\usepackage[euler-digits,small]{eulervm}
%\AtBeginDocument{\renewcommand{\hbar}{\hslash}}
\usepackage{graphicx,multirow,paralist,booktabs}


%\mode<presentation> { \setbeamercovered{transparent} }

\setbeamertemplate{navigation symbols}{}
\makeatletter
\def\beamerorig@set@color{%
  \pdfliteral{\current@color}%
  \aftergroup\reset@color
}
\def\beamerorig@reset@color{\pdfliteral{\current@color}}
\makeatother

%=== GRAPHICS PATH ===========
\graphicspath{{./m4-images/}}
% Marginpar width
%Marginpar width
%\setlength{\marginparsep}{.02in}


%% Captions
% \usepackage{caption}
% \captionsetup{
%   labelsep=quad,
%   justification=raggedright,
%   labelfont=sc
% }

%AMS-TeX packages

\usepackage{amssymb,amsmath,amsthm} 
\usepackage{bm}
\usepackage{color}

\usepackage{hyperref,enumerate}
\usepackage{minitoc,array}


%https://tex.stackexchange.com/a/31370/2269
\usepackage{mathtools,cancel}

\renewcommand{\CancelColor}{\color{red}} %change cancel color to red

\makeatletter
\let\my@cancelto\cancelto %copy over the original cancelto command
\newcommand<>{\cancelto}[2]{\alt#3{\my@cancelto{#1}{#2}}{\mathrlap{#2}\phantom{\my@cancelto{#1}{#2}}}}
% redefine the cancelto command, using \phantom to assure that the
% result doesn't wiggle up and down with and without the arrow
\makeatother


\definecolor{slblue}{rgb}{0,.3,.62}
\hypersetup{
    colorlinks,%
    citecolor=blue,%
    filecolor=blue,%
    linkcolor=blue,
    urlcolor=slblue
}

%%% TIKZ
\usepackage{animate}
\usepackage{tikz}
\usepackage{pgfplots}
\usepackage{pgfplotstable}
\usepackage{pgfgantt}
\usepackage{tikzsymbols}
\pgfplotsset{compat=newest}
\usepgfplotslibrary{groupplots,fillbetween}

\usetikzlibrary{arrows,shapes,positioning,shapes.geometric}
\usetikzlibrary{decorations.markings}
\usetikzlibrary{shadows,automata}
\usetikzlibrary{patterns,matrix}
\usetikzlibrary{trees,mindmap,backgrounds}
%\usetikzlibrary{circuits.ee.IEC}
\usetikzlibrary{decorations.text}
% For Sagnac Picture
\usetikzlibrary{%
    decorations.pathreplacing,%
    decorations.pathmorphing%
}
\tikzset{no shadows/.style={general shadow/.style=}}
%
%\usepackage{paralist}



%%% FORMAT PYTHON CODE
%\usepackage{listings}
% Default fixed font does not support bold face
\DeclareFixedFont{\ttb}{T1}{txtt}{bx}{n}{8} % for bold
\DeclareFixedFont{\ttm}{T1}{txtt}{m}{n}{8}  % for normal

% Custom colors
\definecolor{deepblue}{rgb}{0,0,0.5}
\definecolor{deepred}{rgb}{0.6,0,0}
\definecolor{deepgreen}{rgb}{0,0.5,0}
 

%\usepackage{listings}

% Python style for highlighting
% \newcommand\pythonstyle{\lstset{
% language=Python,
% basicstyle=\footnotesize\ttm,
% otherkeywords={self},             % Add keywords here
% keywordstyle=\footnotesize\ttb\color{deepblue},
% emph={MyClass,__init__},          % Custom highlighting
% emphstyle=\footnotesize\ttb\color{deepred},    % Custom highlighting style
% stringstyle=\color{deepgreen},
% frame=tb,                         % Any extra options here
    % showstringspaces=false            % 
% }}

% % Python environment
% \lstnewenvironment{python}[1][]
% {
% \pythonstyle
% \lstset{#1}
% }
% {}

% % Python for external files
% \newcommand\pythonexternal[2][]{{
% \pythonstyle
% \lstinputlisting[#1]{#2}}}

% Python for inline
% 
% \newcommand\pythoninline[1]{{\pythonstyle\lstinline!#1!}}


\newcommand{\osn}{\oldstylenums}
\newcommand{\dg}{^{\circ}}
\newcommand{\lt}{\left}
\newcommand{\rt}{\right}
\newcommand{\pt}{\phantom}
\newcommand{\tf}{\therefore}
\newcommand{\?}{\stackrel{?}{=}}
\newcommand{\fr}{\frac}
\newcommand{\dfr}{\dfrac}
\newcommand{\ul}{\underline}
\newcommand{\tn}{\tabularnewline}
\newcommand{\nl}{\newline}
\newcommand\relph[1]{\mathrel{\phantom{#1}}}
\newcommand{\cm}{\checkmark}
\newcommand{\ol}{\overline}
\newcommand{\rd}{\color{red}}
\newcommand{\bl}{\color{blue}}
\newcommand{\pl}{\color{purple}}
\newcommand{\og}{\color{orange!90!black}}
\newcommand{\gr}{\color{green!40!black}}
\newcommand{\nin}{\noindent}
\newcommand{\la}{\lambda}
\renewcommand{\th}{\theta}
\newcommand{\al}{\alpha}
\newcommand{\G}{\Gamma}
\newcommand*\circled[1]{\tikz[baseline=(char.base)]{
            \node[shape=circle,draw,thick,inner sep=1pt] (char) {\small #1};}}

\newcommand{\bc}{\begin{compactenum}[\quad--]}
\newcommand{\ec}{\end{compactenum}}

\newcommand{\p}{\partial}
\newcommand{\pd}[2]{\frac{\partial{#1}}{\partial{#2}}}
\newcommand{\dpd}[2]{\dfrac{\partial{#1}}{\partial{#2}}}
\newcommand{\pdd}[2]{\frac{\partial^2{#1}}{\partial{#2}^2}}
\newcommand{\nmfr}[3]{\Phi\left(\frac{{#1} - {#2}}{#3}\right)}


\pgfmathdeclarefunction{poiss}{1}{%
  \pgfmathparse{(#1^x)*exp(-#1)/(x!)}%
  }

\pgfmathdeclarefunction{gauss}{2}{%
  \pgfmathparse{1/(#2*sqrt(2*pi))*exp(-((x-#1)^2)/(2*#2^2))}%
}

\pgfmathdeclarefunction{expo}{2}{%
  \pgfmathparse{#1*exp(-#1*#2)}%
}

\pgfmathdeclarefunction{expocdf}{2}{%
  \pgfmathparse{1 -exp(-#1*#2)}%
}

% \makeatletter
% \long\def\ifnodedefined#1#2#3{%
%     \@ifundefined{pgf@sh@ns@#1}{#3}{#2}%
% }

% \pgfplotsset{
%     discontinuous/.style={
%     scatter,
%     scatter/@pre marker code/.code={
%         \ifnodedefined{marker}{
%             \pgfpointdiff{\pgfpointanchor{marker}{center}}%
%              {\pgfpoint{0}{0}}%
%              \ifdim\pgf@y>0pt
%                 \tikzset{options/.style={mark=*, fill=white}}
%                 \draw [densely dashed] (marker-|0,0) -- (0,0);
%                 \draw plot [mark=*] coordinates {(marker-|0,0)};
%              \else
%                 \tikzset{options/.style={mark=none}}
%              \fi
%         }{
%             \tikzset{options/.style={mark=none}}        
%         }
%         \coordinate (marker) at (0,0);
%         \begin{scope}[options]
%     },
%     scatter/@post marker code/.code={\end{scope}}
%     }
% }

% \makeatother

\renewcommand{\arraystretch}{1.5}
%%%%%%%%%%%%%%%%%%%%%%%%%%%%%%%%%%%%%%%%%%%%%%%%%%%
%%%%%%%%%%%%%%%%%%%%%%%%%%%%%%%%%%%%%%%%%%%%%%%%%%%

\title[CEE 260/MIE 273 M4b: CIs for Proportion]{{\normalsize CEE 260/MIE 273: Probability and Statistics in Civil Engineering} \\
Lecture M4b: Confidence Intervals for a Proportion}
\date[\today]{\footnotesize \today}
\author{{\bf Prof.\ Oke}}
\institute[UMass Amherst]{
  \begin{tikzpicture}[baseline=(current bounding box.center)]
    \node[anchor=base] at (-7,0) (its) {\includegraphics[scale=.3]{UMassEngineering_vert}} ;
  \end{tikzpicture}
}


    
\begin{document}

\maketitle




\begin{frame}
  \frametitle{Outline}
  \tableofcontents
\end{frame}

\section{Preamble}
\begin{frame}
  \frametitle{Recap: parameter estimation}
  \pause

  \begin{itemize}
  \item A point estimator is used to find the estimate $\hat\theta$ for a given parameter $\theta$.

    \pause

  \item An ideal estimator is: \pause unbiased, consistent, efficient and sufficient

  \item The bias of a parameter  estimate is the difference between its expected value and the true value of the parameter: \pause
    \begin{equation}
      \text{Bias}_{\hat\theta} = \mathbb{E}(\hat\theta) - \theta
    \end{equation}

  \end{itemize}
  
\end{frame}
 

\begin{frame}
  \frametitle{Today's objectives}
  \pause

  \begin{itemize}[<+->]
  \item Understand confidence intervals and their importance \medskip
  \item Compute confidence intervals for a proportion
    \begin{itemize}[<+->]
    \item find the critical $z^{*}$ for a corresponding confidence level
    \item compute the standard error
    \item compute the margin of error
    \end{itemize}
  \item Learn how to use the success-failure condition to determine whether CLT holds for a proportion
  \end{itemize}
\end{frame}

\section{Confidence intervals}


\begin{frame}
  \frametitle{Confidence intervals}

\pause
\begin{block}{Definition}\pause
  A confidence interval (CI) defines the range within which a population parameter $\theta$ lies with a given probability  $1-\alpha$ 
\end{block}
\pause

 % \begin{block}{Known population variance: normal distribution used}\pause
    \begin{equation}
      \label{eq:1}
      \langle \theta \rangle_{1-\alpha} \pause
      =  \lt( \hat{\theta} + {\rd  z_{\fr\alpha2}} {\gr  \fr{\sigma}{\sqrt{n}} },\, \pause
      \hat{\theta} + {\rd z_{\lt(1-\fr\alpha2\rt)}}{ \gr \fr{\sigma}{\sqrt{n}} }\rt) 
    \end{equation}\pause
    where:
    \begin{itemize}
    \item $\hat\theta$ is the point estimate \pause
    \item $\gr \fr{\sigma}{\sqrt{n}}$ is the \textbf{standard error} ($SE$) \pause
    \item $ {\rd z_{\lt(1-\fr\alpha2\rt)}} {\gr \fr{\sigma}{\sqrt{n}}} $ is the \textbf{margin of error} ($ME$)\pause
    \item $\rd z_{\fr\alpha2}$ and $\rd  z_{\lt(1-\fr\alpha2\rt)}$ are the \textbf{critical z-scores} \pause
      \begin{equation}
        z_{\fr\alpha2} = - z_{\lt(1-\fr\alpha2\rt)} \equiv - z^{*}
      \end{equation}
      
    \end{itemize}
    \pause
    Thus, we can express the CI as:
    \begin{equation}
      \langle \theta \rangle_{1-\alpha} = \hat\theta \pm z^{*}SE \pause = \hat\theta \pm ME
    \end{equation}
 
\end{frame}


\begin{frame}
  \frametitle{Confidence intervals (cont.)}
  \pause

  We define: \pause
  \begin{itemize}[<+->]
  \item Confidence level: $1-\alpha$\pause
  \item Significance level: $\alpha$
  \item Find critical $z$-score (standardized) values: $z_{\alpha/2}$ and $z_{(1-\alpha/2)}$
  \item Convert these to same scale as original variable $X$
  \end{itemize}
  \pause
  \begin{figure}
  \begin{tikzpicture}
    \begin{axis}[no markers, domain=0:10, samples=100,
      axis x line=center,
      axis y line=none,
      xlabel=$z$, ylabel=$f_X(x)$,,
      height=6cm, width=10cm,
      xtick={-6,0,6},
      xticklabels={$z_{\alpha/2}$,$0$,$z_{(1-\alpha/2)}$},
      ymax=.15,
      ytick=\empty,
      x label style={anchor=west},
      y label style={anchor=south},
      enlargelimits=true, clip=false, axis on top
      %grid style={line width=.1pt, draw=gray},
      % yticklabel style={
      %   /pgf/number format/fixed,
      %   /pgf/number format/fixed zerofill,
      %   /pgf/number format/precision=2
      % },        
      %   grid = major
      ]
      \addplot [blue, domain=-10:10] {gauss(0,3)};
      \addplot [gray, fill=gray!50, domain=-6:6] {gauss(0,3)} \closedcycle;
      \addplot [orange,fill=orange,  domain=6:10] {gauss(0,3)} \closedcycle;
      \addplot [orange,fill=orange,  domain=-10:-6] {gauss(0,3)} \closedcycle;
      \node (d) at (axis cs: 0,.05) {Area: $1-\alpha$};
      %\draw[thick, ->] (d) -- (axis cs: -.6,.05);
      \node (c) at (axis cs: 8.5,.05) {Area: $\fr\alpha2$};
      \draw[thick,->] (c) -- (axis cs: 6.5, 0.003);
      \node (e) at (axis cs: -8.5,.05) {Area: $\fr\alpha2$};
      \draw[thick,->] (e) -- (axis cs: -6.5, 0.003);
      %\draw[thick, |->] (axis cs: 6,-0.025) -- (axis cs: 10,-0.025) node[below,pos=.5] {\small\og Reject $H_0$};
      %\draw[thick, |->] (axis cs: -6,-0.025) -- (axis cs: -10,-0.025) node[below,pos=.5] {\small\og Reject $H_0$};
    \end{axis}
  \end{tikzpicture}
  \caption{Standard normal distribution of the mean}
  \end{figure}
\end{frame}

 
\begin{frame}
  \frametitle{Confidence interval of proportion}
  Given a proportion (occurrence probability) $p$, then
  \begin{eqnarray}
    \mathbb{E}(\hat{P}) &=& \hat{p} \\ \pause
    \mathbb{V}(\hat{P}) &=& \fr{\hat p( 1- \hat p)}{n} \quad \text{(according to CLT)}
  \end{eqnarray}

  \pause
  The confidence interval is given by:
  \begin{equation}
    \label{eq:30}
    \boxed{\langle p\rangle_{1-\alpha}  = \lt(
    \hat{p} + z_{\alpha/2}\sqrt{\fr{\hat{p}(1 - \hat{p})}{n}};  \hat{p} + z_{(1 -\alpha/2)}\sqrt{\fr{\hat{p}(1 - \hat{p})}{n}}
      \rt)}
  \end{equation}
  \pause
  Thus:
  \begin{itemize}
  \item $SE = \sqrt{\fr{\hat{p}(1 - \hat{p})}{n}}$ \pause
  \item $ME =  z_{(1 -\alpha/2)}\sqrt{\fr{\hat{p}(1 - \hat{p})}{n}} = z^{*}\times SE $    
  \end{itemize}
  \pause
  {\rd Must check that success-failure condition holds: $np \geq 10$ and $n(1-p) \geq 10$}
\end{frame}


\section{Identifying confidence levels}
\begin{frame}
  \frametitle{Working with confidence intervals}
  \begin{exampleblock}{Example 1: Identifying confidence levels}
    Given a normal population distribution with known variance:
    \begin{enumerate}[(a)]
    \item What is the confidence level for the interval $\hat{p} \pm 2.81\sigma/\sqrt{n}$?
    \item What is the confidence level for the interval $\hat{p} \pm 1.44\sigma/\sqrt{n}$?
    \item What value of $z_{\alpha/2}$ results in a confidence level of 90\%?
    \end{enumerate}
  \end{exampleblock}
\end{frame}

\begin{frame}
  \frametitle{Working with confidence intervals}
  \begin{exampleblock}{Example 1: Identifying confidence levels (cont.)}\pause
    \begin{enumerate}[(a)] 
    \item What is the confidence level for the interval $\hat{p} \pm 2.81\sigma/\sqrt{n}$?\pause
      \begin{eqnarray*}
        z_{(1-\alpha/2)} &=& +2.81 \\\pause
        1-\alpha/2 &=& \Phi(2.81) \pause = 99.75\% \\\pause
        \alpha/2 &=& 0.25\% \\\pause
        \alpha &=& 0.5\%
      \end{eqnarray*}\pause
      The confidence level is \pause $= 1 - \alpha \pause = \boxed{99.5\%}$.

      \pause

        \begin{tikzpicture}[scale=.8]
      \begin{axis}[no markers, domain=0:10, samples=100,
      axis x line=center,
      axis y line=none,
      xlabel=$z$, ylabel=$f_X(x)$,,
      height=3cm, width=10cm,
      xtick={-2.81,0,2.81},
      xticklabels={$z_{\alpha/2}$,$0$,$z_{(1-\alpha/2)}$},
      ymax=.15,
      ytick=\empty,
      x label style={anchor=west},
      y label style={anchor=south},
      enlargelimits=true, clip=false, axis on top
      %grid style={line width=.1pt, draw=gray},
      % yticklabel style={
      %   /pgf/number format/fixed,
      %   /pgf/number format/fixed zerofill,
      %   /pgf/number format/precision=2
      % },        
      %   grid = major
      ]
      \addplot [blue, domain=-5:5] {gauss(0,1)};
      \addplot [gray, fill=gray!50, domain=-2.81:2.81] {gauss(0,1)} \closedcycle;
      \addplot [orange,fill=orange,  domain=2.81:5] {gauss(0,1)} \closedcycle;
      \addplot [orange,fill=orange,  domain=-5:-2.81] {gauss(0,1)} \closedcycle;
      \node (d) at (axis cs: 0,.15) {Area: $0.995$};
      %\draw[thick, ->] (d) -- (axis cs: -.6,.05);
      \node (c) at (axis cs: 3.15,.1) {\og Area: $0.0025$};
      \draw[thick,->] (c) -- (axis cs: 2.9, 0.005);
      \node (e) at (axis cs: -3.15,.1) {\og Area: $0.0025$};
      \draw[thick,->] (e) -- (axis cs: -2.9, 0.005);
      %\draw[thick, |->] (axis cs: 6,-0.025) -- (axis cs: 10,-0.025) node[below,pos=.5] {\small\og Reject $H_0$};
      %\draw[thick, |->] (axis cs: -6,-0.025) -- (axis cs: -10,-0.025) node[below,pos=.5] {\small\og Reject $H_0$};
    \end{axis}
  \end{tikzpicture}
    \end{enumerate}
  \end{exampleblock}
\end{frame}

\begin{frame}
  \frametitle{Working with confidence intervals}
  \begin{exampleblock}{Example 1: Identifying confidence levels (cont.)}\pause
    \begin{enumerate}[(a)]\setcounter{enumi}{1}
    \item What is the confidence level for the interval $\hat{p} \pm 1.44\sigma/\sqrt{n}$?\pause
      \begin{eqnarray*}
        z_{(1-\alpha/2)} &=& +1.44\\\pause
        1- \alpha/2 &=& \pause \Phi(1.44) \pause = 92.5\% \\\pause
        \alpha &=& 15\% 
      \end{eqnarray*}
      The confidence level is \pause $= 1 - \alpha \pause = \boxed{85\%}$.
      \pause

      \begin{tikzpicture}
      \begin{axis}[no markers, domain=0:10, samples=100,
      axis x line=center,
      axis y line=none,
      xlabel=$z$, ylabel=$f_X(x)$,,
      height=3cm, width=10cm,
      xtick={-1.44,0,1.44},
      xticklabels={$z_{\alpha/2}$,$0$,$z_{(1-\alpha/2)}$},
      ymax=.15,
      ytick=\empty,
      x label style={anchor=west},
      y label style={anchor=south},
      enlargelimits=true, clip=false, axis on top
      %grid style={line width=.1pt, draw=gray},
      % yticklabel style={
      %   /pgf/number format/fixed,
      %   /pgf/number format/fixed zerofill,
      %   /pgf/number format/precision=2
      % },        
      %   grid = major
      ]
      \addplot [blue, domain=-5:5] {gauss(0,1)};
      \addplot [gray, fill=gray!50, domain=-1.44:1.44] {gauss(0,1)} \closedcycle;
      \addplot [orange,fill=orange,  domain=1.44:5] {gauss(0,1)} \closedcycle;
      \addplot [orange,fill=orange,  domain=-5:-1.44] {gauss(0,1)} \closedcycle;
      \node (d) at (axis cs: 0,.15) {Area: $0.85$};
      %\draw[thick, ->] (d) -- (axis cs: -.6,.05);
      \node (c) at (axis cs: 3.15,.1) {\og Area: $0.075$};
      \draw[thick,->] (c) -- (axis cs: 1.8, 0.03);
      \node (e) at (axis cs: -3.15,.1) {\og Area: $0.075$};
      \draw[thick,->] (e) -- (axis cs: -1.8, 0.03);
      %\draw[thick, |->] (axis cs: 6,-0.025) -- (axis cs: 10,-0.025) node[below,pos=.5] {\small\og Reject $H_0$};
      %\draw[thick, |->] (axis cs: -6,-0.025) -- (axis cs: -10,-0.025) node[below,pos=.5] {\small\og Reject $H_0$};
    \end{axis}
  \end{tikzpicture}
    \end{enumerate}
  \end{exampleblock}
\end{frame}

\begin{frame}
  \frametitle{Working with confidence intervals}
  \begin{exampleblock}{Example 1: Identifying confidence levels  (cont.)}\pause
    \begin{enumerate}[(a)]\setcounter{enumi}{2}
    \item What value of $z_{\alpha/2}$ results in a confidence level of 90\%? \pause
      \begin{eqnarray*}
        z_{\alpha/2} &=& \Phi^{-1}(0.05) \\\pause
                     &=& -\Phi^{-1}(0.95) \\\pause
                     &=& -1.64 
      \end{eqnarray*} \pause

            \begin{tikzpicture}
      \begin{axis}[no markers, domain=0:10, samples=100,
      axis x line=center,
      axis y line=none,
      xlabel=$z$, ylabel=$f_X(x)$,,
      height=3cm, width=10cm,
      xtick={-1.64,0,1.64},
      xticklabels={$z_{\alpha/2}$,$0$,$z_{(1-\alpha/2)}$},
      ymax=.15,
      ytick=\empty,
      x label style={anchor=west},
      y label style={anchor=south},
      enlargelimits=true, clip=false, axis on top
      %grid style={line width=.1pt, draw=gray},
      % yticklabel style={
      %   /pgf/number format/fixed,
      %   /pgf/number format/fixed zerofill,
      %   /pgf/number format/precision=2
      % },        
      %   grid = major
      ]
      \addplot [blue, domain=-5:5] {gauss(0,1)};
      \addplot [gray, fill=gray!50, domain=-1.64:1.64] {gauss(0,1)} \closedcycle;
      \addplot [orange,fill=orange,  domain=1.64:5] {gauss(0,1)} \closedcycle;
      \addplot [orange,fill=orange,  domain=-5:-1.64] {gauss(0,1)} \closedcycle;
      \node (d) at (axis cs: 0,.15) {Area: $0.90$};
      %\draw[thick, ->] (d) -- (axis cs: -.6,.05);
      \node (c) at (axis cs: 3.15,.1) {\og Area: $0.05$};
      \draw[thick,->] (c) -- (axis cs: 1.8, 0.03);
      \node (e) at (axis cs: -3.15,.1) {\og Area: $0.05$};
      \draw[thick,->] (e) -- (axis cs: -1.8, 0.03);
      %\draw[thick, |->] (axis cs: 6,-0.025) -- (axis cs: 10,-0.025) node[below,pos=.5] {\small\og Reject $H_0$};
      %\draw[thick, |->] (axis cs: -6,-0.025) -- (axis cs: -10,-0.025) node[below,pos=.5] {\small\og Reject $H_0$};
    \end{axis}
  \end{tikzpicture}
  
    \end{enumerate}
  \end{exampleblock}
\end{frame}



\begin{frame}
  \frametitle{Example 2: Solar power expansion}
  \pause A poll indicates that 88.7\% of a random sample of 1000 American adults supported solar power
  expansion. Compute and interpret a 95\% CI for the population propotion $p$, given that $SE_{\hat p} = 0.01$

  \pause
  \begin{exampleblock}{Solution}\pause
    The corresponding critical z-score $z^{*}$ for a 95\% confidence level is \texttt{norminv}(1 - .025) = 1.96. \pause
    Thus, the CI is given by
    \begin{eqnarray*}
      \langle p \rangle_{90\%} &=& \hat{p} \pm z^{*}\times SE_{\hat p }\\\pause
                               &=& .887 \pm 1.96 (.01) \\\pause
                               &=& (0.8674, .9066)
    \end{eqnarray*}
  \end{exampleblock}
\end{frame}

 
\begin{frame}
  \frametitle{Example 3: Mental health}
  \pause
  The General Social Survey asked the question: “For how many days during the past
30 days was your mental health, which includes stress, depression, and problems with emotions, not good?”
Based on responses from 1,151 US residents, the survey reported a 95\% confidence interval of 3.40 to 4.24
days in 2010.
\begin{enumerate}[\bf(a)]
\item  Interpret this interval in context of the data.
\item What does “95\% confident” mean? Explain in the context of the application.
\item Suppose the researchers think a 99\% confidence level would be more appropriate for this interval. Will
this new interval be smaller or wider than the 95\% confidence interval?
\item  If a new survey were to be done with 500 Americans, do you think the standard error of the estimate
be larger, smaller, or about the same.
\end{enumerate}
\end{frame}
\section{Outlook}




\begin{frame}
  \frametitle{Recap}
  \pause
  Steps to find the CI of a proportion
  \begin{itemize}[<+->]
  \item Compute $n$ and $\hat p$
  \item Check that the success-failure condition holds
  \item Find the critical $z$-score for the confidence level $1-\alpha$
  \item Compute the $SE = \sqrt{p(1-p)/n}$
  \item Find the $ME$ ($z^{*}\times SE$)
  \item Find the CI as $\hat p \pm ME$
  \end{itemize}
\end{frame}

 
\end{document}
%%% Local Variables:
%%% mode: latex
%%% TeX-master: t
%%% End:
