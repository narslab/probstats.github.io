\def\bmode{0} % Mode 0 for presentation, mode 1 for a handout with notes, mode 2 for handout without notes
\if 0\bmode
\documentclass[smaller]{beamer}
\else \if 1\bmode
\immediate\write18{pdflatex -jobname=\jobname-Notes-Handout\space\jobname}
\documentclass[smaller,handout]{beamer}
\usepackage{handoutWithNotes}
\pgfpagesuselayout{2 on 1 with notes}[letterpaper, landscape, border shrink=4mm]
\else \if 2\bmode
\immediate\write18{pdflatex -jobname=\jobname_Handout\space\jobname}
\documentclass[smaller,handout]{beamer}
\fi
\fi
\fi
%\documentclass[smaller, handout]{beamer}

% \documentclass[smaller,handout
% ]{beamer}
%\usepackage{etex}
%\newcommand{\num}{6{} }

% \usetheme[
%   outer/progressbar=foot,
%   outer/numbering=counter,
%  block=fillFF
% ]{metropolis}

%\useoutertheme{metropolis}

\usetheme{Madrid}
\useoutertheme[subsection=false]{miniframes} % Alternatively: miniframes, infolines, split
\useinnertheme{circles}
\usecolortheme{seahorse}

\usepackage[backend=biber,style=authoryear,maxcitenames=2,maxbibnames=99,safeinputenc,url=false,
eprint=false]{biblatex}
\addbibresource{bib/references.bib}
\AtEveryCitekey{\iffootnote{{\tiny}\tiny}{\tiny}}
\usepackage{appendixnumberbeamer}
%\usepackage{pgfpages}
%\setbeameroption{hide notes} % Only slides
%\setbeameroption{show only notes} % Only notes
%\setbeameroption{hide notes} % Only notes
%\setbeameroption{show notes on second screen=right} % Both

% \usepackage[sfdefault]{Fira Sans}

% \setsansfont[BoldFont={Fira Sans}]{Fira Sans Light}
% \setmonofont{Fira Mono}

%\usepackage{fira}
%\setsansfont{Fira}
%\setmonofont{Fira Mono}
% To give a presentation with the Skim reader (http://skim-app.sourceforge.net) on OSX so
% that you see the notes on your laptop and the slides on the projector, do the following:
% 
% 1. Generate just the presentation (hide notes) and save to slides.pdf
% 2. Generate onlt the notes (show only nodes) and save to notes.pdf
% 3. With Skim open both slides.pdf and notes.pdf
% 4. Click on slides.pdf to bring it to front.
% 5. In Skim, under "View -> Presentation Option -> Synhcronized Noted Document"
%    select notes.pdf.
% 6. Now as you move around in slides.pdf the notes.pdf file will follow you.
% 7. Arrange windows so that notes.pdf is in full screen mode on your laptop
%    and slides.pdf is in presentation mode on the projector.

% Give a slight yellow tint to the notes page
%\setbeamertemplate{note page}{\pagecolor{yellow!5}\insertnote}\usepackage{palatino}


%\usetheme{metropolis}
%\usecolortheme{beaver}
\usepackage{xcolor}
\definecolor{darkcandyapplered}{HTML}{A40000}
\definecolor{lightcandyapplered}{HTML}{e74c3c}

%\setbeamercolor{title}{fg=darkcandyapplered}
%\setbeamercolor{frametitle}{bg=darkcandyapplered!80!black!90!white}
%\setbeamertemplate{frametitle}{\bf\insertframetitle}
%\setbeamercolor{footnote mark}{fg=darkcandyapplered}
%\setbeamercolor{footnote}{fg=darkcandyapplered!70}
%\Raggedbottom
%\setbeamerfont{page number in head/foot}{size=\tiny}
%\usepackage[tracking]{microtype}


\setbeamertemplate{frametitle}{%
    \nointerlineskip%
    \begin{beamercolorbox}[wd=\paperwidth,ht=2.0ex,dp=0.6ex]{frametitle}
        \hspace*{1ex}\insertframetitle%
    \end{beamercolorbox}%
}



\setbeamerfont{caption}{size=\footnotesize}
\setbeamercolor{caption name}{fg=darkcandyapplered}


%\usepackage[sc,osf]{mathpazo}   % With old-style figures and real smallcaps.
%\linespread{1.025}              % Palatino leads a little more leading

% Euler for math and numbers
%\usepackage[euler-digits,small]{eulervm}
%\AtBeginDocument{\renewcommand{\hbar}{\hslash}}
\usepackage{graphicx,multirow,paralist,booktabs}


%\mode<presentation> { \setbeamercovered{transparent} }

\setbeamertemplate{navigation symbols}{}
\makeatletter
\def\beamerorig@set@color{%
  \pdfliteral{\current@color}%
  \aftergroup\reset@color
}
\def\beamerorig@reset@color{\pdfliteral{\current@color}}
\makeatother

%=== GRAPHICS PATH ===========
\graphicspath{{./m5-images/}}
% Marginpar width
%Marginpar width
%\setlength{\marginparsep}{.02in}


%% Captions
% \usepackage{caption}
% \captionsetup{
%   labelsep=quad,
%   justification=raggedright,
%   labelfont=sc
% }

%AMS-TeX packages

\usepackage{amssymb,amsmath,amsthm} 
\usepackage{bm}
\usepackage{color}

\usepackage{hyperref,enumerate}
\usepackage{minitoc,array}


%https://tex.stackexchange.com/a/31370/2269
\usepackage{mathtools,cancel}

\renewcommand{\CancelColor}{\color{red}} %change cancel color to red

\makeatletter
\let\my@cancelto\cancelto %copy over the original cancelto command
\newcommand<>{\cancelto}[2]{\alt#3{\my@cancelto{#1}{#2}}{\mathrlap{#2}\phantom{\my@cancelto{#1}{#2}}}}
% redefine the cancelto command, using \phantom to assure that the
% result doesn't wiggle up and down with and without the arrow
\makeatother


\definecolor{slblue}{rgb}{0,.3,.62}
\hypersetup{
    colorlinks,%
    citecolor=blue,%
    filecolor=blue,%
    linkcolor=blue,
    urlcolor=slblue
}

%%% TIKZ
\usepackage{animate}
\usepackage{tikz}
\usepackage{pgfplots}
\usepackage{pgfplotstable}
\usepackage{pgfgantt}
\usepackage{tikzsymbols}
\pgfplotsset{compat=newest}
\usepgfplotslibrary{groupplots,fillbetween}

\usetikzlibrary{arrows,shapes,positioning,shapes.geometric}
\usetikzlibrary{decorations.markings}
\usetikzlibrary{shadows,automata}
\usetikzlibrary{patterns,matrix}
\usetikzlibrary{trees,mindmap,backgrounds}
%\usetikzlibrary{circuits.ee.IEC}
\usetikzlibrary{decorations.text}
% For Sagnac Picture
\usetikzlibrary{%
    decorations.pathreplacing,%
    decorations.pathmorphing%
}
\tikzset{no shadows/.style={general shadow/.style=}}
%
%\usepackage{paralist}



%%% FORMAT PYTHON CODE
%\usepackage{listings}
% Default fixed font does not support bold face
\DeclareFixedFont{\ttb}{T1}{txtt}{bx}{n}{8} % for bold
\DeclareFixedFont{\ttm}{T1}{txtt}{m}{n}{8}  % for normal

% Custom colors
\definecolor{deepblue}{rgb}{0,0,0.5}
\definecolor{deepred}{rgb}{0.6,0,0}
\definecolor{deepgreen}{rgb}{0,0.5,0}
 

%\usepackage{listings}

% Python style for highlighting
% \newcommand\pythonstyle{\lstset{
% language=Python,
% basicstyle=\footnotesize\ttm,
% otherkeywords={self},             % Add keywords here
% keywordstyle=\footnotesize\ttb\color{deepblue},
% emph={MyClass,__init__},          % Custom highlighting
% emphstyle=\footnotesize\ttb\color{deepred},    % Custom highlighting style
% stringstyle=\color{deepgreen},
% frame=tb,                         % Any extra options here
    % showstringspaces=false            % 
% }}

% % Python environment
% \lstnewenvironment{python}[1][]
% {
% \pythonstyle
% \lstset{#1}
% }
% {}

% % Python for external files
% \newcommand\pythonexternal[2][]{{
% \pythonstyle
% \lstinputlisting[#1]{#2}}}

% Python for inline
% 
% \newcommand\pythoninline[1]{{\pythonstyle\lstinline!#1!}}


\newcommand{\osn}{\oldstylenums}
\newcommand{\dg}{^{\circ}}
\newcommand{\lt}{\left}
\newcommand{\rt}{\right}
\newcommand{\pt}{\phantom}
\newcommand{\tf}{\therefore}
\newcommand{\?}{\stackrel{?}{=}}
\newcommand{\fr}{\frac}
\newcommand{\dfr}{\dfrac}
\newcommand{\ul}{\underline}
\newcommand{\tn}{\tabularnewline}
\newcommand{\nl}{\newline}
\newcommand\relph[1]{\mathrel{\phantom{#1}}}
\newcommand{\cm}{\checkmark}
\newcommand{\ol}{\overline}
\newcommand{\rd}{\color{red}}
\newcommand{\bl}{\color{blue}}
\newcommand{\pl}{\color{purple}}
\newcommand{\og}{\color{orange!90!black}}
\newcommand{\gr}{\color{green!40!black}}
\newcommand{\nin}{\noindent}
\newcommand{\la}{\lambda}
\renewcommand{\th}{\theta}
\newcommand{\al}{\alpha}
\newcommand{\G}{\Gamma}
\newcommand*\circled[1]{\tikz[baseline=(char.base)]{
            \node[shape=circle,draw,thick,inner sep=1pt] (char) {\small #1};}}

\newcommand{\bc}{\begin{compactenum}[\quad--]}
\newcommand{\ec}{\end{compactenum}}

\newcommand{\p}{\partial}
\newcommand{\pd}[2]{\frac{\partial{#1}}{\partial{#2}}}
\newcommand{\dpd}[2]{\dfrac{\partial{#1}}{\partial{#2}}}
\newcommand{\pdd}[2]{\frac{\partial^2{#1}}{\partial{#2}^2}}
\newcommand{\nmfr}[3]{\Phi\left(\frac{{#1} - {#2}}{#3}\right)}


\pgfmathdeclarefunction{poiss}{1}{%
  \pgfmathparse{(#1^x)*exp(-#1)/(x!)}%
  }

\pgfmathdeclarefunction{gauss}{2}{%
  \pgfmathparse{1/(#2*sqrt(2*pi))*exp(-((x-#1)^2)/(2*#2^2))}%
}

\pgfmathdeclarefunction{expo}{2}{%
  \pgfmathparse{#1*exp(-#1*#2)}%
}

\pgfmathdeclarefunction{expocdf}{2}{%
  \pgfmathparse{1 -exp(-#1*#2)}%
}

% \makeatletter
% \long\def\ifnodedefined#1#2#3{%
%     \@ifundefined{pgf@sh@ns@#1}{#3}{#2}%
% }

% \pgfplotsset{
%     discontinuous/.style={
%     scatter,
%     scatter/@pre marker code/.code={
%         \ifnodedefined{marker}{
%             \pgfpointdiff{\pgfpointanchor{marker}{center}}%
%              {\pgfpoint{0}{0}}%
%              \ifdim\pgf@y>0pt
%                 \tikzset{options/.style={mark=*, fill=white}}
%                 \draw [densely dashed] (marker-|0,0) -- (0,0);
%                 \draw plot [mark=*] coordinates {(marker-|0,0)};
%              \else
%                 \tikzset{options/.style={mark=none}}
%              \fi
%         }{
%             \tikzset{options/.style={mark=none}}        
%         }
%         \coordinate (marker) at (0,0);
%         \begin{scope}[options]
%     },
%     scatter/@post marker code/.code={\end{scope}}
%     }
% }

% \makeatother

\renewcommand{\arraystretch}{1.5}
%%%%%%%%%%%%%%%%%%%%%%%%%%%%%%%%%%%%%%%%%%%%%%%%%%%
%%%%%%%%%%%%%%%%%%%%%%%%%%%%%%%%%%%%%%%%%%%%%%%%%%%

\title[CEE 260/MIE 273 M5a: Inference for Single Prop.]{{\normalsize CEE 260/MIE 273: Probability and Statistics in Civil Engineering} \\
Lecture M5a: Inference for Single Proportion}
\date[\today]{\footnotesize \today}
\author{{\bf Jimi Oke}}
\institute[UMass Amherst]{
  \begin{tikzpicture}[baseline=(current bounding box.center)]
    \node[anchor=base] at (-7,0) (its) {\includegraphics[scale=.3]{UMassEngineering_vert}} ;
  \end{tikzpicture}
}


    
\begin{document}

\maketitle




\begin{frame}
  \frametitle{Outline}
  \tableofcontents
\end{frame}

\begin{frame}
  \frametitle{Module objectives}
  \pause

  Inference refers to various ways in which we obtain information and make decisions based on data. \pause
  In this module, we consider: (a) \textbf{point estimates}, (b) \textbf{confidence intervals} and 
  (c) \textbf{hypothesis tests} for
  \begin{itemize}
  \item Single proportions
    \pause
    
  \item Difference of two proportions
    \pause

  \item Contigency tables
  \end{itemize}
\end{frame}


\begin{frame}
  \frametitle{Today's objectives}
  \begin{itemize}
  \item Test for normality \pause
  \item Compute CIs (confidence intervals) \pause
  \item Conduct hypothesis tests
  \item Calculate sample size
  \end{itemize}
\end{frame}

\section{Test for normality}

\begin{frame}
  \frametitle{Success-failure condition}\pause
  Given a sample proportion $\hat p$ of size $n$ whose observations are independent. \pause
  We can assume the distribution for $p$ is normal if 
  \begin{itemize}
  \item $np \geq 10$ and

  \item $n(1-p) \geq 10 $
\end{itemize}

\end{frame}

\begin{frame}
  \frametitle{Example 1}
  \pause
  A simple random sample of 826 payday loan borrowers was surveyed to better understand their
interests around regulation and costs. 70\% of the responses supported new regulations on payday
lenders. Is it reasonable to model $\hat{p}$ = 0.70 using a normal distribution?
\end{frame}

\section{CI for  proportion}
\begin{frame}
  \frametitle{CI for proportion}
  \pause
  The confidence interval (CI) for a proportion is given by: \pause
  \begin{equation}
    \hat p \pm z^{*} \times SE \equiv \hat p \pm ME
  \end{equation}

  \pause
  where:
  \begin{eqnarray}
        z^{*} &=& \Phi \lt( 1  - \fr{\alpha}{2} \rt)  \quad \text{\bf critical value}\\\pause
    SE &=& \sqrt{\fr{p(1-p)}{n}} \quad \text{\bf standard error}\\\pause
    ME &=& z^{*} \times SE \quad \text{\bf margin of error}
  \end{eqnarray}
  \pause

  Note that $p$ is unknown, so we use $\hat p$ in its place when computing the standard error.
\end{frame}

\begin{frame}
  \frametitle{Confidence interval (visualization)}

  \begin{center}
  \begin{tikzpicture}[scale=0.8]
  \begin{axis}[
      no markers, 
      domain=0.60:0.80, 
      samples=200,
      axis x line=center,
      axis y line=left,
      xlabel=$\hat{p}$, 
      ylabel=Density,
      height=5cm, 
      width=10cm,
      xtick={0.647, 0.70, 0.753},
      xticklabels={$\hat{p} - ME$, $\hat{p}$, $\hat{p} + ME$},
      ymax=15,
      ytick=\empty,
      x label style={anchor=west},
      y label style={anchor=south},
      enlargelimits=false, 
      clip=false, 
      axis on top,
      tick style={very thick, major tick length=7pt}
  ]

  % Main distribution curve
  \addplot [blue, thick, domain=0.60:0.80] {gauss(0.70, 0.016)};

  % Confidence interval shading
  \addplot [blue!30, fill=blue!30, domain=0.647:0.753] {gauss(0.70, 0.016)} \closedcycle;

  % Tail areas
  \addplot [red!30, fill=red!30, domain=0.60:0.647] {gauss(0.70, 0.016)} \closedcycle;
  \addplot [red!30, fill=red!30, domain=0.753:0.80] {gauss(0.70, 0.016)} \closedcycle;

  % Vertical lines for confidence interval bounds
  \addplot [red, thick, dashed] coordinates {(0.647, 0) (0.647, 10)};
  \addplot [red, thick, dashed] coordinates {(0.753, 0) (0.753, 10)};

  % Mean line
  \addplot [black, thick] coordinates {(0.70, 0) (0.70, 24.9)};

  % Margin of error arrows and labels
  \draw [<->, thick, red] (axis cs: 0.647, 1) -- (axis cs: 0.70, 1);
  \node at (axis cs: 0.6735, 1.2) {\textcolor{red}{\small ME}};
  
  \draw [<->, thick, red] (axis cs: 0.70, 1) -- (axis cs: 0.753, 1);
  \node at (axis cs: 0.7265, 1.2) {\textcolor{red}{\small ME}};

  % Labels
  \node at (axis cs: 0.70, 22) {$\hat{p}$};
  \node at (axis cs: 0.70, 12) {\textcolor{blue}{95\% CI}};
  \node at (axis cs: 0.625, 3) {\textcolor{red}{\tiny 2.5\%}};
  \node at (axis cs: 0.775, 3) {\textcolor{red}{\tiny 2.5\%}};

  \end{axis}
  \end{tikzpicture}
  \end{center}

  \pause
  \begin{itemize}
  \item The confidence interval is symmetric around $\hat{p}$
  \item Margin of error (ME) extends equally in both directions
  \item CI = $[\hat{p} - ME, \hat{p} + ME]$
  \end{itemize}

\end{frame}
\begin{frame}
  \frametitle{Example 1: Elderly drivers}
  \pause
 The Marist Poll published a report stating that 66\% of adults nationally think licensed
drivers should be required to retake their road test once they reach 65 years of age. 
It was also reported
that interviews were conducted on 1,018 American adults.

\begin{enumerate}[\bf(a)]
  \item Verify that the success-failure condition is met. \pause
  \item Find the standard error for the sample proportion. \pause
  \item Find the critical value for a 95\% confidence interval. \pause
  \item Find the margin of error for the confidence interval. \pause
  \item Construct a 95\% confidence interval for $p$, the proportion of adults
\end{enumerate}
\end{frame}

\begin{frame}
  \frametitle{Example 1 (cont.)}

  \begin{enumerate}[\bf(a)]
    \item Success-failure condition:
      \begin{eqnarray*}
      np &=& 1018 \times 0.66 = 671.88 \geq 10 \quad \checkmark\\\pause
      n(1-p) &=& 1018 \times (1-0.66) = 346.12 \geq 10 \quad \checkmark
      \end{eqnarray*} \pause
    \item Standard error: \pause
    \begin{eqnarray*}
      SE &=& \sqrt{\fr{p(1-p)}{n}} = \sqrt{\fr{0.66(1-0.66)}{1018}} = 0.0149 \pause
    \end{eqnarray*}
    \item Critical value for 95\% CI: \pause
      \begin{eqnarray*}
      z^{*} &=& \Phi\lt(1 - \fr{0.05}{2}\rt) = \Phi(0.975) = 1.96 \quad 
      \texttt{\bl norm.cdf(0.975)} \pause
      \end{eqnarray*} \pause
    \item Margin of error:\pause
      \begin{eqnarray*}   
      ME &=& z^{*} \times SE = 1.96 \times 0.0149 = 0.0292 \approx 3\% \pause
      \end{eqnarray*}
    \item Confidence interval:\pause
      \begin{eqnarray*}
      CI &=&  [\hat p - ME, \hat p + ME] = [0.66 - 0.0292, 0.66 + 0.0292] = [0.6308, 0.6892]
      \end{eqnarray*}
  \end{enumerate}

\end{frame}

\begin{frame}
  \frametitle{Example 1 (cont.)}

\begin{center}
\begin{tikzpicture}[scale=0.8]
\begin{axis}[
    no markers, 
    domain=0.58:0.74, 
    samples=200,
    axis x line=center,
    axis y line=left,
    xlabel=$\hat{p}$, 
    ylabel=Density,
    height=5cm, 
    width=10cm,
    xtick={0.631, 0.66, 0.689},
    xticklabels={$0.631$, $0.66$, $0.689$},
    ymax=18,
    ytick=\empty,
    x label style={anchor=west},
    y label style={anchor=south},
    enlargelimits=false, 
    clip=false, 
    axis on top,
    tick style={very thick, major tick length=7pt}
]

% Main distribution curve
\addplot [blue, thick, domain=0.58:0.74] {gauss(0.66, 0.015)};

% Confidence interval shading
\addplot [blue!30, fill=blue!30, domain=0.631:0.689] {gauss(0.66, 0.015)} \closedcycle;

% Tail areas
\addplot [red!30, fill=red!30, domain=0.58:0.631] {gauss(0.66, 0.015)} \closedcycle;
\addplot [red!30, fill=red!30, domain=0.689:0.74] {gauss(0.66, 0.015)} \closedcycle;

% Vertical lines for confidence interval bounds
\addplot [red, thick, dashed] coordinates {(0.631, 0) (0.631, 12)};
\addplot [red, thick, dashed] coordinates {(0.689, 0) (0.689, 12)};

% Mean line
\addplot [black, thick] coordinates {(0.66, 0) (0.66, 26.6)};

% Margin of error arrows and labels
\draw [<->, thick, red] (axis cs: 0.631, -1.2) -- (axis cs: 0.66, -1.2);
\node at (axis cs: 0.6455, -2.5) {\textcolor{red}{\footnotesize .029}};

\draw [<->, thick, red] (axis cs: 0.66, -1.2) -- (axis cs: 0.689, -1.2);
\node at (axis cs: 0.6745, -2.5) {\textcolor{red}{\footnotesize .029}};

% Labels
\node at (axis cs: 0.66, 24) {$\hat{p} = 0.66$};
\node at (axis cs: 0.66, 14) {\textcolor{blue}{\footnotesize 95\% CI: [0.631, 0.689]}};
\node at (axis cs: 0.605, 3) {\textcolor{red}{\tiny 2.5\%}};
\node at (axis cs: 0.715, 3) {\textcolor{red}{\tiny 2.5\%}};

\end{axis}
\end{tikzpicture}
\end{center}

\pause
\textbf{Conclusion:} We are 95\% confident that between 63.1\% and 68.9\% of American adults think licensed drivers should be required to retake their road test once they reach 65 years of age.

\end{frame}



\section{Hypothesis testing}

\begin{frame}
  \frametitle{Summary of hypothesis testing approach}\pause

  \begin{enumerate}[<+->]
  \item \textit{\gr Define} the \textbf{null} ($H_0$) and \textbf{alternative} ($H_1$) hypotheses
  \item \textit{\gr Determine} the appropriate \textbf{test statistic} (and distribution)
  \item \textit{\gr Estimate} the test statistic from the sample data
  \item \textit{\gr Specify} or \textit{\gr identify} the \textbf{level of significance} ($\alpha$)
  \item \textit{\gr Define} the \textbf{region of rejection/critical region} of the null hypothesis by choosing the \textbf{critical value}.
  \item \textit{\gr Decide.} If the test statistic is in the critical region, reject $H_0$. If not, do not reject $H_0$ (fail to reject it)
  \end{enumerate}
\end{frame}

\begin{frame}
  \frametitle{Distribution of the test statistic}
  \pause
  In this lecture, the test statistic is the \textbf{sample proportion}.\pause

  We will assume the normal distribution is the success-failure condition holds.\pause

  \begin{block}{}\pause
    The sample proportion is \textbf{normally} distributed and its variance  is :
    \begin{equation}
      \label{eq:20}
      \mathbb{V}(p) = \fr{p(1-p)}{n} 
    \end{equation}
    And thus, the standard error is:
    \pause
    \begin{equation}
      SE_{\hat{p}} = \sqrt{ \fr{p_{0}(1-p_{0})}{n} }
    \end{equation}
    Thus, to compute the probability (area under curve) of the test statistic, \pause
    we use the z-score:
    \begin{equation}
    \bl z = \fr{p - p_{0}}{SE_{p}}
  \end{equation}
  which is {\bl normally} distributed.
  \end{block}
\end{frame}


\begin{frame}
  \frametitle{Two-sided tests}\pause
  {\bf Case A: both tails}\pause

  \begin{itemize}
  \item $H_0: p = p_0$
  \item $H_1: p \ne p_0$
  \end{itemize}

    \pause
  
  \begin{tikzpicture}
    \begin{axis}[no markers, domain=0:10, samples=100,
      axis x line=center,
      axis y line=none,
      xlabel=$x$, ylabel=$f_X(x)$,,
      height=6cm, width=10cm,
      xtick={-6,0,6},
      xticklabels={$z_{\frac{\alpha}{2}}$,$z$,$z_{1 - \frac{\alpha}{2}}$},
      ymax=.15,
      ytick=\empty,
      x label style={anchor=west},
      y label style={anchor=south},
      enlargelimits=true, clip=false, axis on top
      %grid style={line width=.1pt, draw=gray},
      % yticklabel style={
      %   /pgf/number format/fixed,
      %   /pgf/number format/fixed zerofill,
      %   /pgf/number format/precision=2
      % },        
      %   grid = major
      ]
      \addplot [blue, domain=-10:10] {gauss(0,3)};
      \addplot [gray, fill=gray!50, domain=-6:6] {gauss(0,3)} \closedcycle;
      \addplot [orange,fill=orange,  domain=6:10] {gauss(0,3)} \closedcycle;
      \addplot [orange,fill=orange,  domain=-10:-6] {gauss(0,3)} \closedcycle;
      \node (d) at (axis cs: -6,.1) {Area: $1-\alpha$};
      \draw[thick, ->] (d) -- (axis cs: -.6,.05);
      \node (c) at (axis cs: 8.5,.05) {Area: $\fr\alpha2$};
      \draw[thick,->] (c) -- (axis cs: 6.5, 0.003);
      \node (e) at (axis cs: -8.5,.05) {Area: $\fr\alpha2$};
      \draw[thick,->] (e) -- (axis cs: -6.5, 0.003);
      \draw[thick, |->] (axis cs: 6,-0.025) -- (axis cs: 10,-0.025) node[below,pos=.5] {\small\og Reject $H_0$};
      \draw[thick, |->] (axis cs: -6,-0.025) -- (axis cs: -10,-0.025) node[below,pos=.5] {\small\og Reject $H_0$};
    \end{axis}
  \end{tikzpicture}

  
\end{frame}


\begin{frame}
  \frametitle{One-sided tests}\pause
  {\bf Case B: upper tail}\pause

  \begin{itemize}
  \item $H_0: p = p_0$
  \item $H_1: p > p_0$
  \end{itemize}

  \pause
  
  \begin{tikzpicture}
    \begin{axis}[no markers, domain=0:10, samples=100,
      axis x line=center,
      axis y line=none,
      xlabel=$x$, ylabel=$f_X(x)$,,
      height=6cm, width=10cm,
      xtick={0,6},
      xticklabels={$z$,$z_{1- \alpha/2}$},
      ymax=.15,
      ytick=\empty,
      x label style={anchor=west},
      y label style={anchor=south},
      enlargelimits=true, clip=false, axis on top
      %grid style={line width=.1pt, draw=gray},
      % yticklabel style={
      %   /pgf/number format/fixed,
      %   /pgf/number format/fixed zerofill,
      %   /pgf/number format/precision=2
      % },        
      %   grid = major
      ]
      \addplot [blue, domain=-10:10] {gauss(0,3)};
      \addplot [gray, fill=gray!50, domain=-10:6] {gauss(0,3)} \closedcycle;
      \addplot [orange,fill=orange,  domain=6:10] {gauss(0,3)} \closedcycle;
      \node (d) at (axis cs: -6,.1) {Area: $1-\alpha$};
      \draw[thick, ->] (d) -- (axis cs: -.6,.05);
      \node (c) at (axis cs: 8.5,.05) {Area: $\alpha$};
      \draw[thick,->] (c) -- (axis cs: 6.5, 0.003);
      \draw[thick, |->] (axis cs: 6,-0.025) -- (axis cs: 10,-0.025) node[below,pos=.5] {\small\og Reject $H_0$};
    \end{axis}
  \end{tikzpicture}

\end{frame}


\begin{frame}
  \frametitle{One-sided tests (cont.)}\pause
  {\bf Case C: lower tail}\pause

  \begin{itemize}
  \item $H_0: p = p_0$
  \item $H_1: p < p_0$
  \end{itemize}

    \pause
  
  \begin{tikzpicture}
    \begin{axis}[no markers, domain=0:10, samples=100,
      axis x line=center,
      axis y line=none,
      xlabel=$x$, ylabel=$f_X(x)$,,
      height=6cm, width=10cm,
      xtick={-6,0},
      xticklabels={$z_{\alpha/2}$,$z$},
      ymax=.15,
      ytick=\empty,
      x label style={anchor=west},
      y label style={anchor=south},
      enlargelimits=true, clip=false, axis on top
      %grid style={line width=.1pt, draw=gray},
      % yticklabel style={
      %   /pgf/number format/fixed,
      %   /pgf/number format/fixed zerofill,
      %   /pgf/number format/precision=2
      % },        
      %   grid = major
      ]
      \addplot [blue, domain=-10:10] {gauss(0,3)};
      \addplot [gray, fill=gray!50, domain=-6:10] {gauss(0,3)} \closedcycle;
      \addplot [orange,fill=orange,  domain=-10:-6] {gauss(0,3)} \closedcycle;
      \node (d) at (axis cs: 6,.1) {Area: $1-\alpha$};
      \draw[thick, ->] (d) -- (axis cs: .6,.05);
      \node (c) at (axis cs: -8.5,.05) {Area: $\alpha$};
      \draw[thick,->] (c) -- (axis cs: -6.5, 0.003);
      \draw[thick, |->] (axis cs: -6,-0.025) -- (axis cs: -10,-0.025) node[below,pos=.5] {\small\og Reject $H_0$};
    \end{axis}
  \end{tikzpicture}

\end{frame}


\begin{frame}
  \frametitle{Example 2: Getting enough sleep (OS 5.21)}

    400 students were randomly sampled from a large university, and 289 said they did not get enough sleep. Conduct a
    hypothesis test to check whether this represents a statistically significant difference from 50\%, and use a
    significance level of 0.01

    \begin{enumerate}[Step 1.]
    \item Parameter of interest: \pause $p$ (proportion of students not getting enough sleep) \pause
    \item Null hypothesis: \pause $H_0: p = 289/400 = .723$. \pause
    \item Alternative hypothesis: \pause $H_1: p \ne .723$. \pause
    \item Formula for test statistic value: \pause $z = \fr{p - p_0}{SE_{p}}$ \pause
    \end{enumerate}
\end{frame}

\begin{frame}
  \frametitle{Example 2:  Getting enough sleep  (cont.)}
    \begin{enumerate}[Step 1.]\setcounter{enumi}{4}\pause
    \item Calculate test statistic value: \pause
      \begin{equation*}
        z = \fr{.723  - .5}{\sqrt{.5(.5)/400}}  \pause = 8.92
      \end{equation*}
      \pause

    \item Compute critical value at $\alpha = 0.01$: \pause
      \begin{equation*}
        z_{\alpha/2} = \Phi^{-1}(1 - \alpha/2) = \Phi^{-1}(0.995) = 2.576 \quad 
        \texttt{\bl norm.ppf(0.995)}
      \end{equation*}
      \pause

      \item Compare test statistic to critical value: \pause
      \begin{equation*}
        |z| = 8.92 > 2.576 = z_{\alpha/2}
      \end{equation*}
      \pause
    \item Conclude: \pause

      Using a significance level of 0.01, we  reject $H_0$ since $8.92 > 2.576$.
      Thus, at the 1\% significance level, there is sufficient evidence to conclude that true proportion
      differs from the target value of 0.5.
    \end{enumerate}
\end{frame}


\begin{frame}
  \frametitle{Example 2:  Getting enough sleep  (cont.)}

\begin{center}
\begin{tikzpicture}[scale=0.9]
\begin{axis}[
    no markers, 
    domain=-10:10, 
    samples=200,
    axis x line=center,
    axis y line=left,
    xlabel=$z$, 
    ylabel=Density,
    height=6cm, 
    width=12cm,
    xtick={-8.92, -2.576, 0, 2.576, 8.92},
    xticklabels={$-8.92$, $-z_{\alpha/2}=-2.576$, $0$, $z_{\alpha/2}=2.576$, $z=8.92$},
    ymax=0.45,
    ytick=\empty,
    x label style={anchor=west},
    y label style={anchor=south},
    enlargelimits=false, 
    clip=false, 
    axis on top,
    tick style={very thick, major tick length=7pt},
    xticklabel style={rotate=45, anchor=east}
]

% Main standard normal distribution curve
\addplot [blue, thick, domain=-10:10] {gauss(0, 1)};

% Critical regions (rejection regions) - both tails
\addplot [red!40, fill=red!40, domain=-10:-2.576] {gauss(0, 1)} \closedcycle;
\addplot [red!40, fill=red!40, domain=2.576:10] {gauss(0, 1)} \closedcycle;

% Acceptance region
\addplot [green!30, fill=green!30, domain=-2.576:2.576] {gauss(0, 1)} \closedcycle;

% Vertical lines for critical values
\addplot [red, thick, dashed] coordinates {(-2.576, 0) (-2.576, 0.13)};
\addplot [red, thick, dashed] coordinates {(2.576, 0) (2.576, 0.13)};

% Vertical line for test statistic (way off the chart but we'll show it symbolically)
\addplot [orange, very thick] coordinates {(8.92, 0) (8.92, 0.05)};
\addplot [orange, very thick] coordinates {(-8.92, 0) (-8.92, 0.05)};

% Labels for regions
\node at (axis cs: 0, 0.25) {\textcolor{green!70!black}{\footnotesize Accept $H_0$}};
\node at (axis cs: -5, 0.02) {\textcolor{red}{\tiny Reject $H_0$}};
\node at (axis cs: 5, 0.02) {\textcolor{red}{\tiny Reject $H_0$}};

% Alpha levels
\node at (axis cs: -4, 0.08) {\textcolor{red}{\tiny $\alpha/2 = 0.005$}};
\node at (axis cs: 4, 0.08) {\textcolor{red}{\tiny $\alpha/2 = 0.005$}};

% Test statistic labels
\node at (axis cs: 8.92, 0.15) {\textcolor{orange}{\footnotesize\textbf{Test Statistic}}};
\node at (axis cs: 8.92, 0.12) {\textcolor{orange}{\footnotesize $z = 8.92$}};

% Arrow pointing to test statistic location
\draw [->, thick, orange] (axis cs: 6, 0.3) -- (axis cs: 8.5, 0.08);
\node at (axis cs: 5.5, 0.32) {\textcolor{orange}{\footnotesize Test statistic}};
\node at (axis cs: 5.5, 0.29) {\textcolor{orange}{\footnotesize falls in}};
\node at (axis cs: 5.5, 0.26) {\textcolor{orange}{\footnotesize rejection region}};

\end{axis}
\end{tikzpicture}
\end{center}

\pause
\textbf{Decision:} Since $|z| = 8.92 > 2.576 = z_{\alpha/2}$, we \textcolor{red}{\textbf{reject}} $H_0$ at $\alpha = 0.01$ significance level.

\end{frame}



\begin{frame}
  \frametitle{Example 3: Equipment quality}\pause

  A car manufacturer is considering switching to a new, higher quality piece of equipment that con-
structs vehicle door hinges. 
They figure that they will save money in the long run if this new machine
produces hinges that have flaws no more than 0.2\% of the time.
A random sample of 800 hinges produced by the new machine shows that 4 are flawed.
At the 0.05 significance level, is there enough evidence to conclude that the new machine
produces hinges with a flaw rate greater than 0.2\%?

\end{frame}

\begin{frame}
  \frametitle{Example 3 (cont.)}\pause

This is a one-sided test \pause

\pause
\textbf{Hypotheses:}
\begin{itemize}
  \item $H_0: p = 0.002$
  \item $H_a: p < 0.002$
\end{itemize}

\pause
\textbf{Test Statistic:}
\begin{eqnarray*}
\hat{p} &=& \frac{4}{800} = 0.005 \\ \pause
SE &=& \sqrt{\frac{0.002(1-0.002)}{800}} \approx 0.0016 \\\pause
z &=& \frac{\hat{p} - 0.002}{SE} \approx 1.875
\end{eqnarray*}

\pause
\textbf{Critical Value:} For $\alpha = 0.05$, the critical value is:
\begin{eqnarray*}
z_{\alpha} &=& \Phi^{-1}(1 - \alpha) = \Phi^{-1}(0.95) = 1.645 \quad 
\texttt{\bl norm.ppf(0.95)}
\end{eqnarray*}
\end{frame}



\begin{frame}
  \frametitle{Example 3 (cont.)}\pause

\begin{center}
\begin{tikzpicture}[scale=0.9]
\begin{axis}[
    no markers, 
    domain=-4:4, 
    samples=200,
    axis x line=center,
    axis y line=left,
    xlabel=$z$, 
    ylabel=Density,
    height=5cm, 
    width=10cm,
    xtick={0, 1.645, 1.875},
    xticklabels={$0$, $z_{\alpha}$, $z$},
    ymax=0.45,
    ytick=\empty,
    x label style={anchor=west},
    y label style={anchor=south},
    enlargelimits=false, 
    clip=false, 
    axis on top,
    tick style={very thick, major tick length=7pt}
]

% Main standard normal distribution curve
\addplot [blue, thick, domain=-4:4] {gauss(0, 1)};

% Critical region (right tail only - one-tailed test)
\addplot [red!40, fill=red!40, domain=1.645:4] {gauss(0, 1)} \closedcycle;

% Acceptance region
\addplot [green!30, fill=green!30, domain=-4:1.875] {gauss(0, 1)} \closedcycle;

% Vertical line for critical value
\addplot [red, thick, dashed] coordinates {(1.645, 0) (1.645, 0.23)};

% Vertical line for test statistic
\addplot [orange, very thick] coordinates {(1.875, 0) (1.875, 0.2)};

% Labels for regions
\node at (axis cs: -1, 0.25) {\textcolor{green!70!black}{\footnotesize Fail to Reject $H_0$}};
\node at (axis cs: 2.5, 0.08) {\textcolor{red}{\footnotesize Reject $H_0$}};

% Alpha level
\node at (axis cs: 2.2, 0.15) {\textcolor{red}{\tiny $\alpha = 0.05$}};

% Test statistic label
\node at (axis cs: 1.875, 0.25) {\textcolor{orange}{\footnotesize $z = 1.875$}};

% Decision arrow and text
\draw [->, thick, blue] (axis cs: 1.2, 0.35) -- (axis cs: 1.875, 0.25);
\node at (axis cs: 0.8, 0.38) {\textcolor{blue}{\footnotesize Test statistic falls}};
\node at (axis cs: 0.8, 0.35) {\textcolor{blue}{\footnotesize in acceptance region}};

\end{axis}
\end{tikzpicture}
\end{center}

\pause
\textbf{Decision:} Since $z = 1.875 < 1.645 = z_{\alpha}$, we \textcolor{red}{\textbf{fail to reject}} $H_0$ at $\alpha = 0.05$ significance level.


  

\end{frame}

\section{Using p-values}



\begin{frame}
  \frametitle{$p$-value}\pause
      The $p$-value is the probability of obtaining a test statistic value at least as contradictory to $H_0$ as the value that actually resulted.
    \pause
    \alert{The smaller the $p$-value, the more contradictory are the data to $H_0$.}

  \pause

  \begin{block}{Usefulness of p-value}\pause
  \begin{itemize}[<+->]
  \item Provides more information about the strength of a test
  \item Indicates the smallest level at which the data is significant
  \item Can be compared with $\alpha$ irrespective of which type of test was used
  \end{itemize}
  \end{block}

  \pause



    \begin{enumerate}[Step 1.]
  \item Formulate your hypotheses
  \item Determine the $p$-value from the test statistic
  \item Conclude the test based on a chosen level of significance:
    \begin{enumerate}
    \item $p$-value $\le \alpha \implies$ reject $H_0$ at level $\alpha$.
    \item $p$-value $> \alpha \implies$ do not reject $H_0$ at level $\alpha$.
    \end{enumerate}
  \end{enumerate}
\end{frame}



\begin{frame}
  \frametitle{$p$-value for $z$ tests}\pause

  \begin{minipage}{.6\linewidth}
  \begin{tikzpicture}[scale=.7]
    \begin{axis}[no markers, domain=0:10, samples=100,
      axis x line=center,
      axis y line=none,
      xlabel=$Z$, ylabel=$f_X(x)$,,
      height=4cm, width=10cm,
      xtick={0,6},
      xticklabels={$0$,$z$},
      ymax=.15,
      ytick=\empty,
      x label style={anchor=west},
      y label style={anchor=south},
      enlargelimits=true, clip=false, axis on top
      %grid style={line width=.1pt, draw=gray},
      % yticklabel style={
      %   /pgf/number format/fixed,
      %   /pgf/number format/fixed zerofill,
      %   /pgf/number format/precision=2
      % },        
      %   grid = major
      ]
      \addplot [blue, domain=-10:10] {gauss(0,3)};
      \addplot [gray, fill=gray!50, domain=-10:6] {gauss(0,3)} \closedcycle;
      \addplot [orange,fill=orange,  domain=6:10] {gauss(0,3)} \closedcycle;
      \node (c) at (axis cs: 8.5,.05) {Area: $p$-value};
      \draw[thick,->] (c) -- (axis cs: 6.5, 0.003);
      % \draw[thick, |->] (axis cs: 6,-0.025) -- (axis cs: 10,-0.025) node[below,pos=.5] {\small\og Reject $H_0$};
    \end{axis}
  \end{tikzpicture}
\end{minipage}
\begin{minipage}{.35\linewidth}
  \begin{block}{$p$-value: area in upper tail}\pause
  \begin{equation}
    p  = 1 - \Phi(z)\label{eq:41}
  \end{equation}
\end{block}

\end{minipage}

\pause

\bigskip

\begin{minipage}{.6\linewidth}
  \begin{tikzpicture}[scale=.7]
    \begin{axis}[no markers, domain=0:10, samples=100,
      axis x line=center,
      axis y line=none,
      xlabel=$Z$, ylabel=$f_X(x)$,,
      height=4cm, width=10cm,
      xtick={-6,0},
      xticklabels={$z$,$0$},
      ymax=.15,
      ytick=\empty,
      x label style={anchor=west},
      y label style={anchor=south},
      enlargelimits=true, clip=false, axis on top
      ]
      \addplot [blue, domain=-10:10] {gauss(0,3)};
      \addplot [gray, fill=gray!50, domain=-6:10] {gauss(0,3)} \closedcycle;
      \addplot [orange,fill=orange,  domain=-10:-6] {gauss(0,3)} \closedcycle;
      %\node (d) at (axis cs: 6,.1) {Area: $1-\alpha$};
      %\draw[thick, ->] (d) -- (axis cs: .6,.05);
      \node (c) at (axis cs: -8.5,.05) {Area: $p$-value};
      \draw[thick,->] (c) -- (axis cs: -6.5, 0.003);
    \end{axis}
  \end{tikzpicture}
\end{minipage}
\begin{minipage}{.35\linewidth}
  \begin{block}{$p$-value: area in lower tail} \pause
    \begin{equation}
    p  = \Phi(z)\label{eq:42}
  \end{equation}
\end{block}

  
\end{minipage}

\pause
\bigskip

\begin{minipage}{.6\linewidth}
  \begin{tikzpicture}[scale=0.7]
    \begin{axis}[no markers, domain=0:10, samples=100,
      axis x line=center,
      axis y line=none,
      xlabel=$Z$, ylabel=$f_X(x)$,,
      height=4cm, width=10cm,
      xtick={-6,0,6},
      xticklabels={$-z$,$0$,$z$},
      ymax=.15,
      ytick=\empty,
      x label style={anchor=west},
      y label style={anchor=south},
      enlargelimits=true, clip=false, axis on top
      %grid style={line width=.1pt, draw=gray},
      % yticklabel style={
      %   /pgf/number format/fixed,
      %   /pgf/number format/fixed zerofill,
      %   /pgf/number format/precision=2
      % },        
      %   grid = major
      ]
      \addplot [blue, domain=-10:10] {gauss(0,3)};
      \addplot [gray, fill=gray!50, domain=-6:6] {gauss(0,3)} \closedcycle;
      \addplot [orange,fill=orange,  domain=6:10] {gauss(0,3)} \closedcycle;
      \addplot [orange,fill=orange,  domain=-10:-6] {gauss(0,3)} \closedcycle;
      \node (c) at (axis cs: 8.5,.05) {Area: $0.5p$-value};
      \draw[thick,->] (c) -- (axis cs: 6.5, 0.003);
      \node (e) at (axis cs: -8.5,.05) {Area: $0.5p$-value};
      \draw[thick,->] (e) -- (axis cs: -6.5, 0.003);
    \end{axis}
  \end{tikzpicture}
\end{minipage}
\begin{minipage}{.35\linewidth}
  \begin{block}{$p$-value: sum of area in both tails} \pause
  \begin{equation}
    \label{eq:43}
    p = 2 (1 - \Phi(|z|))
  \end{equation}
\end{block}

\end{minipage}
\end{frame}



\begin{frame}
  \frametitle{Hypothesis testing using $p$-value approach}
  \begin{exampleblock}{Example 4: Getting enough sleep (OS 5.21)}

    400 students were randomly sampled from a large university, and 289 said they did not get enough sleep. Conduct a
    hypothesis test to check whether this represents a statistically significant difference from 50\%, and use a
    significance level of 0.01

    \begin{enumerate}[Step 1.]
    \item Parameter of interest: \pause $p$ (proportion of students not getting enough sleep) \pause
    \item Null hypothesis: \pause $H_0: \hat{p} = 289/400 = .723$. \pause
    \item Alternative hypothesis: \pause $H_1: p \ne .723$. \pause
    \item Formula for test statistic value: \pause $z = \fr{p - p_0}{SE_{p}}$ \pause
    \end{enumerate}
  \end{exampleblock}
\end{frame}

\begin{frame}
  \frametitle{Hypothesis testing using $p$-value approach}
  \begin{exampleblock}{Example 4:  Getting enough sleep  (cont.)}
    \begin{enumerate}[Step 1.]\setcounter{enumi}{4}\pause
    \item Calculate test statistic value: \pause
      \begin{equation*}
        z = \fr{.723  - .5}{\sqrt{.5(.5)/400}}  \pause = 8.92
      \end{equation*}
      \pause

    \item Determine $p$-value \pause (two-tailed test): \pause
      \begin{equation*}
        p\text{-value} = 2(1 - \Phi(8.92)) \pause = 0.0
      \end{equation*}
      \pause

      \item  Compare $p$-value to $\alpha = 0.01$: \pause
      \begin{equation*}
        0.0 \le 0.01
      \end{equation*}
      \pause

    \item Conclude: \pause

      Using a significance level of 0.01, we  reject $H_0$ since $0.0<0.01$.
    \end{enumerate}
  \end{exampleblock}
\end{frame}




\begin{frame}
  \frametitle{Example 5: Equipment quality (p-value approach)}\pause

  A car manufacturer is considering switching to a new, higher quality piece of equipment that con-
structs vehicle door hinges. 
They figure that they will save money in the long run if this new machine
produces hinges that have flaws no more than 0.2\% of the time.
A random sample of 800 hinges produced by the new machine shows that 4 are flawed.
At the 0.05 significance level, is there enough evidence to conclude that the new machine
produces hinges with a flaw rate greater than 0.2\%?

\end{frame}

\begin{frame}
  \frametitle{Example 5 (cont.)}\pause

This is a one-sided test \pause

\pause
\textbf{Hypotheses:}
\begin{itemize}
  \item $H_0: p = 0.002$
  \item $H_a: p < 0.002$
\end{itemize}

\pause
\textbf{Test Statistic:}
\begin{eqnarray*}
\hat{p} &=& \frac{4}{800} = 0.005 \\ \pause
SE &=& \sqrt{\frac{0.002(1-0.002)}{800}} \approx 0.0016 \\\pause
z &=& \frac{\hat{p} - 0.002}{SE} \approx 1.875
\end{eqnarray*}

\pause
\textbf{p-value:} For $z = 1.875$, the $p$-value is:
\begin{eqnarray*}
p\text{-value} &=& 1 - \Phi(1.875) \approx 0.0304 \quad
\texttt{\bl 1 - norm.cdf(1.875)}
\end{eqnarray*}
\end{frame}


\begin{frame}
  \frametitle{Example 5 (cont.)}\pause

\begin{center}
\begin{tikzpicture}[scale=0.9]
\begin{axis}[
    no markers, 
    domain=-4:4, 
    samples=200,
    axis x line=center,
    axis y line=left,
    xlabel=$z$, 
    ylabel=Density,
    height=5cm, 
    width=10cm,
    xtick={0, 1.875},
    xticklabels={$0$, $z=1.875$},
    ymax=0.45,
    ytick=\empty,
    x label style={anchor=west},
    y label style={anchor=south},
    enlargelimits=false, 
    clip=false, 
    axis on top,
    tick style={very thick, major tick length=7pt}
]

% Main standard normal distribution curve
\addplot [blue, thick, domain=-4:4] {gauss(0, 1)};

% p-value region (right tail for one-sided test)
\addplot [orange!40, fill=orange!40, domain=1.875:4] {gauss(0, 1)} \closedcycle;

\addplot [red!40, fill=red!40, domain=1.96:4] {gauss(0, 1)} \closedcycle;

% Vertical line for test statistic
\addplot [orange, very thick] coordinates {(1.875, 0) (1.875, 0.2)};

% Labels
\node at (axis cs: 1.875, 0.35) {\textcolor{orange}{\footnotesize $z = 1.875$}};
\node at (axis cs: 2.5, 0.08) {\textcolor{orange}{\footnotesize p-value = 0.030}};

% Decision comparison box
\node[draw, fill=white, align=center] at (axis cs: -1.5, 0.35) {
  \textbf{Decision:} \\
  $p\text{-value} = 0.03$ \\
  $\alpha = 0.05$ \\
  \textcolor{red}{$0.03 < 0.05$} \\
  \textbf{Fail to Reject } $H_0$
};

\end{axis}
\end{tikzpicture}
\end{center}

\pause
\textbf{Decision:} Since the p-value $= 0.03 < \alpha = 0.05$, we \textcolor{red}{\textbf{fail to reject}} $H_0$ at the 5\% significance level.

\pause
\textbf{Conclusion:} There is insufficient evidence to conclude that the new machine produces hinges with a flaw rate greater than 0.2\%.

\end{frame}

\section{Outlook}
\begin{frame}
  \frametitle{Recap of this lecture }
  \begin{itemize}
  \item Definition of hypothesis testing
    \begin{itemize}
    \item Null hypothesis (default/expected outcome) $H_{0}$ \pause
    \item Alternate hypothesis (what we want to test/support; research hypothesis) $H_{1}$ or $H_{A}$\pause
    \item One-tailed or two-tailed
    \end{itemize}
  \item Types of errors:
    \begin{itemize}
    \item Type I: false positive
    \item Type II: false negative
    \end{itemize}
  \item Test statistic:
    \begin{itemize}
    \item Sample proportion with independent observations and large enough sample size (normal distribution); \pause
      $Z$-statistic: \pause

      \begin{equation}
     z= \fr{p - p_{0}}{\sqrt{\fr{p_{0}(1-p_{0})}{n} }}
    \end{equation}
    \pause
    \end{itemize}

  \item The $p$-value is the minimum probability of a Type I error.\pause
    \begin{itemize}
    \item Upper-tailed test: $p-\text{value} = 1- \Phi(z)$; \pause Python: \texttt{\bl norm.sf(z)} \pause
    \item Lower-tailed test: $p-\text{value} = \Phi(z)$; \pause Python: \texttt{\bl norm.cdf(z)} \pause
    \item Two-tailed test: $p-\text{value} = 2(1- \Phi(|z|))$;\\ \pause 
    Python: \texttt{\bl 2 * norm.cdf(np.abs(z))} \pause
    \end{itemize}
  \end{itemize}
\end{frame}

\end{document}
%%% Local Variables:
%%% mode: latex
%%% TeX-master: t
%%% End:
