% \def\bmode{2} % Mode 0 for presentation, mode 1 for a handout with notes, mode 2 for handout without notes
% \if 0\bmode
% \documentclass[smaller]{beamer}
% \else \if 1\bmode
% \immediate\write18{pdflatex -jobname=\jobname-Notes-Handout\space\jobname}
% \documentclass[smaller,handout]{beamer}
% \usepackage{handoutWithNotes}
% \pgfpagesuselayout{2 on 1 with notes}[letterpaper, landscape, border shrink=4mm]
% \else \if 2\bmode
% \immediate\write18{pdflatex -jobname=\jobname_Handout\space\jobname}
% \documentclass[smaller,handout]{beamer}
% \fi
% \fi
% \fi
\documentclass[smaller, handout]{beamer}

% \documentclass[smaller,handout
% ]{beamer}
%\usepackage{etex}
%\newcommand{\num}{6{} }

% \usetheme[
%   outer/progressbar=foot,
%   outer/numbering=counter,
%  block=fillFF
% ]{metropolis}

%\useoutertheme{metropolis}

\usetheme{Madrid}
\useoutertheme[subsection=false]{miniframes} % Alternatively: miniframes, infolines, split
\useinnertheme{circles}
\usecolortheme{seahorse}

\usepackage[backend=biber,style=authoryear,maxcitenames=2,maxbibnames=99,safeinputenc,url=false,
eprint=false]{biblatex}
\addbibresource{bib/references.bib}
\AtEveryCitekey{\iffootnote{{\tiny}\tiny}{\tiny}}
\usepackage{appendixnumberbeamer}
%\usepackage{pgfpages}
%\setbeameroption{hide notes} % Only slides
%\setbeameroption{show only notes} % Only notes
%\setbeameroption{hide notes} % Only notes
%\setbeameroption{show notes on second screen=right} % Both

% \usepackage[sfdefault]{Fira Sans}

% \setsansfont[BoldFont={Fira Sans}]{Fira Sans Light}
% \setmonofont{Fira Mono}

%\usepackage{fira}
%\setsansfont{Fira}
%\setmonofont{Fira Mono}
% To give a presentation with the Skim reader (http://skim-app.sourceforge.net) on OSX so
% that you see the notes on your laptop and the slides on the projector, do the following:
% 
% 1. Generate just the presentation (hide notes) and save to slides.pdf
% 2. Generate onlt the notes (show only nodes) and save to notes.pdf
% 3. With Skim open both slides.pdf and notes.pdf
% 4. Click on slides.pdf to bring it to front.
% 5. In Skim, under "View -> Presentation Option -> Synhcronized Noted Document"
%    select notes.pdf.
% 6. Now as you move around in slides.pdf the notes.pdf file will follow you.
% 7. Arrange windows so that notes.pdf is in full screen mode on your laptop
%    and slides.pdf is in presentation mode on the projector.

% Give a slight yellow tint to the notes page
%\setbeamertemplate{note page}{\pagecolor{yellow!5}\insertnote}\usepackage{palatino}


%\usetheme{metropolis}
%\usecolortheme{beaver}
\usepackage{xcolor}
\definecolor{darkcandyapplered}{HTML}{A40000}
\definecolor{lightcandyapplered}{HTML}{e74c3c}

%\setbeamercolor{title}{fg=darkcandyapplered}
%\setbeamercolor{frametitle}{bg=darkcandyapplered!80!black!90!white}
%\setbeamertemplate{frametitle}{\bf\insertframetitle}
%\setbeamercolor{footnote mark}{fg=darkcandyapplered}
%\setbeamercolor{footnote}{fg=darkcandyapplered!70}
%\Raggedbottom
%\setbeamerfont{page number in head/foot}{size=\tiny}
%\usepackage[tracking]{microtype}


\setbeamertemplate{frametitle}{%
    \nointerlineskip%
    \begin{beamercolorbox}[wd=\paperwidth,ht=2.0ex,dp=0.6ex]{frametitle}
        \hspace*{1ex}\insertframetitle%
    \end{beamercolorbox}%
}



\setbeamerfont{caption}{size=\footnotesize}
\setbeamercolor{caption name}{fg=darkcandyapplered}


%\usepackage[sc,osf]{mathpazo}   % With old-style figures and real smallcaps.
%\linespread{1.025}              % Palatino leads a little more leading

% Euler for math and numbers
%\usepackage[euler-digits,small]{eulervm}
%\AtBeginDocument{\renewcommand{\hbar}{\hslash}}
\usepackage{graphicx,multirow,paralist,booktabs}


%\mode<presentation> { \setbeamercovered{transparent} }

\setbeamertemplate{navigation symbols}{}
\makeatletter
\def\beamerorig@set@color{%
  \pdfliteral{\current@color}%
  \aftergroup\reset@color
}
\def\beamerorig@reset@color{\pdfliteral{\current@color}}
\makeatother

%=== GRAPHICS PATH ===========
\graphicspath{{./images/}}
% Marginpar width
%Marginpar width
%\setlength{\marginparsep}{.02in}


%% Captions
% \usepackage{caption}
% \captionsetup{
%   labelsep=quad,
%   justification=raggedright,
%   labelfont=sc
% }

%AMS-TeX packages

\usepackage{amssymb,amsmath,amsthm} 
\usepackage{bm}
\usepackage{color}

\usepackage{hyperref,enumerate}
\usepackage{minitoc,array}


%https://tex.stackexchange.com/a/31370/2269
\usepackage{mathtools,cancel}

\renewcommand{\CancelColor}{\color{red}} %change cancel color to red

\makeatletter
\let\my@cancelto\cancelto %copy over the original cancelto command
\newcommand<>{\cancelto}[2]{\alt#3{\my@cancelto{#1}{#2}}{\mathrlap{#2}\phantom{\my@cancelto{#1}{#2}}}}
% redefine the cancelto command, using \phantom to assure that the
% result doesn't wiggle up and down with and without the arrow
\makeatother


\definecolor{slblue}{rgb}{0,.3,.62}
\hypersetup{
    colorlinks,%
    citecolor=blue,%
    filecolor=blue,%
    linkcolor=blue,
    urlcolor=slblue
}

%%% TIKZ
\usepackage{animate}
\usepackage{tikz}
\usepackage{pgfplots}
\usepackage{pgfplotstable}
\usepackage{pgfgantt}
\usepackage{tikzsymbols}
\pgfplotsset{compat=newest}
\usepgfplotslibrary{groupplots,fillbetween}

\usetikzlibrary{arrows,shapes,positioning,shapes.geometric}
\usetikzlibrary{decorations.markings}
\usetikzlibrary{shadows,automata}
\usetikzlibrary{patterns,matrix}
\usetikzlibrary{trees,mindmap,backgrounds}
%\usetikzlibrary{circuits.ee.IEC}
\usetikzlibrary{decorations.text}
% For Sagnac Picture
\usetikzlibrary{%
    decorations.pathreplacing,%
    decorations.pathmorphing%
}
\tikzset{no shadows/.style={general shadow/.style=}}
%
%\usepackage{paralist}



%%% FORMAT PYTHON CODE
%\usepackage{listings}
% Default fixed font does not support bold face
\DeclareFixedFont{\ttb}{T1}{txtt}{bx}{n}{8} % for bold
\DeclareFixedFont{\ttm}{T1}{txtt}{m}{n}{8}  % for normal

% Custom colors
\definecolor{deepblue}{rgb}{0,0,0.5}
\definecolor{deepred}{rgb}{0.6,0,0}
\definecolor{deepgreen}{rgb}{0,0.5,0}
 

%\usepackage{listings}

% Python style for highlighting
% \newcommand\pythonstyle{\lstset{
% language=Python,
% basicstyle=\footnotesize\ttm,
% otherkeywords={self},             % Add keywords here
% keywordstyle=\footnotesize\ttb\color{deepblue},
% emph={MyClass,__init__},          % Custom highlighting
% emphstyle=\footnotesize\ttb\color{deepred},    % Custom highlighting style
% stringstyle=\color{deepgreen},
% frame=tb,                         % Any extra options here
    % showstringspaces=false            % 
% }}

% % Python environment
% \lstnewenvironment{python}[1][]
% {
% \pythonstyle
% \lstset{#1}
% }
% {}

% % Python for external files
% \newcommand\pythonexternal[2][]{{
% \pythonstyle
% \lstinputlisting[#1]{#2}}}

% Python for inline
% 
% \newcommand\pythoninline[1]{{\pythonstyle\lstinline!#1!}}


\newcommand{\osn}{\oldstylenums}
\newcommand{\dg}{^{\circ}}
\newcommand{\lt}{\left}
\newcommand{\rt}{\right}
\newcommand{\pt}{\phantom}
\newcommand{\tf}{\therefore}
\newcommand{\?}{\stackrel{?}{=}}
\newcommand{\fr}{\frac}
\newcommand{\dfr}{\dfrac}
\newcommand{\ul}{\underline}
\newcommand{\tn}{\tabularnewline}
\newcommand{\nl}{\newline}
\newcommand\relph[1]{\mathrel{\phantom{#1}}}
\newcommand{\cm}{\checkmark}
\newcommand{\ol}{\overline}
\newcommand{\rd}{\color{red}}
\newcommand{\bl}{\color{blue}}
\newcommand{\pl}{\color{purple}}
\newcommand{\og}{\color{orange!90!black}}
\newcommand{\gr}{\color{green!40!black}}
\newcommand{\nin}{\noindent}
\newcommand{\la}{\lambda}
\renewcommand{\th}{\theta}
\newcommand{\al}{\alpha}
\newcommand{\G}{\Gamma}
\newcommand*\circled[1]{\tikz[baseline=(char.base)]{
            \node[shape=circle,draw,thick,inner sep=1pt] (char) {\small #1};}}

\newcommand{\bc}{\begin{compactenum}[\quad--]}
\newcommand{\ec}{\end{compactenum}}

\newcommand{\p}{\partial}
\newcommand{\pd}[2]{\frac{\partial{#1}}{\partial{#2}}}
\newcommand{\dpd}[2]{\dfrac{\partial{#1}}{\partial{#2}}}
\newcommand{\pdd}[2]{\frac{\partial^2{#1}}{\partial{#2}^2}}
\newcommand{\nmfr}[3]{\Phi\left(\frac{{#1} - {#2}}{#3}\right)}


\pgfmathdeclarefunction{poiss}{1}{%
  \pgfmathparse{(#1^x)*exp(-#1)/(x!)}%
  }

\pgfmathdeclarefunction{gauss}{2}{%
  \pgfmathparse{1/(#2*sqrt(2*pi))*exp(-((x-#1)^2)/(2*#2^2))}%
}

\pgfmathdeclarefunction{expo}{2}{%
  \pgfmathparse{#1*exp(-#1*#2)}%
}

\pgfmathdeclarefunction{expocdf}{2}{%
  \pgfmathparse{1 -exp(-#1*#2)}%
}

% \makeatletter
% \long\def\ifnodedefined#1#2#3{%
%     \@ifundefined{pgf@sh@ns@#1}{#3}{#2}%
% }

% \pgfplotsset{
%     discontinuous/.style={
%     scatter,
%     scatter/@pre marker code/.code={
%         \ifnodedefined{marker}{
%             \pgfpointdiff{\pgfpointanchor{marker}{center}}%
%              {\pgfpoint{0}{0}}%
%              \ifdim\pgf@y>0pt
%                 \tikzset{options/.style={mark=*, fill=white}}
%                 \draw [densely dashed] (marker-|0,0) -- (0,0);
%                 \draw plot [mark=*] coordinates {(marker-|0,0)};
%              \else
%                 \tikzset{options/.style={mark=none}}
%              \fi
%         }{
%             \tikzset{options/.style={mark=none}}        
%         }
%         \coordinate (marker) at (0,0);
%         \begin{scope}[options]
%     },
%     scatter/@post marker code/.code={\end{scope}}
%     }
% }

% \makeatother

\renewcommand{\arraystretch}{1.5}
%%%%%%%%%%%%%%%%%%%%%%%%%%%%%%%%%%%%%%%%%%%%%%%%%%%
%%%%%%%%%%%%%%%%%%%%%%%%%%%%%%%%%%%%%%%%%%%%%%%%%%%

\title[CEE 260/MIE 273 M5b: Inference for Diff.dro of Two Props.]{{\normalsize CEE 260/MIE 273: Probability and Statistics in Civil Engineering} \\
Lecture M5b: Inference for Difference of Two Proportions}
\date[\today]{\footnotesize \today}
\author{{\bf Jimi Oke}}
\institute[UMass Amherst]{
  \begin{tikzpicture}[baseline=(current bounding box.center)]
    \node[anchor=base] at (-7,0) (its) {\includegraphics[scale=.3]{UMassEngineering_vert}} ;
  \end{tikzpicture}
}

\newcommand{\hpp}{\hat{p_1} - \hat{p_2}}
\newcommand{\pp}{p_1 - p_2}    
\begin{document}

\maketitle




\begin{frame}
  \frametitle{Outline}
  \tableofcontents
\end{frame}

 


\begin{frame}
  \frametitle{Today's objectives}

  \begin{itemize}
  \item Normality test for difference of two proportions $(p_1 - p_2)$ \pause
  \item Compute CIs for $\pp$  \pause
  \item Conduct hypothesis tests for $\pp$
  \item Using the pooled proportion $\hat p_{pooled}$
  \end{itemize}
\end{frame}

\begin{frame}
	\frametitle{Difference of two proportions}\pause
	\begin{itemize}
		\item Earlier, we considered how to perform inference for a \textit{\rd single} population proportion $p$, using sample estimates $\hat p$ (sample proportion) and $n$ (sample size)
		\item However, there are cases whereby we want to compare proportions from \textit{\bl two groups/populations}
		\item In such cases, we perform inference on the \textbf{difference} of two population proportions $p_1$ and $p_2$
		\item Thus, the parameter of interest is $p_1 - p_2$, and we define the following sample statistics: \pause
		\begin{itemize}
			\item $\hat p_1$: sample proportion for group 1
			\item $\hat p_2$: sample proportion for group 2
			\item $\hpp$: difference two sample proportions
			\item $n_1$: sample size of group 1
			\item $n_2$: sample size of group 2
		\end{itemize}
	\end{itemize}
\end{frame}

\section{Test for normality}

\begin{frame}
  \frametitle{Normality conditions}\pause

  The difference of two sample proportions $\hpp$ can be assumed to follow a normal distribution if: \pause

  \begin{itemize}
    \item The data are obtained from 2 independent random samples (or from a randomized experiment) \textbf{\rd [Independence (extended)]}
    \item The {\bf\rd success-failure condition} holds for both groups separately, i.e. \pause
    \begin{eqnarray}
      n_1\hat p_1 &\ge& 10 \\
      n_1(1-\hat p_1) &\ge& 10 \\\pause
      n_2 \hat p_2 &\ge& 10 \\
      n_2 (1-\hat p_2) &\ge& 10
    \end{eqnarray}
  \end{itemize}
  \pause
  If these conditions hold, then we can use the normal distribution to find appropriate critical values in order to compute CIs and perform hypothesis tests.

\end{frame}

\begin{frame}
  \frametitle{Standard error of $\hpp$}
  \pause
  In order to compute CIs and perform hypothesis tests for a difference of two proportions $p_1 - p_2$, we need to first find the standard error.
  
  If the normality conditions are satisfied, then the {\bf\gr standard error} of $\hpp$ is given by:
  \pause
  \begin{equation}\gr
    SE_{\hpp} = \sqrt{\fr{p_1(1-p_1)}{n_1} + \fr{p_2(1-p_2)}{n_2}}
  \end{equation}
  \pause
  where: 
  \begin{itemize}
    \item $p_1$: population proportion for group 1
    \item $p_2$: population proportion for group 2 \pause
    \item $n_1$: sample size of group 1 \pause
    \item $n_2$: sample size of group 2
  \end{itemize}
\pause
\begin{alertblock}{}
	In cases where we do not know the population proportions (which is typically the case), we can \textbf{approximate} the $SE$ using the sample proportions $\hat p_1$ and $\hat p_2$. \pause Thus:
  \begin{equation}\gr
	SE_{\hpp} \approx \sqrt{\fr{\hat p_1(1-\hat p_1)}{n_1} + \fr{\hat p_2(1- \hat p_2)}{n_2}}
\end{equation}
\vspace{-3ex}
\end{alertblock}
\end{frame}

\section{CI for $\pp$}
\begin{frame}
  \frametitle{CI for difference of two proportions}
  \pause
  The confidence interval (CI) for a difference of two proportions is given by: \pause
  \begin{equation}
   {\pl \langle \pp \rangle_{(1-\alpha)} } 
   =  \underbrace{\bl \hpp}_{\bl \text{point estimate}}  \pm  \underbrace{\rd z^{*} \times SE}_{\rd \text{margin of error}}
  \end{equation}

  \pause
  where:
  \begin{eqnarray}
    SE &\approx&  \sqrt{\fr{\hat p_1(1-\hat p_1)}{n_1} + \fr{\hat p_2(1- \hat p_2)}{n_2}} \\\pause
    z^{*} &=& \Phi^{-1} \lt( 1  - \fr{\alpha}{2} \rt) \quad \text{(critical $Z$-score)}
  \end{eqnarray}
  \pause
  Also:
  \begin{itemize}
  	\item $\Phi^{-1}$ is the inverse CDF function of the standard normal distribution
  	\item $\gr \alpha$ is defined as the \textbf{\gr level of significance}
  	\item $\og 1-\alpha$ is the \textbf{\og level of confidence}
  \end{itemize}
  
  \begin{exampleblock}{Relationship between confidence and significance levels}
  	For example, if 95\% is the desired confidence level, then $1-\alpha = .95$ and $\alpha = 0.05$
  \end{exampleblock}
\end{frame}

\begin{frame}
  \frametitle{Example 1: CPR Study}
  \pause
  Construct and interpret a 95\% confidence interval for the difference between two groups of patients in a cardiopulmonary resuscitation (CPR) study. \pause
  The treatment group received a blood thinner, while the control group did not. \pause
  Outcome variable of interest: proportion of patients who survived for at least 24 hours. \pause

  \begin{center}
    \begin{tabular}{l l l l}
       & \bf Survived & \bf Died & \bf Total \\\midrule
      \bf Treatment $(t)$& 14 & 26 & 40 \\
       \bf Control $(c)$ & 11 & 39 & 50 \\\midrule
       \bf Total & 25 & 65 & 90 \\
      
    \end{tabular}
  \end{center}
  \pause
  Define: \pause
  \begin{itemize}
    \item $p_t$: survival rate in treatment group \pause
    \item $p_c$: survival rate in control group
  \end{itemize}
\end{frame}

\begin{frame}
  \frametitle{Example 1: CPR Study (cont.)}\pause
  First, we compute the difference of the two sample proportions: \pause
  \begin{equation*}
    \hat p_t - \hat p_c = \fr{14}{40} - \fr{11}{50} = \pause 0.35 - 0.22 = 0.13
  \end{equation*}
  \pause
  Next, we compute the $SE$: \pause
  \begin{eqnarray*}
    SE &\approx& \sqrt{\fr{\hat p_t(1- \hat p_t)}{n_t} + \fr{\hat p_c(1-\hat p_c)}{n_c}} \\\pause
      &=&  \sqrt{\fr{0.35(1-0.35)}{40} + \fr{0.22(1-0.22)}{50}} \pause = 0.095
  \end{eqnarray*}
  \pause

\end{frame}



\begin{frame}
  \frametitle{Example 1: CPR Study (cont.)}\pause
  Then we obtain the appropriate critical Z-score $z^*$.\\ \pause 
  For a 95\% CI, $\alpha = 0.05$. Thus, \pause
  \begin{eqnarray*}
    z^* &=& \Phi^{-1}( 1 - \alpha/2) = \Phi^{-1}(1 - 0.05/2) \pause = \Phi^{-1}(0.975) \\\pause
     &=&  \mathtt{norminv(.975)} = 1.96
  \end{eqnarray*}
  \pause
	Below is the standard normal distribution showing the critical $Z$-scores corresponding to the desired confidence level of 95\%:
\begin{center}
	\begin{tikzpicture}[
		declare function={gamma(\z)=
			2.506628274631*sqrt(1/\z)+ 0.20888568*(1/\z)^(1.5)+ 0.00870357*(1/\z)^(2.5)
			- (174.2106599*(1/\z)^(3.5))/25920- (715.6423511*(1/\z)^(4.5))/1244160)*exp((-ln(1/\z)-1)*\z;},
		declare function={student(\x,\n)= gamma((\n+1)/2.)/(sqrt(\n*pi) *gamma(\n/2.)) *((1+(\x*\x)/\n)^(-(\n+1)/2.));}, scale=.9
		]
		\begin{axis}[no markers, domain=-5:5, samples=100,
			axis x line=center,
			axis y line=none,
			xlabel=$Z$, ylabel=$f_X(x)$,
			height=3cm, width=14cm,
			every x tick/.style={color=black, thick},
			xtick={-1.96, 0,1.96},
			xticklabels={$-1.96$,0,$+1.96$},
			ymax=.15,
			ytick=\empty,
			x label style={anchor=north},
			y label style={anchor=south},
			enlargelimits=true, clip=false, axis on top
			]
			\addplot [blue, domain=-5:5] {gauss(0,1)};
			\addplot [gray, fill=gray!50, domain=-1.96:1.96] {gauss(0,1)} \closedcycle;
			\addplot [orange,fill=orange,  domain=-5:-1.96] {gauss(0,1)} \closedcycle;
			\addplot [orange,fill=orange,  domain=1.96:5] {gauss(0,1)} \closedcycle;
			\node (d) at (axis cs: 0,.1) {Area $= 1- \alpha =  .95$};
			\node (c) at (axis cs: 3.2,.08) {\og Area = $\fr\alpha2 = .025$};
			\draw[thick,->] (c) -- (axis cs: 2, 0.03);
			\node (c) at (axis cs: -3.2,.08) {\og Area = $\fr\alpha2 = .025$};
			\draw[thick,->] (c) -- (axis cs: -2, 0.03);
		\end{axis}
	\end{tikzpicture}
	% \caption{Hypothesis test for Problem 6.1}

\end{center}  


\end{frame}

\begin{frame}
	\frametitle{Example 1: CPR Study (cont.)}
	\pause
  Thus, the CI is given by: 

\begin{eqnarray*}
	\langle p_t -  p_c \rangle_{.95} &=& 0.13 \pm (1.96\times 0.095) =  0.13 \pm 0.186 \\\pause &=& (-0.056, 0.316)
\end{eqnarray*}
\pause
Now, we show the corresponding sampling distribution of $p_t - p_c$  with the computed CIs (i.e.\ the standard normal distribution converted into the actual scale of the point estimate $\hpp$)
	\begin{center}
		\begin{tikzpicture}[
			declare function={gamma(\z)=
				2.506628274631*sqrt(1/\z)+ 0.20888568*(1/\z)^(1.5)+ 0.00870357*(1/\z)^(2.5)
				- (174.2106599*(1/\z)^(3.5))/25920- (715.6423511*(1/\z)^(4.5))/1244160)*exp((-ln(1/\z)-1)*\z;},
			declare function={student(\x,\n)= gamma((\n+1)/2.)/(sqrt(\n*pi) *gamma(\n/2.)) *((1+(\x*\x)/\n)^(-(\n+1)/2.));}, scale=.9
			]
			
			\begin{axis}[no markers, domain=-5:5, samples=100,
				axis x line=center,
				axis y line=none,
				xlabel=$(p_t - p_c)$, ylabel=$f_X(x)$,
				height=3cm, width=14cm,
				every x tick/.style={color=black, thick},
				xtick={-1.96, 0,1.96},
				xticklabels={$-0.056$, 0.13, 0.316},
				ymax=.15,
				ytick=\empty,
				x label style={anchor=north},
				y label style={anchor=south},
				enlargelimits=true, clip=false, axis on top
				]
				\addplot [blue, domain=-5:5] {gauss(0,1)};
				\addplot [gray, fill=gray!50, domain=-1.96:1.96] {gauss(0,1)} \closedcycle;
				\addplot [orange,fill=orange,  domain=-5:-1.96] {gauss(0,1)} \closedcycle;
				\addplot [orange,fill=orange,  domain=1.96:5] {gauss(0,1)} \closedcycle;
				\node (d) at (axis cs: 0,.1) {Area $= 1- \alpha =  .95$};
				\node (c) at (axis cs: 3.2,.08) {\og Area = $\fr\alpha2 = .025$};
				\draw[thick,->] (c) -- (axis cs: 2, 0.03);
				\node (c) at (axis cs: -3.2,.08) {\og Area = $\fr\alpha2 = .025$};
				\draw[thick,->] (c) -- (axis cs: -2, 0.03);
			\end{axis}
		\end{tikzpicture}

	\end{center}

\end{frame}

\begin{frame}
		\frametitle{Example 1: CPR Study (cont.)}
		\begin{exampleblock}{Interpretation}
		Thus, we are 95\% confident (\textit{OR} 95\% of the time), the difference in survival rate between those who are treated by the blood thinner and those who are not lies between $-5.6$ and $31.6$ percentage points.
	\end{exampleblock}	
\end{frame}
\section{Hypothesis testing}

\begin{frame}
	\frametitle{Hypothesis testing using critical value}
	\begin{compactenum}
		\item State the hypotheses $H_0$ and $H_1$
		\item Compute the point estimate  $\hpp$
		\item Find the standard error  $SE$
		\item Find the test statistic $z = \fr{(\hpp) - \Delta_0}{SE}$, where $\Delta_0$ is the null value of the difference
		\item Find the critical value(s) $z^*$
		\item Compare the test statistic to the critical value
		\item Clearly state the outcome from your hypothesis test  
		\item Write a final concluding statement in response to the question  
	\end{compactenum}
	
\end{frame}

\begin{frame}
  \frametitle{Hypothesis testing using p-value}
\begin{compactenum}
  \item State the hypotheses  $H_0$ and $H_1$
  \item Compute the point estimate  $\hpp$
  \item Find the standard error  $SE$
  \item Find the p-value  
  \item Compare the p-value to the level of significance $\alpha$  
  \item Clearly state the outcome from your hypothesis test  
  \item Write a final concluding statement in response to the question  
  \end{compactenum}

\end{frame}



\begin{frame}
	\frametitle{Pooled proportion}\pause
	If the hypothesis test is to check whether $p_1 = p_2$ or that $p_1 - p_2 = 0$ (null hypothesis), then:
	\begin{eqnarray}
		H_0: p_1 - p_2 &=& \Delta_0 =  0 \\
		H_1: p_1 - p_2 &\ne& 0
	\end{eqnarray}
	
	In this case, we use the \textbf{pooled proportion} $\hat p_{pooled}$ to verify the success-failure condition and to estimate the standard error
	\pause
	\begin{eqnarray}
		\hat p_{pooled} &=& \fr{\text{number of successes}}{\text{number of successes} + \text{number of failures}} \\[2mm]
		&=& \fr{\text{number of successes}}{\text{total number of cases}}\\[2mm]
		&=&
		\fr{\hat p_1 n_1 + \hat p_2 n_2}{n_1 + n_2} 
	\end{eqnarray}
	\pause
	Thus:
	  \begin{eqnarray*}
		SE &\approx& \sqrt{\fr{\hat p_{pooled}(1- \hat p_{pooled})}{n_1} + \fr{\hat p_{pooled}(1-\hat p_{pooled})}{n_2}} 
	\end{eqnarray*}
	

\end{frame}

\begin{frame}
  \frametitle{Example 3: Mammogram study}
  \pause
  A study is conducted over a 30-year period with about 90,000 female participants to determine whether mammograms (X-ray procedure to test for breast cancer) are more effective than non-mammogram exams. \pause

  \begin{center}
    \begin{tabular}{lll}
       & \multicolumn{2}{c}{Death from breast cancer?} \\\midrule
       & Yes & No \\\midrule
       Mammogram ($m$) & 500 & 44,425 \\
       Control ($c$)& 505 & 44,405
    \end{tabular}
  \end{center}
  \pause
  Conduct a hypothesis test determine whether there is no significant evidence to suggest that mammograms are more effective. Use a significance level $\alpha = 0.05$
  
  \begin{exampleblock}{Hypotheses}
  	Thus, the null hypothesis $H_0$ is: $p_m - p_c = 0$ (i.e.\ the null difference $\Delta_0 = 0$). \\
  	And the alternative hypothesis $H_1$ is: $p_m - p_c \ne 0$.
  \end{exampleblock}
\end{frame}

\begin{frame}
	\frametitle{Example 3 (cont.)}
	
		In this case the pooled proportion is given by
	\pause
	\begin{eqnarray*}
		\hat p_{pooled} &=& \fr{\text{\# patients who died from breast cancer in the entire study}}{
			\text{\# patients in the entire study}} \\[2mm]
			&=& \fr{500 + 505}{500 + 44,425 + 505 + 44,405}\\[2mm]
			&=& 0.0112
	\end{eqnarray*}
	Thus, the $SE$ is given by:
		  \begin{eqnarray*}
		SE &\approx& \sqrt{\fr{\hat p_{pooled}(1- \hat p_{pooled})}{n_m} + \fr{\hat p_{pooled}(1-\hat p_{pooled})}{n_c}} \\
		&=& \sqrt{\fr{0.0112( 1- 0.0112)}{44,925} + \fr{0.0112(1-0.0112)}{44,910}} = 0.0007
	\end{eqnarray*}
\end{frame}

\begin{frame}
	\frametitle{Example 3 (cont.): test statistic}
	Now we find the point estimate $\hat p_m - \hat p_c$: 
	
	\begin{equation}
		\hat p_m - \hat p_c = \fr{500}{44,925} - \fr{505}{44,910} = -0.00012
	\end{equation}
	
	Then, the test statistic is given by:
	\begin{eqnarray*}
		z &=& \fr{(\hat p_m - \hat p_c) - \Delta_0}{SE} \\
		&=& \fr{-0.00012 - 0}{.0007} = -0.17
	\end{eqnarray*}
\end{frame}

\begin{frame}
	\frametitle{Example 3 (cont.): critical value}
	The critical value $z^*$ in this case is given by:
	
	\begin{eqnarray*}
		z^* &=& \Phi^{-1}(1 - \alpha/2) = \Phi^{-1}(1 - .025) = 1.96
	\end{eqnarray*}
	
		\begin{center}
		\begin{tikzpicture}[
			declare function={gamma(\z)=
				2.506628274631*sqrt(1/\z)+ 0.20888568*(1/\z)^(1.5)+ 0.00870357*(1/\z)^(2.5)
				- (174.2106599*(1/\z)^(3.5))/25920- (715.6423511*(1/\z)^(4.5))/1244160)*exp((-ln(1/\z)-1)*\z;},
			declare function={student(\x,\n)= gamma((\n+1)/2.)/(sqrt(\n*pi) *gamma(\n/2.)) *((1+(\x*\x)/\n)^(-(\n+1)/2.));}, scale=.9
			]
			
			\begin{axis}[no markers, domain=-5:5, samples=100,
				axis x line=center,
				axis y line=none,
				xlabel=$Z$, ylabel=$f_X(x)$,
				height=3cm, width=14cm,
				every x tick/.style={color=black, thick},
				xtick={-1.96, -.17, 1.96},
				xticklabels={$-1.96$, $-0.17$,  1.96},
				ymax=.15,
				ytick=\empty,
				x label style={anchor=north},
				y label style={anchor=south},
				enlargelimits=true, clip=false, axis on top
				]
				\addplot [blue, domain=-5:5] {gauss(0,1)};
				\addplot [gray, fill=gray!50, domain=-1.96:1.96] {gauss(0,1)} \closedcycle;
				\addplot [orange,fill=orange,  domain=-5:-1.96] {gauss(0,1)} \closedcycle;
				\addplot [orange,fill=orange,  domain=1.96:5] {gauss(0,1)} \closedcycle;
				\node (d) at (axis cs: 0,.1) {Area $= 1- \alpha =  .95$};
				\node (c) at (axis cs: 3.2,.08) {\og Area = $\fr\alpha2 = .025$};
				\draw[thick,->] (c) -- (axis cs: 2, 0.03);
				\node (c) at (axis cs: -3.2,.08) {\og Area = $\fr\alpha2 = .025$};
				\draw[thick,->] (c) -- (axis cs: -2, 0.03);
			\end{axis}
		\end{tikzpicture}
		
	\end{center}
	
	Since $-0.17 > -1.96$ and $-0.17 < 1.96$ (i.e. the test statistic is contained within the interval ($-1.96$, $1.96$)), then we \textbf{fail to reject} the null hypothesis $H_0$. \textbf{Thus, we conclude that there is no evidence to suggest that mammogram exams are significantly more effective than non-mammogram exams}.
\end{frame}

\begin{frame}
	\frametitle{Example 3 (cont.)}
	Next, we will conduct the hypothesis test using p-values (to be continued)
\end{frame}
\end{document}
%%% Local Variables:
%%% mode: latex
%%% TeX-master: t
%%% End:
