\documentclass[10pt,twoside]{article}
\usepackage{etex}

\raggedbottom

%geometry (sets margin) and other useful packages
\usepackage{geometry}
\geometry{top=1in, left=1in,right=1in,bottom=1in}
 \usepackage{graphicx,booktabs,calc}
 
\usepackage{listings}


% Marginpar width
%Marginpar width
\newcommand{\pts}[1]{\marginpar{ \small\hspace{0pt} \textit{[#1]} } } 
\setlength{\marginparwidth}{.5in}
%\reversemarginpar
%\setlength{\marginparsep}{.02in}

 
%\usepackage{cmbright}lstinputlisting
%\usepackage[T1]{pbsi}


\usepackage{chngcntr,mathtools}
%\counterwithin{figure}{section}
%\numberwithin{equation}{section}

%\usepackage{listings}
\graphicspath{{./images/}}

%AMS-TeX packages
\usepackage{amssymb,amsmath,amsthm} 
\usepackage{bm}
\usepackage[mathscr]{eucal}
\usepackage{colortbl}
\usepackage{color}


\usepackage{subfigure,hyperref,enumerate,polynom,polynomial}
\usepackage{multirow,minitoc,fancybox,array,multicol}

\definecolor{slblue}{rgb}{0,.3,.62}
\hypersetup{
    colorlinks,%
    citecolor=blue,%
    filecolor=blue,%
    linkcolor=blue,
    urlcolor=slblue
}

%%%TIKZ
\usepackage{tikz}

\usepackage{pgfplots}
\pgfplotsset{compat=newest}

\usetikzlibrary{arrows,shapes,positioning}
\usetikzlibrary{decorations.markings}
\usetikzlibrary{shadows}
\usetikzlibrary{patterns}
%\usetikzlibrary{circuits.ee.IEC}
\usetikzlibrary{decorations.text}
% For Sagnac Picture
\usetikzlibrary{%
    decorations.pathreplacing,%
    decorations.pathmorphing%
}

\tikzstyle arrowstyle=[black,scale=2]
\tikzstyle directed=[postaction={decorate,decoration={markings,
    mark=at position .65 with {\arrow[arrowstyle]{stealth}}}}]
\tikzstyle reverse directed=[postaction={decorate,decoration={markings,
    mark=at position .65 with {\arrowreversed[arrowstyle]{stealth};}}}]
\tikzstyle dir=[postaction={decorate,decoration={markings,
    mark=at position .98 with {\arrow[arrowstyle]{latex}}}}]
\tikzstyle rev dir=[postaction={decorate,decoration={markings,
    mark=at position .98 with {\arrowreversed[arrowstyle]{latex};}}}]

\usepackage{ctable}

%
%Redefining sections as problems
%
\makeatletter
\newenvironment{exercise}{\@startsection 
	{section}
	{1}
	{-.2em}
	{-3.5ex plus -1ex minus -.2ex}
    	{1.3ex plus .2ex}
    	{\pagebreak[3]%forces pagebreak when space is small; use \eject for better results
	\large\bf\noindent{Exercise 1.\hspace{-1.5ex} }
	}
	}
	%{\vspace{1ex}\begin{center} \rule{0.3\linewidth}{.3pt}\end{center}}
	%\begin{center}\large\bf \ldots\ldots\ldots\end{center}}
\makeatother

%
%Fancy-header package to modify header/page numbering 
%
\usepackage{fancyhdr}
\pagestyle{fancy}
%\addtolength{\headwidth}{\marginparsep} %these change header-rule width
%\addtolength{\headwidth}{\marginparwidth}
%\fancyheadoffset{30pt}
%\fancyfootoffset{30pt}
\fancyhead[LO,RE]{\small Oke}
\fancyhead[RO,LE]{\small Page \thepage} 
\fancyfoot[RO,LE]{\small E1 Formula Sheet} 
\fancyfoot[LO,RE]{\small \scshape CEE 260/MIE 273} 
\cfoot{} 
\renewcommand{\headrulewidth}{0.1pt} 
\renewcommand{\footrulewidth}{0.1pt}
%\setlength\voffset{-0.25in}
%\setlength\textheight{648pt}


\usepackage{paralist}

\newcommand{\osn}{\oldstylenums}
\newcommand{\lt}{\left}
\newcommand{\rt}{\right}
\newcommand{\pt}{\phantom}
\newcommand{\tf}{\therefore}
\newcommand{\?}{\stackrel{?}{=}}
\newcommand{\fr}{\frac}
\newcommand{\dfr}{\dfrac}
\newcommand{\ul}{\underline}
\newcommand{\tn}{\tabularnewline}
\newcommand{\nl}{\newline}
\newcommand\relph[1]{\mathrel{\phantom{#1}}}
\newcommand{\cm}{\checkmark}
\newcommand{\ol}{\overline}
\newcommand{\rd}{\color{red}}
\newcommand{\bl}{\color{blue}}
\newcommand{\pl}{\color{purple}}
\newcommand{\og}{\color{orange!90!black}}
\newcommand{\gr}{\color{green!40!black}}
\newcommand{\nin}{\noindent}
\newcommand{\la}{\lambda}
\renewcommand{\th}{\theta}
\newcommand*\circled[1]{\tikz[baseline=(char.base)]{
            \node[shape=circle,draw,thick,inner sep=1pt] (char) {\small #1};}}

\newcommand{\bc}{\begin{compactenum}[\quad--]}
\newcommand{\ec}{\end{compactenum}}

\newcommand{\n}{\\[2mm]}
%% GREEK LETTERS
\newcommand{\al}{\alpha}
\newcommand{\gam}{\gamma}
\newcommand{\eps}{\epsilon}
\newcommand{\sig}{\sigma}

\newcommand{\p}{\partial}
\newcommand{\pd}[2]{\frac{\partial{#1}}{\partial{#2}}}
\newcommand{\dpd}[2]{\dfrac{\partial{#1}}{\partial{#2}}}
\newcommand{\pdd}[2]{\frac{\partial^2{#1}}{\partial{#2}^2}}
\newcommand{\mr}{\mathbb{R}}
\newcommand{\xs}{x^{*}}
\newenvironment{solution}
{\medskip\par\quad\quad\begin{minipage}[c]{.8\textwidth}}{\medskip\end{minipage}}

\newcommand{\nmfr}[3]{\Phi\left(\frac{{#1} - {#2}}{#3}\right)}

%%%%%%%%%%%%%%%%%%%%%%%%%%%%%%%%%%%%%%%%%%%%%%%%%%%
%%%%%%%%%%%%%%%%%%%%%%%%%%%%%%%%%%%%%%%%%%%%%%%%%%%

\begin{document}

\lstset{language=C++,
                basicstyle=\tiny\ttfamily,
                keywordstyle=\color{blue}\ttfamily,
                stringstyle=\color{red}\ttfamily,
                commentstyle=\color{gray}\ttfamily,
                morecomment=[l][\color{gray}]{\#}
}


\thispagestyle{empty}


\nin{\LARGE E1 Formula Sheet}\hfill{\bf Prof. Oke}

\medskip\hrule\medskip

\nin {\small CEE 260/MIE 273: Probability \& Statistics in Civil Engineering
\hfill\textit{ 10.14.2025}}

\section{Set theory}
Properties:
    \begin{align}
      A \cup B &= B \cup A \\
      A \cap B &= B \cap A\\
      (A \cup B) \cup C &= A \cup ( B \cup C) \\
      (AB)C &= A(BC)\\
    (A \cup B) \cap C &= (A \cap C) \cup (B \cap C)  \\
      (AB) \cup C &= (A\cup C) \cap (B \cup C) 
    \end{align}
De Morgan's rule:
\begin{align}
  \overline{E_1 \cup E_2 \cup \cdots \cup E_n} &= \overline{E_1} \cap \overline{E_2} \cap \cdots \cap \overline{E_n}\\
  \overline{E_1 \cap E_2 \cap \cdots \cap E_n} &= \overline{E_1} \cup \overline{E_2} \cup \cdots \cup \overline{E_n}
\end{align}
\section{Mathematics of probability}
General rules:
\begin{align}
  P(\ol{E}) &= 1- P(E) \\
  P(A\cup B) &= P(A) + P(B) - P(AB) \\
  P(A\cap B) &= P(AB) = P(A|B)P(B) \\
  P(A|B) &= \fr{P(AB)}{P(B)}\\
  P(A|B) &= 1 - P(\ol{A}|B)
\end{align}
Mutually exclusive events:
\begin{equation}
    P(A\cup B) = P(A) + P(B) 
\end{equation}
Statistically independent events:
\begin{align}
  P(AB) &= P(A)P(B) \\
  P(A|B) &= P(A) \\
  P(B|A) &= P(B)
\end{align}
Total probability theorem:
\begin{equation}
  P(A) = P(A|E_1)P(E_1) + P(A|E_2)P(E_2) + \cdots + P(A|E_n)P(E_n) = \sum_{i=1}^n P(A|E_i)P(E_i)
\end{equation}
Bayes' theorem:
\begin{equation}
  \label{eq:2}
  P(E_i|A) = \fr{P(A|E_i)P(E_i)}{\sum_{j=1}^nP(A|E_j)P(E_j)}
\end{equation}

\section{Counting}
Number of ways to arrange $n$ objects:
\begin{equation}
  n! = n(n-1)(n-2)\cdots 3\times 2\times 1
\end{equation}
Permutations (arrangements) of $n$ objects taken $k$ at a time:
\begin{equation}
  _nP_k = \fr{n!}{(n-k)!}
\end{equation}
Combinations of $n$ objects taken $k$ at a time:
\begin{equation}
  _nC_k = {n \choose k} = \fr{n!}{k!(n-k)!}
\end{equation}    


\section{Probability distributions}
~
\begin{table}[ht!]\small
  \begin{tabular}{l l l l l} \toprule
    \bf Distribution & \bf Mean &\bf Variance & \bf Median & $P(X \leq x)$\\ \toprule
    Uniform($a,b$) & $\fr{a+b}2$ & $\fr{(b-a)^2}{12}$ & $\fr{a+b}2$ & $\texttt{uniform.cdf(x, a, b)}$ \\[2mm]
    $\mathcal{N}(\mu,\sigma)$ & $\mu$ & $\sigma^2$ & $\mu$ & \texttt{norm.cdf(x, mu, sigma)} \\[2mm]
    $\text{Lognormal}(\mu,\sigma)$ & $\lt[\exp\lt(\mu + \fr12 \sigma^2\rt)\rt]\exp(2\mu + \sigma^2)$ & $\exp\lt(\sigma^2\rt) - 1$ & $\exp\lt(\mu\rt)$ & \texttt{lognorm.cdf(x, sigma, np.exp(mu))} \\[2mm]
    $\text{Binomial}(n,p)$ & $np$ & $np(1-p)$ & $\lfloor (n+1)p \rfloor$ & \texttt{binom.cdf(x, n, p)} \\[2mm]
    $\text{Poisson}(\lambda)$ & $\lambda$ & $\lambda$ & $\lfloor \lambda \rfloor$ & \texttt{poisson.cdf(x, lambda)} \\[2mm]
    $\text{Exponential}(\lambda)$ & $\fr1{\lambda}$ & $\fr1{\lambda^2}$ & $\fr{\ln 2}{\lambda}$ & \texttt{expon.cdf(x, 1/lambda)} \\\bottomrule
  \end{tabular}
  \caption{Common probability distributions.}
  \label{tab:distributions}
\end{table}

Note: the $\lfloor \cdot \rfloor$ symbol means to round down to the nearest integer.

%\eject
% \section{Joint distributions}
% For two \textbf{discrete} r.v.'s $X$ and $Y$:
% \begin{align}
%   p_{X,Y}(x_i,y_j)   &=  p_{X|Y}(x_i|y_j) p_Y(y_j)  =   p_{Y|X}(y_j|y_i)p_X(x_i) \\
%   p_X(x_i) &= \sum_{y_j}p_{X,Y}(x_i,y_j)  \\
%   p_Y(y_j) &= \sum_{x_i}p_{X,Y}(x_i,y_j)  
% \end{align}
% If $X$ and $Y$ are statistically independent:
% \begin{equation}
%   \label{eq:50a}
%    p_{X,Y}(x_i,y_j)  = p_X(x_i)p_Y(y_j)
% \end{equation}
% For two \textbf{continuous} r.v.'s $X$ and $Y$:
% \begin{align}
%   f_{X,Y}(x,y) &= f_{X|Y}(x|y)f_Y(y) = f_{Y|X}(y|x)f_X(x)\\
%      f_X(x) &= \int_{-\infty}^\infty f_{X,Y}(x,y)dy \\ 
%   f_Y(y) &=  \int_{-\infty}^\infty f_{X,Y}(x,y)dx
% \end{align}
% If $X$ and $Y$ are statistically independent:
% \begin{equation}
%   \label{eq:50}
%   f_{X,Y}(x,y) = f_X(x) f_Y(y)
% \end{equation}
% Covariance and correlation:
% \begin{align}
%   Cov(X,Y) &= E(XY) - E(X)E(Y) = \rho_{XY}\sigma_X\sigma_Y\\
%   \rho_{XY} &= \fr{Cov(X,Y)}{\sigma_X\sigma_Y}
% \end{align}

% \eject
% \section{Functions of random variables}
% If $Z = aX \pm bY$, where $X$ and $Y$ are normal r.v.'s:
% \begin{align}
%   E(Z) &= aE(X) \pm bE(Y) \\
%   Var(Z) &= a^2Var(X) + b^2Var(Y) \pm 2ab\,Cov(X,Y)
% \end{align}
% % Where $X_i$ are any number of statistically independent random variables, and $Z = \sum_{i=1}^n a_iX_i$, then:
% % \begin{align}
% %   E(Z) &= \sum_{i=1}^n a_iE(X_i) \\
% %   Var(Z) &=\sum_{i=1}^n a_i^2Var(X_i) 
%              %   \end{align}
% For \textbf{statistically independent} random variables:\\

% If $Z = \sum_{i=1}^n X_i$, where $X_i\sim \text{Poisson}(v_i)$, then:
% \begin{equation}
%   \label{eq:21}
%   v_Z= \sum_{i=1}^n v_i
% \end{equation}

% If $Z  = \sum_{i=1}^n a_i X_i$, where $X_i \sim \mathcal{N}(\mu_{X_i},\sigma_{X_i})$, then:
% \begin{equation}
%   \label{eq:14}
% \mu_Z = \sum_{i=1}^na_i\mu_{X_i} \quad  \sigma_Z^2 = \sum_{i=1}^n a_i^2\sigma_{X_i}^2
% \end{equation}

% If $Z = \prod_{i=1}^nX_i$, where $X_i \sim \text{Lognormal}(\la_{X_i},\zeta_{X_i})$, then:
% \begin{equation}
%   \label{eq:22}
%   \la_Z = \sum_{i=1}^n\la_{X_i} \quad \zeta_Z^2 = \sum_{i=1}^n\zeta_{X_i}^2
% \end{equation}

\end{document}
