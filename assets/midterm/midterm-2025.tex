\documentclass[12pt,twoside]{article}
\usepackage{etex}

\raggedbottom

%geometry (sets margin) and other useful packages
\usepackage{geometry}
\geometry{top=1in, left=1in,right=1in,bottom=1in}
 \usepackage{graphicx,booktabs,ragged2e,calc}
 
\usepackage{listings}

% Marginpar width
%Marginpar width
\newcommand{\pts}[1]{\marginpar{ \small\hspace{0pt} \textit{[#1]} } } 
\setlength{\marginparwidth}{.5in}
%\reversemarginpar
%\setlength{\marginparsep}{.02in}

 
%\usepackage{cmbright}lstinputlisting
%\usepackage[T1]{pbsi}

\renewcommand{\baselinestretch}{1.5}

\usepackage{chngcntr,mathtools}
%\counterwithin{figure}{section}
%\numberwithin{equation}{section}

%\usepackage{listings}

%AMS-TeX packages
\usepackage{amssymb,amsmath,amsthm} 
\usepackage{bm}
\usepackage[mathscr]{eucal}
\usepackage{colortbl}
\usepackage{color}


\usepackage{subcaption,hyperref,enumerate,polynom,polynomial}
\usepackage{multirow,minitoc,fancybox,array,multicol}

\definecolor{slblue}{rgb}{0,.3,.62}
\hypersetup{
    colorlinks,%
    citecolor=blue,%
    filecolor=blue,%
    linkcolor=blue,
    urlcolor=slblue
}

%%%TIKZ
\usepackage{tikz}

\usepackage{pgfplots,pgfplotstable}
\pgfplotsset{compat=newest}

\usetikzlibrary{calc}
\usetikzlibrary{arrows,shapes,positioning}
\usetikzlibrary{decorations.markings}
\usetikzlibrary{shadows}
\usetikzlibrary{patterns}
%\usetikzlibrary{circuits.ee.IEC}
\usetikzlibrary{decorations.text}
% For Sagnac Picture
\usetikzlibrary{%
    decorations.pathreplacing,%
    decorations.pathmorphing%
}

\tikzstyle arrowstyle=[black,scale=2]
\tikzstyle directed=[postaction={decorate,decoration={markings,
    mark=at position .65 with {\arrow[arrowstyle]{stealth}}}}]
\tikzstyle reverse directed=[postaction={decorate,decoration={markings,
    mark=at position .65 with {\arrowreversed[arrowstyle]{stealth};}}}]
\tikzstyle dir=[postaction={decorate,decoration={markings,
    mark=at position .98 with {\arrow[arrowstyle]{latex}}}}]
\tikzstyle rev dir=[postaction={decorate,decoration={markings,
    mark=at position .98 with {\arrowreversed[arrowstyle]{latex};}}}]

\usepackage{ctable}

%
%Redefining sections as problems
%
\makeatletter
\newenvironment{exercise}{\@startsection 
	{section}
	{1}
	{-.2em}
	{-3.5ex plus -1ex minus -.2ex}
    	{1.3ex plus .2ex}
    	{\pagebreak[3]%forces pagebreak when space is small; use \eject for better results
	\large\bf\noindent{Exercise 1.\hspace{-1.5ex} }
	}
	}
	%{\vspace{1ex}\begin{center} \rule{0.3\linewidth}{.3pt}\end{center}}
	%\begin{center}\large\bf \ldots\ldots\ldots\end{center}}
\makeatother

%
%Fancy-header package to modify header/page numbering 
%
\usepackage{fancyhdr}
\pagestyle{fancy}
%\addtolength{\headwidth}{\marginparsep} %these change header-rule width
%\addtolength{\headwidth}{\marginparwidth}
%\fancyheadoffset{30pt}
%\fancyfootoffset{30pt}
\fancyhead[LO,RE]{\small Prof. Oke}
\fancyhead[RO,LE]{\small Page \thepage} 
\fancyfoot[RO,LE]{\small E1: Midterm} 
\fancyfoot[LO,RE]{\small \scshape CEE 260/MIE 273} 
\cfoot{} 
\renewcommand{\headrulewidth}{0.1pt} 
\renewcommand{\footrulewidth}{0.1pt}
%\setlength\voffset{-0.25in}
%\setlength\textheight{648pt}


\usepackage{paralist}

\newcommand{\osn}{\oldstylenums}
\newcommand{\lt}{\left}
\newcommand{\rt}{\right}
\newcommand{\pt}{\phantom}
\newcommand{\tf}{\therefore}
\newcommand{\?}{\stackrel{?}{=}}
\newcommand{\fr}{\frac}
\newcommand{\dfr}{\dfrac}
\newcommand{\ul}{\underline}
\newcommand{\tn}{\tabularnewline}
\newcommand{\nl}{\newline}
\newcommand\relph[1]{\mathrel{\phantom{#1}}}
\newcommand{\cm}{\checkmark}
\newcommand{\ol}{\overline}
\newcommand{\rd}{\color{red}}
\newcommand{\bl}{\color{blue}}
\newcommand{\pl}{\color{purple}}
\newcommand{\og}{\color{orange!90!black}}
\newcommand{\gr}{\color{green!40!black}}
\newcommand{\nin}{\noindent}
\newcommand{\la}{\lambda}
\renewcommand{\th}{\theta}
\newcommand*\circled[1]{\tikz[baseline=(char.base)]{
            \node[shape=circle,draw,thick,inner sep=1pt] (char) {\small #1};}}

\newcommand{\bc}{\begin{compactenum}[\quad--]}
\newcommand{\ec}{\end{compactenum}}

\newcommand{\n}{\\[2mm]}
%% GREEK LETTERS
\newcommand{\al}{\alpha}
\newcommand{\gam}{\gamma}
\newcommand{\eps}{\epsilon}
\newcommand{\sig}{\sigma}

\newcommand{\p}{\partial}
\newcommand{\pd}[2]{\frac{\partial{#1}}{\partial{#2}}}
\newcommand{\dpd}[2]{\dfrac{\partial{#1}}{\partial{#2}}}
\newcommand{\pdd}[2]{\frac{\partial^2{#1}}{\partial{#2}^2}}
\newcommand{\mr}{\mathbb{R}}
\newcommand{\xs}{x^{*}}
\newenvironment{solution}
{\medskip\par\quad\quad\begin{minipage}[c]{.8\textwidth}}{\medskip\end{minipage}}

\newcommand{\nmfr}[3]{\Phi\left(\frac{{#1} - {#2}}{#3}\right)}

% https://tex.stackexchange.com/questions/198383/drawing-cumulative-distribution-function-for-a-discrete-variable
\makeatletter
\long\def\ifnodedefined#1#2#3{%
    \@ifundefined{pgf@sh@ns@#1}{#3}{#2}%
}

\pgfplotsset{
    discontinuous/.style={
    scatter,
    scatter/@pre marker code/.code={
        \ifnodedefined{marker}{
            \pgfpointdiff{\pgfpointanchor{marker}{center}}%
             {\pgfpoint{0}{0}}%
             \ifdim\pgf@y>0pt
                \tikzset{options/.style={mark=*, fill=white}}
                \draw [densely dashed] (marker-|0,0) -- (0,0);
                \draw plot [mark=*] coordinates {(marker-|0,0)};
             \else
                \tikzset{options/.style={mark=none}}
             \fi
        }{
            \tikzset{options/.style={mark=none}}        
        }
        \coordinate (marker) at (0,0);
        \begin{scope}[options]
    },
    scatter/@post marker code/.code={\end{scope}}
    }
}

\makeatother

%%%%%%%%%%%%%%%%%%%%%%%%%%%%%%%%%%%%%%%%%%%%%%%%%%%
%%%%%%%%%%%%%%%%%%%%%%%%%%%%%%%%%%%%%%%%%%%%%%%%%%%
% Here's where you edit the Class, Exam, Date, etc.
\newcommand{\class}{CEE 260/MIE 273: Probability \& Statistics in Civil Engineering}
\newcommand{\term}{Fall 2025}
\newcommand{\examnum}{Midterm Exam}
\newcommand{\examdate}{10/08/25}
\newcommand{\timelimit}{2 Hours}


\newcolumntype{R}[1]{>{\RaggedLeft\arraybackslash}p{#1}}

\pgfmathdeclarefunction{poiss}{1}{%
  \pgfmathparse{(#1^x)*exp(-#1)/(x!)}%
  }

\pgfmathdeclarefunction{gauss}{2}{%
  \pgfmathparse{1/(#2*sqrt(2*pi))*exp(-((x-#1)^2)/(2*#2^2))}%
}

\pgfmathdeclarefunction{expo}{2}{%
  \pgfmathparse{#1*exp(-#1*#2)}%
}

\usetikzlibrary{math}

% https://tex.stackexchange.com/questions/461758/asymmetric-distribution-gauss-curve
\tikzmath{%
  function h1(\x, \lx) { return (9*\lx + 3*((\lx)^2) + ((\lx)^3)/3 + 9); };
  function h2(\x, \lx) { return (3*\lx - ((\lx)^3)/3 + 4); };
  function h3(\x, \lx) { return (9*\lx - 3*((\lx)^2) + ((\lx)^3)/3 + 7); };
  function skewnorm(\x, \l) {
    \x = (\l < 0) ? -\x : \x;
    \l = abs(\l);
    \e = exp(-(\x^2)/2);
    return (\l == 0) ? 1 / sqrt(2 * pi) * \e: (
      (\x < -3/\l) ? 0 : (
      (\x < -1/\l) ? \e / (8 * sqrt(2 * pi)) * h1(\x, \x*\l) : (
      (\x <  1/\l) ? \e / (4 * sqrt(2 * pi)) * h2(\x, \x*\l) : (
      (\x <  3/\l) ? \e / (8 * sqrt(2 * pi)) * h3(\x, \x*\l) : (
      sqrt(2/pi) * \e)))));
  };
}

\def\cdf(#1)(#2)(#3){0.5*(1+(erf((#1-#2)/(#3*sqrt(2)))))}%
% to be used: \cdf(x)(mean)(variance)

\DeclareMathOperator{\CDF}{cdf}
\tikzset{
    declare function={
        normcdf(\x,\m,\s)=1/(1 + exp(-0.07056*((\x-\m)/\s)^3 - 1.5976*(\x-\m)/\s));
    }
}

\begin{document}

%\begin{flushright}
%  \noindent \begin{tabular}{p{2.8in} r l}
\title{\vspace{-5ex}{\sc \examnum}}
\author{\bf \class}
\date{October 14, 2025\\[4mm]
   {\sc time limit:} {\bf \sc Two Hours}
}
% \end{tabular}\\
%\end{flushright}
%
\clearpage
\maketitle
\noindent\rule[1ex]{\textwidth}{.1pt}
\subsection*{Name}
Please print your name clearly in the box below.\\
\begin{center}
  \framebox(300,40){\Huge\phantom{t}} \\
  \vspace{2ex}
\end{center}

\vspace{40ex}
{\it Turn to the next page to read the instructions.}
\thispagestyle{empty}


\eject
\subsection*{Instructions}
This exam contains\textbf{ 22 pages} (including the front and back pages) and \textbf{10 problems, 106 points} (with 5 points extra credit). You have {\bf 2 hours} to complete it.  You may print out the PDF, complete it and upload as a
PDF on Moodle, or \textit{neatly} answer the questions on blank pages of paper, scan and upload.
% Check to see if any pages are missing.
% Make sure you have written your name on the front page.
% If for any reason you have any loose pages, put your initials on the
% top of these pages.\\

\noindent This is an \textbf{open resource examination}. You are expected to complete the exam individually. Asking
anyone (colleague, friend, tutor, etc) questions about the exam is \textit{not allowed}. If any questions arise during the exam, direct them to me (via email).
\\
 

\noindent The following rules apply: 
% \begin{minipage}[t]{3.7in}
% \vspace{0pt}
\begin{itemize}

%\item \textbf{If you use a ``fundamental theorem'' you must indicate this} and explain
%why the theorem may be applied.

\item \textbf{Organize your work}, in a reasonably neat and coherent way.% , in
% the space provided
 Work scattered all over the page without a clear ordering will
receive very little credit.

\item \textbf{Show ALL your work where appropriate}.
  The work you show will be evaluated as well as your final answer.
  Thus, provide ample justification for each step you take.
  Indicate when you have used a probability table or MATLAB/Python to obtain a result.
  In the case of MATLAB/Python, briefly include the function or statement you used to arrive at your result.
  In the long response questions, simply putting down an answer without showing your steps  will not merit full credit.
  {\bf EXCEPTION:} For short response or ``True/False'' questions, \textit{no explanations are required}.
  However, the more work you show, the greater your chance of receiving partial credit if your final answer is incorrect.

\item If you need more space, use the blank pages at the end, and clearly indicate when and where you have done this.

\item Questions are roughly in order of the lectures, so later questions may not necessarily be harder.
  If you are stuck on a problem, it may be better to skip it and get to it later.

\item Manage your time wisely.% Do not spend too much time on problems with fewer points.

\end{itemize}

%\noindent Do not write in the table to the right.
%\end{minipage}
%\hfill
%\begin{minipage}[t]{2.3in}
% \vspace{0pt}
% %\cellwidth{3em}
% \gradetablestretch{2}
% \vqword{Problem}
% \addpoints % required here by exam.cls, even though questions haven't started yet.
% \gradetable[v]%[pages]  % Use [pages] to have grading table by page instead of question
  % \vspace{0pt}
  % \renewcommand{\arraystretch}{2.8}
  % \quad  \begin{tabular}{| c | c| c|}\hline
  %   Problem  & Points & Score \\\hline
  %          1 & 10 & \quad\quad~~~\\\hline
  %          2 & 12 & \\\hline
  %          3 & 10 & \\\hline
  %          4 & 18 & \\\hline
  %         \bf Total & \bf 50 & \\\hline
  % \end{tabular}
%\end{minipage}
\newpage % End of cover page


% \lstset{language=C++,
%                 basicstyle=\tiny\ttfamily,
%                 keywordstyle=\color{blue}\ttfamily,
%                 stringstyle=\color{red}\ttfamily,
%                 commentstyle=\color{gray}\ttfamily,
%                 morecomment=[l][\color{gray}]{\#}
% }


% \thispagestyle{empty}

% \cellwidth{10em}
% \gradetablestretch{2}
% \vqword{Problem}
% \addpoints % required here by exam.cls, even though questions haven't started yet.
% \gradetable[v]%[pages]  % Use [pages] to have grading table by page instead of question
% \vspace{0pt}

% \thispagestyle{empty}
% \noindent{\it Do not write anything on this page. Please turn over.}\\[5mm]

% \begin{table*}[h!]
%   \center
%   \LARGE
%   \renewcommand{\arraystretch}{1.5}
%   \quad
%   \begin{tabular}{R{1.5in} R{1.5in} R{1.5in}}\toprule 
%     \bf \it Problem  & \bf Score & \bf Points \\\midrule
%           \it 1 & & \bf 6 \\\midrule
%           \it 2 & & \bf 10  \\\midrule
%           \it 3 & & \bf 5  \\\midrule
%            \it 4 & & \bf 5  \\\midrule
%            \it 5 & & \bf 7\\\midrule
%            \it 6 & & \bf 7  \\\midrule
%            \it 7 & & \bf 10 \\\midrule
%     \bf \it TOTAL & & \bf 50  \\\bottomrule
%   \end{tabular}

%   \end{table*}
% \newpage

\section*{Problem 1 \quad {\it True/False questions (15 points)}}
Respond ``T'' ({\it True})  or  ``F'' (\textit{False}) to the following statements.
Use the boxes provided. Each response is worth 1 point.
Note that a statement can only be regarded as true in this framework if it always holds in all circumstances.
If a statement does not hold under a given condition not already explicitly excluded, then it should be regarded as false.

\begin{enumerate}[\bf (i)]
\item \hfill
  \begin{minipage}{.1\linewidth}
    \framebox(40,40){ \gr } %F
  \end{minipage}\quad
  \begin{minipage}{.85\linewidth}
    The probability of the union of mutually exclusive events is 1.
  \end{minipage}
  \smallskip  
\item \hfill
  \begin{minipage}{.1\linewidth}
    \framebox(40,40){ \gr } % T
  \end{minipage}\quad
  \begin{minipage}{.85\linewidth}
   If two events $A$ and $B$ are collectively exhaustive, then $P(A) - P(B) = 1$.%F
  \end{minipage}
  \smallskip
\item \hfill
  \begin{minipage}{.1\linewidth}
    \framebox(40,40){ \gr }% F
  \end{minipage}\quad
  \begin{minipage}{.85\linewidth}
    For a given online account, your password must have seven small or capital letters. Assuming you are the first user to create an account, the number of possibilities for your password is $52^{7}$.
  \end{minipage}
  \smallskip
\item \hfill
  \begin{minipage}{.1\linewidth}
    \framebox(40,40){ \gr } %T
  \end{minipage}\quad
  \begin{minipage}{.85\linewidth}
    The area under the curve of a PDF can be less than 1.
  \end{minipage}  
  \smallskip
\item \hfill
  \begin{minipage}{.1\linewidth}
    \framebox(40,40){ \gr } %F
  \end{minipage}\quad
  \begin{minipage}{.85\linewidth}
    You begin your turn in a board game by rolling a six-sided die. If all outcomes are equally likely, then the probability of rolling a ``2'' is $\fr26$.
  \end{minipage}  
  \smallskip
\item \hfill
  \begin{minipage}{.1\linewidth}
    \framebox(40,40){\gr  } %T
  \end{minipage}\quad
  \begin{minipage}{.85\linewidth}
    In a right-skewed distribution, the mean is not equal to the median.
  \end{minipage}
  \smallskip
\item \hfill
  \begin{minipage}{.1\linewidth}
    \framebox(40,40){\gr } %F
  \end{minipage}\quad
  \begin{minipage}{.85\linewidth}
    $P(A\cup B) = P(A) + P(B)$ for two events $A$ and $B$ that are mutually exclusive.
  \end{minipage}
  \smallskip  
\item \hfill
  \begin{minipage}{.1\linewidth}
    \framebox(40,40){\gr } %F
  \end{minipage}\quad
  \begin{minipage}{.85\linewidth}
    If $A$ and $B$ are statistically independent events, then $P(AB)$ is always equal to $P(A)P(B)$.
  \end{minipage}
  \smallskip
\item \hfill
  \begin{minipage}{.1\linewidth}
    \framebox(40,40){\gr } %T
  \end{minipage}\quad
  \begin{minipage}{.85\linewidth}
   The complement of the union of collectively exhaustive events yields an empty set.
  \end{minipage}  
  \smallskip  
  \item \hfill
    \begin{minipage}{.1\linewidth}
      \framebox(40,40){\gr  } %F
    \end{minipage}\quad
    \begin{minipage}{.85\linewidth}
      The maximum value of any cumulative distribution function is $1$.
     \end{minipage}  
  \smallskip  
\item \hfill
  \begin{minipage}{.1\linewidth}
    \framebox(40,40){\gr } %F
  \end{minipage}\quad
  \begin{minipage}{.85\linewidth}
    Under certain conditions, a binomial distribution with parameters $(n,p)$ can be approximated by a normal
    distribution with $\sigma = np(1-p)$.
  \end{minipage}  
  \smallskip
\item \hfill
  \begin{minipage}{.1\linewidth}
    \framebox(40,40){ \gr }% F
  \end{minipage}\quad
  \begin{minipage}{.85\linewidth}
    The probability of the number of tails that occur in 100 tosses of a coin (which can yield either a head or a tail)
    can be appropriately modeled by a binomial distribution.
  \end{minipage}
  \smallskip
  \item \hfill
  \begin{minipage}{.1\linewidth}
    \framebox(40,40){\gr  } %T
  \end{minipage}\quad
  \begin{minipage}{.85\linewidth}
    Given a normal distribution with parameters $\mu$ and $\sigma$, the median of the distribution is $\mu^{2}$.
  \end{minipage}
  \smallskip   
  \item \hfill
    \begin{minipage}{.1\linewidth}
      \framebox(40,40){\gr } %F
    \end{minipage}\quad
    \begin{minipage}{.85\linewidth}
      The standard normal variate $Z$ has a variance of 1.
    \end{minipage}
  \smallskip
  \item \hfill
    \begin{minipage}{.1\linewidth}
      \framebox(40,40){\gr } %F
    \end{minipage}\quad
    \begin{minipage}{.85\linewidth}
      The probability of the elapsed time between events that constitute a Poisson process can be modeled using an exponential distribution.
    \end{minipage}
    % \smallskip    
    % \item \hfill
    % \begin{minipage}{.1\linewidth}
    %   \framebox(40,40){\gr } %T
    % \end{minipage}\quad
    % \begin{minipage}{.85\linewidth}
    %   If a variable $X$ is lognormally distributed with parameters $\mu$ and $\sigma$, then the median of $X$ is given by $\exp(\mu)$.
    % \end{minipage}

    
  \end{enumerate}
  \eject

\section*{Problem 2 \quad {\it Venn diagrams (4 points)}}
Write the combination of events (using set notation)  depicted in each of the figures below.
  \begin{enumerate}[\bf (a)]
\begin{minipage}{.47\linewidth}
\item  ~
  \begin{center}
      \begin{tikzpicture}[baseline={($(current bounding box.north)-(0,1.6ex)$)},scale=.4]
    \draw[thick]  (-4,-3) rectangle (7,3) node[above left] {$\bm S$};
    \draw[thick] (0,0) circle (2 cm) node[left] {$\bm{A}$};
    \draw[thick] (3,0) circle (2 cm) node[right] {$\bm{B}$};

    \begin{scope}
      %\clip (0,0) circle (2 cm);
      \fill[ pattern=north west lines, opacity=.75](3,0) circle (2 cm);
     % \fill[ pattern=north west lines](0,0) circle (2 cm);      
    \end{scope}
  \end{tikzpicture} 
\end{center}
  \vspace{2ex}
  \framebox(200,30){\Huge\phantom{t}}    

\item ~
  \begin{center}
      \begin{tikzpicture}[baseline={($(current bounding box.north)-(0,1.6ex)$)},scale=.4]
   \draw[thick]  (-4,-3) rectangle (7,3) node[above left] {$\bm S$};
        \fill[ pattern=north west lines, opacity=.75](-4,-3) rectangle (7,3) ;
    \filldraw[thick,fill=white] (0,0) circle (2 cm) node[left] {$\bm{A}$};
    \draw[thick] (3,0) circle (2 cm) node[right] {$\bm{B}$};
    \begin{scope}
    \end{scope}
  \end{tikzpicture}
\end{center}
  \vspace{2ex}
     \framebox(200,30){\Huge\phantom{t}}     

\item ~
  \begin{center}
      \begin{tikzpicture}[baseline={($(current bounding box.north)-(0,1.6ex)$)},scale=.5]
% Draw the main rectangle
\draw[thick, black] (0,0) rectangle (9,6) node[above left] {$\bm S$};

% Partition the rectangle into 3 regions
% Vertical partitions creating 3 equal regions
\draw[thick] (3,0) -- (3,6);
\draw[thick] (6,0) -- (6,6);

% Label the regions
\node  at (1.5,.5) {$\bm{A}$};
\node  at (4.5,.5) {$\bm{B}$};
\node at (7.5,.5) {$\bm{C}$};

% Draw the ellipse E (centered, spanning multiple regions)
\draw[thick, blue, line width=1.5pt] (4.5,3) ellipse (3 and 2);

% Label the ellipse
\node[blue] at (2,5) {$\bm E$};

% Shade the intersection of ellipse E with region C
\begin{scope}
\clip (6,0) rectangle (9,6);
\fill[pattern=north east lines, pattern color=black, opacity=0.6] (4.5,3) ellipse (3 and 2);
\end{scope}

% Alternative: solid fill instead of pattern
% \begin{scope}
% \clip (6,0) rectangle (9,6);
% \fill[red!40, opacity=0.8] (4.5,3) ellipse (3 and 2);
% \end{scope}

% Redraw boundaries for clarity
\draw[thick] (3,0) -- (3,6);
\draw[thick] (6,0) -- (6,6);
\draw[thick, black] (0,0) rectangle (9,6);
\draw[thick, blue, line width=1.5pt] (4.5,3) ellipse (3 and 2);

\end{tikzpicture}
\end{center}
  \vspace{2ex}
     \framebox(200,30){\Huge\phantom{t}}    
      
 \end{minipage}\hfill
\begin{minipage}
  {.47\linewidth}
\item ~
  \begin{center}
      \begin{tikzpicture}[baseline={($(current bounding box.north)-(0,1.6ex)$)},scale=.5]
    % Draw the main rectangle
    \draw[thick, black] (0,0) rectangle (9,6) node[above left] {$\bm S$};

    % Partition the rectangle into 3 regions
    % Vertical partitions creating 3 equal regions
    \draw[thick] (3,0) -- (3,6);
    \draw[thick] (6,0) -- (6,6);

    % Label the regions
    \node  at (1.5,.5) {$\bm{A}$};
    \node  at (4.5,.5) {$\bm{B}$};
    \node at (7.5,.5) {$\bm{C}$};

    % Draw the ellipse E (centered, spanning multiple regions)
    \draw[thick, blue, line width=1.5pt] (4.5,3) ellipse (3 and 2);

    % Label the ellipse
    \node[blue] at (2,5) {$\bm E$};

    % Shade region B outside the intersection with E
    \begin{scope}
    \clip (3,0) rectangle (6,6);
    \fill[pattern=north east lines, pattern color=black, opacity=0.6] (0,0) rectangle (9,6);
    \end{scope}

    % Remove shading from the intersection of B and E
    \begin{scope}
    \clip (3,0) rectangle (6,6);
    \fill[white] (4.5,3) ellipse (3 and 2);
    \end{scope}

    % Redraw boundaries for clarity
    \draw[thick] (3,0) -- (3,6);
    \draw[thick] (6,0) -- (6,6);
    \draw[thick, black] (0,0) rectangle (9,6);
    \draw[thick, blue, line width=1.5pt] (4.5,3) ellipse (3 and 2);

    \end{tikzpicture}
 \end{center}
   
  \vspace{2ex}
     \framebox(200,30){\Huge\phantom{t}}     
 
  \bigskip
 
  \item 
  \begin{center}
    \begin{tikzpicture}[scale=.8]
    \filldraw[thick, pattern=north west lines]  (-4,-5.5) rectangle (7,3)  node[above left] {$\bm S$};
    % \draw[thick] (0,0) circle (2 cm);% node[left] {$\bm{A}$};
    % \draw[thick] (3,0) circle (2 cm);% node[right] {$\bm{B}$};
    % \draw[thick] (1.5,-2.6) circle (2 cm);% node[] {$\bm{C}$};
    \filldraw[thick,fill=white] (0,0) circle (2 cm) node[left] {$\bm{A}$};
    \draw[thick] (3,0) circle (2 cm) node[right] {$\bm{B}$};
    \draw[thick] (1.5,-2.6) circle (2 cm) node[] {$\bm{C}$};

    %    \begin{scope}
    %   \clip (0,0) circle (2 cm);
    %   \clip (1.5,-2.6) circle (2 cm);
    %   \clip (3,0) circle (2 cm);      
    %  % \fill[white]  (3,0) circle (2 cm);
    % \end{scope}
    
    % \begin{scope}
    %   \clip (0,0) circle (2 cm);
    %   \fill[pattern=north west lines, ]  (1.5,-2.6) circle (2 cm);      
    % \end{scope}

    % \begin{scope}
    %   \clip (1.5,-2.6) circle (2 cm);
    %    \clip (3,0) circle (2 cm);
    %    \fill[pattern=north west lines,]  (1.5,-2.6) circle (2 cm);      
    %  \end{scope}
    % \begin{scope}
    %   \clip (0,0) circle (2 cm);
    %   \fill[pattern=north west lines, ]  (3,0) circle (2 cm);      
    % \end{scope}

   \end{tikzpicture}
 \end{center}
  \vspace{2ex}

     \framebox(300,40){\Huge\phantom{t}}

  
\end{minipage}
\end{enumerate} 

\eject
\section*{Problem 3 \quad {\it Short answer questions (22 points)}}

\begin{enumerate}[\bf (a)]
\item The PMF of a random variable $X$ is given in the figure below. 
  Use the figure to answer the following questions.
    
  \begin{figure}[h!]
    \centering
    \begin{tikzpicture}
      %\pgfplotsset{minor grid style = {line width = 0.1pt}}
      %\pgfplotsset{major grid style = {line width = 0.4pt}}
      \begin{axis}[
        axis x line=center,
        axis y line=center,
        xtick={0,2,...,19},
        ytick={0.02,0.04,...,0.16},
        domain = 0:18,
        samples = 19,
        xlabel={$x$},
        ylabel={$P(X=x)$},
        xlabel style={right},
        ylabel style={above},
        ymax=0.15,
        xmax=20,
        x post scale=1.4,
        grid=both,
        grid style={line width=.1pt, draw=gray!50!black},
        yticklabel style={
            /pgf/number format/fixed,
            /pgf/number format/fixed zerofill,
            /pgf/number format/precision=2
          }
       % minor tick num =2
        ]
        \addplot+[ycomb,black,ultra thick,opacity=.5,mark options={scale=1.5,opacity=.8}] {poiss(8))}; %\addlegendentry{$\lambda = 1$}}
      \end{axis}
    \end{tikzpicture}
  \end{figure}
  \vspace{2ex}

  \begin{enumerate}[\bf (i)]
    \item Estimate the probability $P(X = 4)$. \pts{1} % Answer: 0.24

    \begin{minipage}[]{.8\linewidth}
      \framebox(300,40){\Huge\phantom{t}}     
    \end{minipage}
  
    \bigskip
    \bigskip

  \item Estimate the probability $P(6 < X \le 8)$. \pts{2}  
  
  \begin{minipage}[]{.1\linewidth}
    {\bf Answer:}
  \end{minipage}\qquad
  \begin{minipage}[]{.8\linewidth}
    \framebox(300,60){\Huge\phantom{t}}     
  \end{minipage}

  \bigskip
  \bigskip

  \item If the distribution shown has a single parameter $\lambda$, which of the following would be your best guess for $\lambda$? (Circle or underline the correct answer.)\pts{1}
 \begin{enumerate}[A.]
  \item 2
  \item 8
  \item 10
  \item 12
 \end{enumerate} 
  \end{enumerate}

  \eject
\item Shade the area under the curve that represents the probability $P( X > 8)$. \marginpar{\it[1]}

  \medskip

  \begin{figure}[h!]
    \centering
      \begin{tikzpicture}
    \begin{axis}[no markers, domain=0:10, samples=100,
      axis lines*=left, xlabel=$x$, ylabel=$f_X(x)$,,
      height=6cm, width=10cm,
      %xtick={-3, -2, -1, 0, 1, 2, 3},
      %ytick=\empty,
      enlargelimits=false, clip=false, axis on top,
      grid style={line width=.1pt, draw=gray},
      yticklabel style={
            /pgf/number format/fixed,
            /pgf/number format/fixed zerofill,
            /pgf/number format/precision=2
          },        
      grid = major]
      \addplot [ultra thick, domain=-2:12] {gauss(5,2)};

    \end{axis}
  \end{tikzpicture}

\end{figure}
  

\item Shade the area under the curve that represents the probability $P( 4 \le X < 6)$. \marginpar{\it[1]}

  \medskip

  \begin{figure}[h!]
    \centering
      \begin{tikzpicture}
    \begin{axis}[no markers, domain=0:10, samples=100,
      axis lines*=left, xlabel=$x$, ylabel=$f_X(x)$,,
      height=6cm, width=10cm,
      %xtick={-3, -2, -1, 0, 1, 2, 3},
      %ytick=\empty,
      enlargelimits=false, clip=false, axis on top,
      grid style={line width=.1pt, draw=gray},
      yticklabel style={
            /pgf/number format/fixed,
            /pgf/number format/fixed zerofill,
            /pgf/number format/precision=2
          },        
      grid = major]
      \addplot [ultra thick, domain=-2:12] {gauss(5,2)};

    \end{axis}
  \end{tikzpicture}

\end{figure}


\item If the curve in part {\bf (c)} is the PDF of a normal distribution, what is its mean value? \pts{1}

\begin{minipage}[]{.1\linewidth}
  {\bf Answer:}
\end{minipage}\qquad
\begin{minipage}[]{.8\linewidth}
  \framebox(300,40){\Huge\phantom{t}}     
\end{minipage}

\eject

\item Shade the area under the curve that gives you the probability $P( X \le 1 \cap X > 2)$. \marginpar{\it[2]}
    \begin{figure}[h!]
    \centering
    \begin{tikzpicture}
      \begin{axis}[no marks,
        samples = 100,
        axis x line=center,
        axis y line=center,
        xtick={0,1,...,12},
        ytick={0.2,0.4,...,1},
        domain = 0:12,
        xlabel={$t$},
        ylabel={$f(t)$},
        xlabel style={right},
        ylabel style={above },
        ymax=1.2,
        xmax=12,
        x post scale=1.4,
        grid style={line width=.1pt, draw=gray},
        grid=both,
        yticklabel style={
          /pgf/number format/fixed,
          /pgf/number format/fixed zerofill,
          /pgf/number format/precision=1
        }
        ]
        \addplot+[blue, ultra thick,opacity=1] {expo(.75,x)}; % \addlegendentry{\large $\bm{f_{U}}$}
      \end{axis}
    \end{tikzpicture}
  \end{figure}


\item The figure shows the graph of the CDF of an exponential random variate.\pts{1}
  What is the 30th percentile of this distribution? 
\begin{figure}[h!]
  \centering
    \begin{tikzpicture}[
    declare function={expcdf(\t,\l)= 1 - exp(-\l*\t);}
    ]
    \begin{axis}[
      samples at={0,.01,...,2},
      xlabel=$t$,
      ylabel=$F_{T}(t)$,
      xlabel style={right},
      ylabel style={above right},
  %    xtick={0,20,...,100},
      ytick={0.0,0.1,...,1},
      %axis x line=center,
      %axis y line=center,
      ymin=0,
      xmin= 0,
      xmax = 2,
      ymax =1,
      grid=both,
      minor tick num=5,
      major grid style={line width=1pt,draw=gray},
      grid style={line width=.2pt, draw=gray!50},
      %y post scale = .7,
      x post scale=1.8
      %legend style={at={(1.25, 1)},anchor=north east},
      % yticklabel style={
      %   /pgf/number format/fixed,
      %   /pgf/number format/fixed zerofill,
      %   /pgf/number format/precision=2
      % }
      ]
      \addplot+[ultra thick,samples=100,no markers] {expcdf(x,5))};
      %\addlegendentry{$N(\mu =12, \sigma^2=4.8)$};}
    \end{axis}
  \end{tikzpicture}

  \caption{CDF of an exponential random variate}
  \label{fig:cdf}
\end{figure}

  \vspace{2ex}
  \begin{minipage}[]{.1\linewidth}
    {\bf Answer:}
  \end{minipage}\qquad
  \begin{minipage}[]{.8\linewidth}
    \framebox(300,40){\Huge\phantom{t}}     
  \end{minipage}

\eject

\item Below is the CDF of a given normal distribution. Use the figure to answer the following 5 questions (i) -- (v).

  \begin{figure}[h!]
    \centering
      \begin{tikzpicture}
    \begin{axis}[no markers, domain=1:7, samples=100,
      axis lines*=left, xlabel=$x$, ylabel=$F_X(x)$,,
      height=8cm, width=16cm,ymax=1,
      xtick={1, 1.5, ..., 7},
      ytick={0,0.125,.25,...,1},
      enlargelimits=false, clip=false, %axis on top,
      grid style={line width=.1pt, draw=gray!40},
      major grid style={line width=.7pt, draw=gray},
      yticklabel style={
            /pgf/number format/fixed,
            /pgf/number format/fixed zerofill,
            /pgf/number format/precision=3
          },        
      grid = both,minor tick num=4]
      % addplot [ultra thick, domain=-4:12] {gauss(4,3)};
      \addplot+[blue, ultra thick, smooth] {normcdf(x, 4,1)};
     % \addplot [dashed, ultra thick, green!50!black] coordinates {(-1, 0.25) (.65,.25)};
     % \addplot [dashed, ultra thick, green!50!black] coordinates {(.65, 0.25) (.65,0)};
  
%      \addplot [thick,draw=none, pattern=north west lines,  domain=6:12] {gauss(4,3)} \closedcycle;
%      \addplot [dashed, ultra thick,] coordinates {(6,0) (6,{gauss(5,2)})};
    \end{axis}
  \end{tikzpicture}

\end{figure}

\begin{enumerate}[\bf(i)]
\item What is the mean of this distribution? \pts{1}
  
\begin{minipage}[]{.1\linewidth}
    {\bf Answer:}
  \end{minipage}\qquad
  \begin{minipage}[]{.8\linewidth}
    \framebox(300,40){\phantom{\Huge t} }     
  \end{minipage}
  \bigskip
  
\item What is the mode of this distribution? \pts{1}
  
\begin{minipage}[]{.1\linewidth}
    {\bf Answer:}
  \end{minipage}\qquad
  \begin{minipage}[]{.8\linewidth}
    \framebox(300,40){\phantom{\Huge t} }     
  \end{minipage}
  \bigskip
  
\item Estimate the first quartile. \pts{1}
  
 \begin{minipage}[]{.1\linewidth}
    {\bf Answer:}
  \end{minipage}\qquad
  \begin{minipage}[]{.8\linewidth}
    \framebox(300,40){\Huge\phantom{t}}     
  \end{minipage}

  \bigskip
\item Estimate the third quartile. \pts{1}
  
 \begin{minipage}[]{.1\linewidth}
    {\bf Answer:}
  \end{minipage}\qquad
  \begin{minipage}[]{.8\linewidth}
    \framebox(300,40){\Huge\phantom{t}}     
  \end{minipage}

  \bigskip
\item Compute the interquartile range (IQR). \pts{1}
  
 \begin{minipage}[]{.1\linewidth}
    {\bf Answer:}
  \end{minipage}\qquad
  \begin{minipage}[]{.8\linewidth}
    \framebox(300,40){\Huge\phantom{t}}     
  \end{minipage}
\end{enumerate}
  \eject
% \item The median of a lognormal variate is $6$. What is the mean of the associated normal variate? \pts{2}

%   \vspace{2ex}
%  \begin{minipage}[]{.1\linewidth}
%     {\bf Answer:}
%   \end{minipage}\qquad
%   \begin{minipage}[]{.8\linewidth}
%     \framebox(300,80){\Huge\phantom{t}}     
%   \end{minipage}

  
%   \eject
  
\item Consider the  PDFs of the \marginpar{\it[1]} epxonential random variates $T$, $U$, $V$ and $W$ (measured in hours) shown in the figure below. 
  Which of them  has the greatest  mean? %T_z
  \begin{figure}[h!]
    \centering
    \begin{tikzpicture}
      \begin{axis}[no marks,
        samples = 100,
        axis x line=center,
        axis y line=center,
        xtick={0,1,...,6},
        ytick={0,0.5,...,2},
        domain = 0:12,
        xlabel={$t$},
        ylabel={$f(t)$},
        xlabel style={right},
        ylabel style={above },
        ymax=2.2,
        xmax=6,
        x post scale=1.4,
        yticklabel style={
          /pgf/number format/fixed,
          /pgf/number format/fixed zerofill,
          /pgf/number format/precision=1
        }
        ]
        \addplot+[draw=black,ultra thick,opacity=1] {expo(.5,x)}; \addlegendentry{\large $\bm{T}$}        
        \addplot+[draw=orange,ultra thick,opacity=1] {expo(1,x)}; \addlegendentry{\large $\bm{U}$}
        \addplot+[draw=blue,dashed, ultra thick,opacity=1] {expo(1.5,x)}; \addlegendentry{\large $\bm{V}$}
        \addplot+[draw=red,dotted, ultra thick,opacity=1] {expo(2,x)}; \addlegendentry{\large $\bm{W}$}
      \end{axis}
    \end{tikzpicture}
  \end{figure}

  \vspace{2ex}
 \begin{minipage}[]{.1\linewidth}
    {\bf Answer:}
  \end{minipage}\qquad
  \begin{minipage}[]{.8\linewidth}
    \framebox(300,40){\Huge\phantom{t}}     
  \end{minipage}

  \bigskip
  \bigskip
  
\item In the figure above, which random \pts{1} variable has standard deviation of 2 hours?  
   
  \vspace{2ex}
 \begin{minipage}[]{.1\linewidth}
    {\bf Answer:}
  \end{minipage}\qquad
  \begin{minipage}[]{.8\linewidth}
    \framebox(300,40){\Huge\phantom{t}}     
  \end{minipage}
  

   \bigskip
   \bigskip
  
\item In which of the  distributions does the mode appear to be equal to the mean? \pts{1}
  Circle (a), (b) or (c). \\ %(b)
  
  \begin{figure}[h!]
    \centering
    \begin{subfigure}{.3\textwidth}
      \caption{}
      \begin{tikzpicture}[
        declare function={gamma(\z)=
          2.506628274631*sqrt(1/\z)+ 0.20888568*(1/\z)^(1.5)+ 0.00870357*(1/\z)^(2.5)-
          (174.2106599*(1/\z)^(3.5))/25920- (715.6423511*(1/\z)^(4.5))/1244160)*exp((-ln(1/\z)-1)*\z;},
        declare function={gammapdf(\x,\k,\theta) = 1/(\theta^\k)*1/(gamma(\k))*\x^(\k-1)*exp(-\x/\theta);}
        ]

        \begin{axis}[
          no markers, domain=-5:5, samples=100,
          axis lines=left, xlabel=$m$, ylabel=$f_M(m)$,
          %very axis y label/.style={at=(current axis.above origin),anchor=east},
          %every axis x label/.style={at=(current axis.right of origin),anchor=north},
          height=4cm, width=6cm,
          xtick=\empty, ytick=\empty,
          enlargelimits=false, clip=false, axis on top,
          %xlabel style={right},
          %ylabel style={above},
          grid = major,
          yticklabel style={
            /pgf/number format/fixed,
            /pgf/number format/fixed zerofill,
            /pgf/number format/precision=2
          }
          ]
          \addplot [very thick, domain=-5:5] {gauss(0,1)};
        \end{axis}
      \end{tikzpicture}

    \end{subfigure}
    \begin{subfigure}{.3\textwidth}
            \caption{}
      \begin{tikzpicture}[
        declare function={gamma(\z)=
          2.506628274631*sqrt(1/\z)+ 0.20888568*(1/\z)^(1.5)+ 0.00870357*(1/\z)^(2.5)-
          (174.2106599*(1/\z)^(3.5))/25920- (715.6423511*(1/\z)^(4.5))/1244160)*exp((-ln(1/\z)-1)*\z;},
        declare function={gammapdf(\x,\k,\theta) = 1/(\theta^\k)*1/(gamma(\k))*\x^(\k-1)*exp(-\x/\theta);}
        ]

        \begin{axis}[
          no markers, domain=0:9, samples=100,
          axis lines=left, xlabel=$n$, ylabel=$f_N(n)$,
          %every axis y label/.style={at=(current axis.above origin),anchor=east},
          %every axis x label/.style={at=(current axis.right of origin),anchor=north},
          height=4cm, width=6cm,
          xtick=\empty, ytick=\empty,
          % xticklabels={$n^*$},
          % xticklabels={$\bar n (\theta_t)$},
          enlargelimits=false, clip=false, axis on top,
          %xlabel style={right},
          %ylabel style={above},
          grid = major,
          yticklabel style={
            /pgf/number format/fixed,
            /pgf/number format/fixed zerofill,
            /pgf/number format/precision=2
          }
          ]
          \addplot [very thick,cyan!20!black,domain=0:20] {gammapdf(x,2,2)};
          %\addplot [fill=cyan!20, draw=none, domain=0:6.0] {gammapdf(x,2,2)} \closedcycle;
          %\addplot [very thick, fill=white!20!white, draw=none, domain=6.01:20] {gammapdf(x,2,2)} \closedcycle;
        \end{axis}
      \end{tikzpicture}
    \end{subfigure}
    \begin{subfigure}{.3\textwidth}
      \caption{}
      \begin{tikzpicture}
        
        \begin{axis}[
          every axis plot post/.append style={
            mark=none, domain=-5.5:3.5, samples=200, very thick
          },
          axis lines=left, xlabel=$r$, ylabel=$f_R(r)$,
          height=4cm, width=6cm,
          xtick=\empty, ytick=\empty,
          enlargelimits=false, clip=false, axis on top
          ]
          \addplot[black]    {skewnorm(x, -5)};
          % \addplot[green]  {skewnorm(x, -2)};
          % \addplot[gray]   {skewnorm(x,  0)};
          % \addplot[blue]   {skewnorm(x,  2)};
          % \addplot[orange] {skewnorm(x,  4)};
          % \legend{$\lambda=-4$,$\lambda=-2$,$\lambda=0$,$\lambda=2$,$\lambda=4$}
        \end{axis}

      \end{tikzpicture}
    \end{subfigure}
  \end{figure}
 \eject
\item Which of the  distributions is right-skewed? \pts{1}
  Circle (a), (b) or (c). \\ %(b)
  
  \begin{figure}[h!]
    \centering
    \begin{subfigure}{.3\textwidth}
      \caption{}
      \begin{tikzpicture}[
        declare function={gamma(\z)=
          2.506628274631*sqrt(1/\z)+ 0.20888568*(1/\z)^(1.5)+ 0.00870357*(1/\z)^(2.5)-
          (174.2106599*(1/\z)^(3.5))/25920- (715.6423511*(1/\z)^(4.5))/1244160)*exp((-ln(1/\z)-1)*\z;},
        declare function={gammapdf(\x,\k,\theta) = 1/(\theta^\k)*1/(gamma(\k))*\x^(\k-1)*exp(-\x/\theta);}
        ]

        \begin{axis}[
          no markers, domain=-5:5, samples=100,
          axis lines=left, xlabel=$m$, ylabel=$f_M(m)$,
          %very axis y label/.style={at=(current axis.above origin),anchor=east},
          %every axis x label/.style={at=(current axis.right of origin),anchor=north},
          height=4cm, width=6cm,
          xtick=\empty, ytick=\empty,
          enlargelimits=false, clip=false, axis on top,
          %xlabel style={right},
          %ylabel style={above},
          grid = major,
          yticklabel style={
            /pgf/number format/fixed,
            /pgf/number format/fixed zerofill,
            /pgf/number format/precision=2
          }
          ]
          \addplot [very thick, domain=-5:5] {gauss(0,1)};
        \end{axis}
      \end{tikzpicture}

    \end{subfigure}
    \begin{subfigure}{.3\textwidth}
            \caption{}
      \begin{tikzpicture}[
        declare function={gamma(\z)=
          2.506628274631*sqrt(1/\z)+ 0.20888568*(1/\z)^(1.5)+ 0.00870357*(1/\z)^(2.5)-
          (174.2106599*(1/\z)^(3.5))/25920- (715.6423511*(1/\z)^(4.5))/1244160)*exp((-ln(1/\z)-1)*\z;},
        declare function={gammapdf(\x,\k,\theta) = 1/(\theta^\k)*1/(gamma(\k))*\x^(\k-1)*exp(-\x/\theta);}
        ]

        \begin{axis}[
          no markers, domain=0:9, samples=100,
          axis lines=left, xlabel=$n$, ylabel=$f_N(n)$,
          %every axis y label/.style={at=(current axis.above origin),anchor=east},
          %every axis x label/.style={at=(current axis.right of origin),anchor=north},
          height=4cm, width=6cm,
          xtick=\empty, ytick=\empty,
          % xticklabels={$n^*$},
          % xticklabels={$\bar n (\theta_t)$},
          enlargelimits=false, clip=false, axis on top,
          %xlabel style={right},
          %ylabel style={above},
          grid = major,
          yticklabel style={
            /pgf/number format/fixed,
            /pgf/number format/fixed zerofill,
            /pgf/number format/precision=2
          }
          ]
          \addplot [very thick,cyan!20!black,domain=0:20] {gammapdf(x,2,2)};
          %\addplot [fill=cyan!20, draw=none, domain=0:6.0] {gammapdf(x,2,2)} \closedcycle;
          %\addplot [very thick, fill=white!20!white, draw=none, domain=6.01:20] {gammapdf(x,2,2)} \closedcycle;
        \end{axis}
      \end{tikzpicture}
    \end{subfigure}
    \begin{subfigure}{.3\textwidth}
      \caption{}
      \begin{tikzpicture}
        
        \begin{axis}[
          every axis plot post/.append style={
            mark=none, domain=-5.5:3.5, samples=200, very thick
          },
          axis lines=left, xlabel=$r$, ylabel=$f_R(r)$,
          height=4cm, width=6cm,
          xtick=\empty, ytick=\empty,
          enlargelimits=false, clip=false, axis on top
          ]
          \addplot[black]    {skewnorm(x, -5)};
          % \addplot[green]  {skewnorm(x, -2)};
          % \addplot[gray]   {skewnorm(x,  0)};
          % \addplot[blue]   {skewnorm(x,  2)};
          % \addplot[orange] {skewnorm(x,  4)};
          % \legend{$\lambda=-4$,$\lambda=-2$,$\lambda=0$,$\lambda=2$,$\lambda=4$}
        \end{axis}

      \end{tikzpicture}
    \end{subfigure}
  \end{figure}


\item %$X \sim \mathcal{N}(\mu=3,\sigma^2=1)$. 
The figure below shows the PDF of a normal distribution with a variance of 1. \pts{3} Compute the probability indicated by the shaded portion of the PDF below.

    \begin{figure}[h!]
    \centering
      \begin{tikzpicture}
    \begin{axis}[no markers, domain=0:10, samples=100,
      axis lines*=left, xlabel=$x$, ylabel=$f_X(x)$,,
      height=6cm, width=10cm,
      xtick={-2, -1, ...,12 },
      %ytick=\empty,
      enlargelimits=false, clip=false, axis on top,
      grid style={line width=.1pt, draw=gray},
      yticklabel style={
            /pgf/number format/fixed,
            /pgf/number format/fixed zerofill,
            /pgf/number format/precision=2
          },        
      grid = major]
      \addplot [ultra thick, domain=-2:8] {gauss(3,1)};
      \addplot [thick,draw=none, pattern=north west lines,  domain=4:7] {gauss(3,1)} \closedcycle;
      %\addplot [dashed, ultra thick,] coordinates {(7,0) (7,{gauss(3,1)})};
      \addplot [ultra thick, dashed] coordinates {(4,0) (4,{gauss(3,1)})};      
    \end{axis}
  \end{tikzpicture}

\end{figure}

      \vspace{2ex}
  \begin{minipage}[]{.1\linewidth}
    {\bf Answer:}
  \end{minipage}\qquad
  \begin{minipage}[]{.8\linewidth}
    \framebox(390,120){\phantom{\Huge t}   }     
  \end{minipage}

%\item%  The number of ... that ... is binomially distributed. The probability of ... is  ...
  % Use your table to find the probability that at most 3 out of 7 cars will be defective...

% \item The figure below shows the CDF of a discrete random variable $X$. \marginpar{\it[2pts]}
%   To aid you in properly interpreting this plot, you are given that $P(X\le 2) = 0.2$.
%   What is the probability $P(X >3)$? %answer: 0.6
  
%   \begin{figure}[h!]
%     \centering
%     \begin{tikzpicture}
%       \begin{axis}[
%         xlabel=$x$, ylabel=$F_X(x)$,
%         clip=false,
%         jump mark left,
%         ymin=0,ymax=1,
%         xmin=0, xmax=5,
%         every axis plot/.style={very thick},
%         discontinuous,
%         grid = both,
%         grid style={line width=.2pt, draw=gray},
%         table/create on use/cumulative distribution/.style={
%           create col/expr={\pgfmathaccuma + \thisrow{f(x)}}   
%         }
%         ]
%         \addplot [black] table [y=cumulative distribution]{
%           x f(x)
%           0 1/15
%           1 2/15
%           2 1/5
%           3 4/15
%           4 1/3
%           5 0
%         };
% \end{axis}
% \end{tikzpicture}
% \end{figure}

%  \begin{minipage}[]{.1\linewidth}
%     {\bf Answer:}
%   \end{minipage}\qquad
%   \begin{minipage}[]{.8\linewidth}
%     \framebox(300,40){\Huge\phantom{t}}     
%   \end{minipage}
  
\end{enumerate}

\eject

\section*{Problem 4 \quad {\it Counting (7 points)}}



\begin{enumerate}[\bf (a)]
\item License plates in a certain state consist of \pts{3} 1 digit, followed by 3 letters and then 2 digits. How many different
license plates can be manufactured?
\vspace{40ex}

\item In how many ways can you arrange 6 items? \pts{2}
\vspace{20ex}

\item How many distinct groups of 3 can you obtain from a larger group of 5 items? \pts{2}
\vspace{40ex}
  
\end{enumerate}



\eject
\section*{Problem 5 \quad {\it Bayes' and total probability (9 points)}}
Given that $P (A) = 0.6$, $P (B) = 0.3$ and $P(C) = 0.1$ represent the production of machines in a
factory. The conditional probabilities of damaged items are $P(D|A) = 0.02$, $P (D|B) = 0.03$ and
$P (D|C) = 0.04$.

\begin{enumerate}[\bf (a)]
\item Find the total probability $P(D)$. \pts{3}
    \vspace{25ex}

  \item Find the \pts{3} probability that an item was produced by machine $B$, given that it is damaged.
    \vspace{25ex}

  \item Draw a Venn \pts{3} diagram depicting the  events $A$, $B$, $C$ and $D$ in sample space $S$.
    

\end{enumerate}

\eject
\section*{Problem 6 \quad {\it Binomial distribution (9 points; 5 points EC)}}
%%% Page 187 deVore #72

80\% of all vehicles inspected  at a certain facility station pass the inspection.
Assuming that successive vehicles pass or fail independently of one another, find the following:
\begin{enumerate}[\bf (a)]
\item The standard deviation \pts{2} of the binomial distribution governing the probability of the next 10 vehicles passing inspection.
  \vspace{20ex}
  
\item The probability that half of the next 10 vehicles inspected pass. \pts{3}
  \vspace{30ex}
  
\item The probability that at least 4 of the next 10 vehicles inspected fail. \pts{4} %% 1 - none failing
  \vspace{30ex}

  \eject

\item (Extra Credit) Use the normal distribution \pts{5} to estimate the probability that 95 of the next 100 vehicles inspected will pass inspection. (Show all your work to earn all the extra points.)
\end{enumerate}

\eject

\section*{Problem 7 \quad {\it Poisson distribution (8 points)}}

Hourly arrivals at a small coffee shop can be model using a Poisson distribution with a variance of 4.

\begin{enumerate}[\bf (a)]
\item What is the mean of the Poisson distribution?\pts{2}
  \vspace{20ex}

\item What is the probability that \pts{2} 5 customers will arrive in a given hour?
  \vspace{30ex}

\item Find the \pts{4} probability that no more than 8 customers will arrive over a duration of 2 consecutive hours.
  
\end{enumerate}

 
\eject

% \begin{quote}
%   \gr
%   \begin{align*}
%     P(X \ge 3) &= 1 - P(X\le 2) \\
%                &= 1 - \sum_{k=0}^2 {8 \choose k} (0.6)^2(0.4)^6 \\
%                &= 1 - 0.0498 \quad \text{(from table)} \\
%                &= \boxed{0.9502}
%   \end{align*}
% \end{quote}

\section*{Problem 8  \quad {\it Normal distribution (7 points)}}
The mean daily high temperature in June in LA is 77$^{\circ}$F with a  standard deviation of $5^{\circ}$F.  Suppose
that the temperatures in June closely follow a normal distribution.

\begin{enumerate}[\bf (a)]


\item What is the 50th \pts{1}  percentile of daily high temperatures in June in LA?
\vspace{10ex}

\item What is the \pts{3} probability that the high temperature on a random day in June in LA is lower than $60^{\circ}$F?
  \vspace{30ex}


  \item What is the \pts{3} probability that the high temperature on a random day in June in LA is between 60$^{\circ}$ and $80^{\circ}$F?
  \vspace{30ex}


\end{enumerate}


%\vspace{40ex}


\eject
\section*{Problem 9 \quad {\it Logormal distribution (10 points)}}

The lifetime (in years) of a machine is lognormally distributed with 
$\mu = 2$ and $\sigma = 0.5$.
\begin{enumerate}[(a)]
  \item  What is the median lifetime of the machine? \pts{2}
  \vspace{10ex}

\item Find the mean lifetime.\pts{2}
  \vspace{25ex}
  
\item What is the probability that the lifetime is at least 5 years? \pts{3}
  \vspace{30ex}

\item What is the probability that the lifetime is less than 2 years? \pts{3}
  
\end{enumerate}

\eject

\section*{Problem 10 \quad {\it Exponential distribution (8 points)}}
The delay time $T$ of a bus is exponentially distributed with $\la = 1$ (mean rate of occurrence per hour).


\begin{enumerate}[(a)]
\item What is the mean of $T$? \pts{1}
  \vspace{10ex}
  % \begin{quote}
  %   \gr
  %   \begin{align*}
  %     P(T \le 0.25) &= 1 - e^{- 3(0.25)} \\
  %                   &= \boxed{0.528}
  %   \end{align*}
  % \end{quote}

\item What is the variance of $T$? \pts{1}
  \vspace{10ex}

  
\item What is the probability that a train is delayed by no more than 15 minutes? \pts{3}
  \vspace{25ex}
  
\item Given that you have already waited for 30 minutes, \pts{3}
   what is the probability that a certain bus will be further delayed by more than 15 minutes?
%   \vspace{30ex}
%   % \begin{quote}
%   %   \gr
%   %   Use the memoryless property:
%   %   \begin{align*}
%   %     P(T > 1|T>0.5) &= P(T>0.5) \\
%   %                    &= e^{-3(0.5)} \\
%   %                    &\approx \boxed{0.777}
%   %   \end{align*}
%   % \end{quote}
\end{enumerate}

% \eject

% \section*{Problem 11 \quad {\it Joint distributions (11 points; 3 points EC)}}
% The joint PMF of two random variables $X$ and $Y$ is shown in the table below. \\

% \begin{center}
% \begin{tabular}{l l | l l l}
%   & & \multicolumn{3}{c}{$y$} \\
%   \multicolumn{2}{c|}{$p_{X,Y}(x,y)$}
%   & \bf 0 & \bf 1 &\bf  2 \\ \hline
%   \multirow{3}{*}{~~~~$x$~~~~}
%   &\bf 0 & 0.10 & 0.06 & 0.02 \\ 
%   &\bf 1 & 0.12 & 0.04 & 0.01 \\
%   &\bf 2 & 0.20 & 0.30 & 0.15 \\
% \end{tabular}
% \end{center}
% \medskip

% \begin{enumerate}[\bf (i)]
  
% \item Find $P(X = 0 \cap Y = 1)$.\pts{2}
%   \vspace{20ex}
  
% \item Compute $P(X < 1)$. \pts{3}
%    \vspace{20ex}

%  \item Compute $P(Y \ge 1)$. \pts{3}
%    \vspace{20ex}

%    \eject
% \item Compute $P(Y = 0 | X = 0)$. \pts{3}
% \vspace{40ex}

% \item (Extra Credit) Find the marginal distribution $p_{X}(x)$. \pts{3}
  
% \end{enumerate}

% \eject

% \section*{Problem 7 \quad {\it Functions of multiple random variables (10 points)}}
% %%% Page 160, dVore #21
% The differential settlement between two piles in a foundation is given by:
% \begin{equation}
%   \label{eq:2}
%   D  = S_1  - S_2
% \end{equation}
% where $S_1$ and $S_2$ are normally distributed.

% You are given the following:
% \begin{align}
%   \mu_{S_1} = \mu_{S_2} &= 3.0 \\
%   \sigma_{S_1} = \sigma_{S_2} &= 0.7 \\
%   \rho &= 0.8 
% \end{align}
% First, convince yourself that the mean of $D$ is given by
% \begin{equation}
%   \label{eq:3}
%   \mu_D = \mu_{S_1} - \mu_{S_2} = 3 - 3 = 0
% \end{equation}

% \begin{enumerate}[\bf (i)]
% \item Now, show that the variance of $D$ is equal to 0.196. \marginpar{\it [4pts]}
%   % \begin{quote}
%   %   \gr
%   %   \begin{align*}
%   %     Var(D) &= Var(S_1) + Var(S_2) - 2Cov(S_1,S_2) \\
%   %     Cov(S_1,S_2)
%   %            &= \rho\sigma_{S_1}\sigma_{S_2} \\
%   %            &= 0.8(0.7)(0.7) = 0.392 \\
%   %     \therefore Var(D) &= \sigma_{S_1}^2  + \sigma_{S_2}^2 - 2(0.392) \\
%   %            &= 0.7^2 + 0.7^2 - 0.784 \\
%   %            &= 0.98 - 0.784 \\
%   %            &= 0.196
%   %   \end{align*}
%   % \end{quote}
% \eject
% \item Find the probability that the magnitude of the differential settlement $|D|$\marginpar{\it [6pts]}
%    is no more than 0.5 inch. 
%   % \begin{quote}
%   %   \gr
%   %   \begin{align*}
%   %     P(|D| \le 0.5)  &= P(-0.5 < D \le 0.5) \\
%   %     &= \nmfr{0.5}{0}{\sqrt{0.196}} - \nmfr{-0.5}{0}{\sqrt{0.196}} \\
%   %     &= \Phi(1.129) - \Phi(-1.129)\\
%   %     &= \Phi(1.129) - (1 - \Phi(1.129)) \\
%   %     &= 0.871 - 1 + 0.871
%   %     &= \boxed{0.741}
%   %   \end{align*}
%   % \end{quote}
% \end{enumerate}


% \newpage

% \vfill
% \begin{center}
%   BLANK PAGE
% \end{center}
% \newpage

% \vfill
% \begin{center}
%   BLANK PAGE
% \end{center}

% \newpage
% \vfill
% \begin{center}
%   BLANK PAGE
% \end{center}

\newpage
~
\thispagestyle{empty}
\vfill
\begin{center}
  \includegraphics[width=1in]{umass-seal}
  
  {\sc
    CEE 260/MIE 273 $|$ Jimi Oke $|$ Fall 2024\\
    Department of Civil and Environmental Engineering \\
    University of Massachusetts Amherst
  }
\end{center}
\end{document}

%%% Local Variables:
%%% mode: latex
%%% TeX-master: t
%%% End:
