\documentclass[12pt,twoside]{article}
\usepackage{etex}

\raggedbottom

%geometry (sets margin) and other useful packages
\usepackage{geometry}
\geometry{top=1in, left=1in,right=1in,bottom=1in}
 \usepackage{graphicx,booktabs,ragged2e,calc}
 
\usepackage{listings}

% Marginpar width
%Marginpar width
\newcommand{\pts}[1]{\marginpar{ \small\hspace{0pt} \textit{[#1]} } } 
\setlength{\marginparwidth}{.5in}
%\reversemarginpar
%\setlength{\marginparsep}{.02in}

 
%\usepackage{cmbright}lstinputlisting
%\usepackage[T1]{pbsi}

%\renewcommand{\baselinestretch}{1.5}

\usepackage{chngcntr,mathtools}
%\counterwithin{figure}{section}
%\numberwithin{equation}{section}

%\usepackage{listings}

%AMS-TeX packages
\usepackage{amssymb,amsmath,amsthm} 
\usepackage{bm}
\usepackage[mathscr]{eucal}
\usepackage{colortbl}
\usepackage{color}


\usepackage{subcaption,hyperref,enumerate,polynom,polynomial}
\usepackage{multirow,minitoc,fancybox,array,multicol}

\definecolor{slblue}{rgb}{0,.3,.62}
\hypersetup{
    colorlinks,%
    citecolor=blue,%
    filecolor=blue,%
    linkcolor=blue,
    urlcolor=slblue
}

%%%TIKZ
\usepackage{tikz}

\usepackage{pgfplots,pgfplotstable}
\pgfplotsset{compat=newest}

\usetikzlibrary{calc}
\usetikzlibrary{arrows,shapes,positioning}
\usetikzlibrary{decorations.markings}
\usetikzlibrary{shadows}
\usetikzlibrary{patterns}
%\usetikzlibrary{circuits.ee.IEC}
\usetikzlibrary{decorations.text}
% For Sagnac Picture
\usetikzlibrary{%
    decorations.pathreplacing,%
    decorations.pathmorphing%
}

\tikzstyle arrowstyle=[black,scale=2]
\tikzstyle directed=[postaction={decorate,decoration={markings,
    mark=at position .65 with {\arrow[arrowstyle]{stealth}}}}]
\tikzstyle reverse directed=[postaction={decorate,decoration={markings,
    mark=at position .65 with {\arrowreversed[arrowstyle]{stealth};}}}]
\tikzstyle dir=[postaction={decorate,decoration={markings,
    mark=at position .98 with {\arrow[arrowstyle]{latex}}}}]
\tikzstyle rev dir=[postaction={decorate,decoration={markings,
    mark=at position .98 with {\arrowreversed[arrowstyle]{latex};}}}]

\usepackage{ctable}

%
%Redefining sections as problems
%
\makeatletter
\newenvironment{exercise}{\@startsection 
	{section}
	{1}
	{-.2em}
	{-3.5ex plus -1ex minus -.2ex}
    	{1.3ex plus .2ex}
    	{\pagebreak[3]%forces pagebreak when space is small; use \eject for better results
	\large\bf\noindent{Exercise 1.\hspace{-1.5ex} }
	}
	}
	%{\vspace{1ex}\begin{center} \rule{0.3\linewidth}{.3pt}\end{center}}
	%\begin{center}\large\bf \ldots\ldots\ldots\end{center}}
\makeatother

%
%Fancy-header package to modify header/page numbering 
%
\usepackage{fancyhdr}
\pagestyle{fancy}
%\addtolength{\headwidth}{\marginparsep} %these change header-rule width
%\addtolength{\headwidth}{\marginparwidth}
%\fancyheadoffset{30pt}
%\fancyfootoffset{30pt}
\fancyhead[LO,RE]{\small Prof. Oke}
\fancyhead[RO,LE]{\small Page \thepage} 
\fancyfoot[RO,LE]{\small Midterm} 
\fancyfoot[LO,RE]{\small \scshape CEE 260/MIE 273} 
\cfoot{} 
\renewcommand{\headrulewidth}{0.1pt} 
\renewcommand{\footrulewidth}{0.1pt}
%\setlength\voffset{-0.25in}
%\setlength\textheight{648pt}


\usepackage{paralist}

\newcommand{\osn}{\oldstylenums}
\newcommand{\lt}{\left}
\newcommand{\rt}{\right}
\newcommand{\pt}{\phantom}
\newcommand{\tf}{\therefore}
\newcommand{\?}{\stackrel{?}{=}}
\newcommand{\fr}{\frac}
\newcommand{\dfr}{\dfrac}
\newcommand{\ul}{\underline}
\newcommand{\tn}{\tabularnewline}
\newcommand{\nl}{\newline}
\newcommand\relph[1]{\mathrel{\phantom{#1}}}
\newcommand{\cm}{\checkmark}
\newcommand{\ol}{\overline}
\newcommand{\rd}{\color{red}}
\newcommand{\bl}{\color{blue}}
\newcommand{\pl}{\color{purple}}
\newcommand{\og}{\color{orange!90!black}}
\newcommand{\gr}{\color{green!40!black}}
\newcommand{\nin}{\noindent}
\newcommand{\la}{\lambda}
\renewcommand{\th}{\theta}
\newcommand*\circled[1]{\tikz[baseline=(char.base)]{
            \node[shape=circle,draw,thick,inner sep=1pt] (char) {\small #1};}}

\newcommand{\bc}{\begin{compactenum}[\quad--]}
\newcommand{\ec}{\end{compactenum}}

\newcommand{\n}{\\[2mm]}
%% GREEK LETTERS
\newcommand{\al}{\alpha}
\newcommand{\gam}{\gamma}
\newcommand{\eps}{\epsilon}
\newcommand{\sig}{\sigma}

\newcommand{\p}{\partial}
\newcommand{\pd}[2]{\frac{\partial{#1}}{\partial{#2}}}
\newcommand{\dpd}[2]{\dfrac{\partial{#1}}{\partial{#2}}}
\newcommand{\pdd}[2]{\frac{\partial^2{#1}}{\partial{#2}^2}}
\newcommand{\mr}{\mathbb{R}}
\newcommand{\xs}{x^{*}}
\newenvironment{solution}
{\medskip\par\quad\quad\begin{minipage}[c]{.8\textwidth}}{\medskip\end{minipage}}

\newcommand{\nmfr}[3]{\Phi\left(\frac{{#1} - {#2}}{#3}\right)}

% https://tex.stackexchange.com/questions/198383/drawing-cumulative-distribution-function-for-a-discrete-variable
\makeatletter
\long\def\ifnodedefined#1#2#3{%
    \@ifundefined{pgf@sh@ns@#1}{#3}{#2}%
}

\pgfplotsset{
    discontinuous/.style={
    scatter,
    scatter/@pre marker code/.code={
        \ifnodedefined{marker}{
            \pgfpointdiff{\pgfpointanchor{marker}{center}}%
             {\pgfpoint{0}{0}}%
             \ifdim\pgf@y>0pt
                \tikzset{options/.style={mark=*, fill=white}}
                \draw [densely dashed] (marker-|0,0) -- (0,0);
                \draw plot [mark=*] coordinates {(marker-|0,0)};
             \else
                \tikzset{options/.style={mark=none}}
             \fi
        }{
            \tikzset{options/.style={mark=none}}        
        }
        \coordinate (marker) at (0,0);
        \begin{scope}[options]
    },
    scatter/@post marker code/.code={\end{scope}}
    }
}

\makeatother

%%%%%%%%%%%%%%%%%%%%%%%%%%%%%%%%%%%%%%%%%%%%%%%%%%%
%%%%%%%%%%%%%%%%%%%%%%%%%%%%%%%%%%%%%%%%%%%%%%%%%%%
% Here's where you edit the Class, Exam, Date, etc.
\newcommand{\class}{CEE 260/MIE 273: Probability \& Statistics in Civil Engineering}
\newcommand{\term}{Fall 2025}
\newcommand{\examnum}{Midterm Exam}
\newcommand{\examdate}{10/08/25}
\newcommand{\timelimit}{2 Hours}


\newcolumntype{R}[1]{>{\RaggedLeft\arraybackslash}p{#1}}

\pgfmathdeclarefunction{poiss}{1}{%
  \pgfmathparse{(#1^x)*exp(-#1)/(x!)}%
  }

\pgfmathdeclarefunction{gauss}{2}{%
  \pgfmathparse{1/(#2*sqrt(2*pi))*exp(-((x-#1)^2)/(2*#2^2))}%
}

\pgfmathdeclarefunction{expo}{2}{%
  \pgfmathparse{#1*exp(-#1*#2)}%
}

\usetikzlibrary{math}

% https://tex.stackexchange.com/questions/461758/asymmetric-distribution-gauss-curve
\tikzmath{%
  function h1(\x, \lx) { return (9*\lx + 3*((\lx)^2) + ((\lx)^3)/3 + 9); };
  function h2(\x, \lx) { return (3*\lx - ((\lx)^3)/3 + 4); };
  function h3(\x, \lx) { return (9*\lx - 3*((\lx)^2) + ((\lx)^3)/3 + 7); };
  function skewnorm(\x, \l) {
    \x = (\l < 0) ? -\x : \x;
    \l = abs(\l);
    \e = exp(-(\x^2)/2);
    return (\l == 0) ? 1 / sqrt(2 * pi) * \e: (
      (\x < -3/\l) ? 0 : (
      (\x < -1/\l) ? \e / (8 * sqrt(2 * pi)) * h1(\x, \x*\l) : (
      (\x <  1/\l) ? \e / (4 * sqrt(2 * pi)) * h2(\x, \x*\l) : (
      (\x <  3/\l) ? \e / (8 * sqrt(2 * pi)) * h3(\x, \x*\l) : (
      sqrt(2/pi) * \e)))));
  };
}

\def\cdf(#1)(#2)(#3){0.5*(1+(erf((#1-#2)/(#3*sqrt(2)))))}%
% to be used: \cdf(x)(mean)(variance)

\DeclareMathOperator{\CDF}{cdf}
\tikzset{
    declare function={
        normcdf(\x,\m,\s)=1/(1 + exp(-0.07056*((\x-\m)/\s)^3 - 1.5976*(\x-\m)/\s));
    }
}

\begin{document}

%\begin{flushright}
%  \noindent \begin{tabular}{p{2.8in} r l}
\title{\vspace{-5ex}{\sc \examnum}}
\author{\bf \class}
\date{October 14, 2025\\[4mm]
   {\sc time limit:} {\bf \sc Two Hours}
}
% \end{tabular}\\
%\end{flushright}
%
\clearpage
\maketitle
% \noindent\rule[1ex]{\textwidth}{.1pt}
% \subsection*{Name}
% Please print your name clearly in the box below.\\
% \begin{center}
%   \framebox(300,40){\Huge\phantom{t}} \\
%   \vspace{2ex}
% \end{center}

% \vspace{40ex}
% {\it Turn to the next page to read the instructions.}
% \thispagestyle{empty}


\subsection*{Instructions}
This exam contains\textbf{ 10 pages} (including the front and back pages) and \textbf{9 problems, 64 points}. 
You have {\bf 2 hours} to complete it.  Either print out the PDF and complete by hand and upload, or complete it digitally 
and upload as a PDF on Canvas.
% Check to see if any pages are missing.
% Make sure you have written your name on the front page.
% If for any reason you have any loose pages, put your initials on the
% top of these pages.\\
\noindent This is an \textbf{open resource examination}. You are expected to complete the exam individually. Asking
anyone (colleague, friend, tutor, etc) questions about the exam is \textit{\rd \bf not allowed}. If any questions arise during the exam, direct them to me (via email).
\\
 

\noindent The following rules apply: 
% \begin{minipage}[t]{3.7in}
% \vspace{0pt}
\begin{itemize}

%\item \textbf{If you use a ``fundamental theorem'' you must indicate this} and explain
%why the theorem may be applied.

\item \textbf{Organize your work}, in a reasonably neat and coherent way.% , in

\item \textbf{Show ALL your work where appropriate}.
  The work you show will be evaluated as well as your final answer.
  Thus, provide ample justification for each step you take.
  Indicate when you have used Python to obtain a result and show the function or statement you used to arrive at your result.
  In the long response questions, simply putting down an answer without showing your steps  will not merit full credit.
  {\bf EXCEPTION:} For short response or ``True/False'' questions, \textit{no explanations are required}.
  However, the more work you show, the greater your chance of receiving partial credit.

\item As much as possible, put your answers in the alloted space in order to facilitate grading.

\item Questions are roughly in order of the lectures, so later questions may not necessarily be harder.
  If you are stuck on a problem, it may be better to skip it and get to it later.

\item Manage your time wisely.% Do not spend too much time on problems with fewer points.

\end{itemize}

%\noindent Do not write in the table to the right.
%\end{minipage}
%\hfill
%\begin{minipage}[t]{2.3in}
% \vspace{0pt}
% %\cellwidth{3em}
% \gradetablestretch{2}
% \vqword{Problem}
% \addpoints % required here by exam.cls, even though questions haven't started yet.
% \gradetable[v]%[pages]  % Use [pages] to have grading table by page instead of question
  % \vspace{0pt}
  % \renewcommand{\arraystretch}{2.8}
  % \quad  \begin{tabular}{| c | c| c|}\hline
  %   Problem  & Points & Score \\\hline
  %          1 & 10 & \quad\quad~~~\\\hline
  %          2 & 12 & \\\hline
  %          3 & 10 & \\\hline
  %          4 & 18 & \\\hline
  %         \bf Total & \bf 50 & \\\hline
  % \end{tabular}
%\end{minipage}
\newpage % End of cover page


% \lstset{language=C++,
%                 basicstyle=\tiny\ttfamily,
%                 keywordstyle=\color{blue}\ttfamily,
%                 stringstyle=\color{red}\ttfamily,
%                 commentstyle=\color{gray}\ttfamily,
%                 morecomment=[l][\color{gray}]{\#}
% }


% \thispagestyle{empty}

% \cellwidth{10em}
% \gradetablestretch{2}
% \vqword{Problem}
% \addpoints % required here by exam.cls, even though questions haven't started yet.
% \gradetable[v]%[pages]  % Use [pages] to have grading table by page instead of question
% \vspace{0pt}

% \thispagestyle{empty}
% \noindent{\it Do not write anything on this page. Please turn over.}\\[5mm]

% \begin{table*}[h!]
%   \center
%   \LARGE
%   \renewcommand{\arraystretch}{1.5}
%   \quad
%   \begin{tabular}{R{1.5in} R{1.5in} R{1.5in}}\toprule 
%     \bf \it Problem  & \bf Score & \bf Points \\\midrule
%           \it 1 & & \bf 6 \\\midrule
%           \it 2 & & \bf 10  \\\midrule
%           \it 3 & & \bf 5  \\\midrule
%            \it 4 & & \bf 5  \\\midrule
%            \it 5 & & \bf 7\\\midrule
%            \it 6 & & \bf 7  \\\midrule
%            \it 7 & & \bf 10 \\\midrule
%     \bf \it TOTAL & & \bf 50  \\\bottomrule
%   \end{tabular}

%   \end{table*}
% \newpage

\section*{Problem 1 \quad {\it True/False questions (10 points)}}
Respond ``T'' ({\it True})  or  ``F'' (\textit{False}) to the following statements.
Use the boxes provided. Each response is worth 1 point.
Note that a statement can only be regarded as true in this framework if it always holds in all circumstances.
If a statement does not hold under a given condition not already explicitly excluded, then it should be regarded as false.

\begin{enumerate}[\bf (i)]


\item \hfill
  \begin{minipage}{.1\linewidth}
    \framebox(40,40){ \gr }% F
  \end{minipage}\quad
  \begin{minipage}{.85\linewidth}
    For a given online account, your password must have 10 alphanumeric characters (no caps). Assuming you are the first user to create an account, the number of possibilities for your password is $26^{10}$.
  \end{minipage}
  \smallskip
\item \hfill
  \begin{minipage}{.1\linewidth}
    \framebox(40,40){ \gr } %T
  \end{minipage}\quad
  \begin{minipage}{.85\linewidth}
    The area under the curve of a PDF is given by the integral of the PDF.
  \end{minipage}  
  \smallskip
\item \hfill
  \begin{minipage}{.1\linewidth}
    \framebox(40,40){\gr  } %T
  \end{minipage}\quad
  \begin{minipage}{.85\linewidth}
    In a right-skewed distribution, the mean is less than the median.
  \end{minipage}
  \smallskip
\item \hfill
  \begin{minipage}{.1\linewidth}
    \framebox(40,40){\gr } %F
  \end{minipage}\quad
  \begin{minipage}{.85\linewidth}
    $P(A\cup B) = P(A) - P(B)$ for two events $A$ and $B$ that are mutually exclusive.
  \end{minipage}
  \smallskip  
\item \hfill
  \begin{minipage}{.1\linewidth}
    \framebox(40,40){\gr } %F
  \end{minipage}\quad
  \begin{minipage}{.85\linewidth}
    If $A$ and $B$ are statistically independent events, then $P(A|B) = P(A)$.
  \end{minipage}
  \smallskip
  \item \hfill
    \begin{minipage}{.1\linewidth}
      \framebox(40,40){\gr  } %F
    \end{minipage}\quad
    \begin{minipage}{.85\linewidth}
      The minimum value of any cumulative distribution function is $0$.
     \end{minipage}  
  \smallskip  
\item \hfill
  \begin{minipage}{.1\linewidth}
    \framebox(40,40){\gr } %F
  \end{minipage}\quad
  \begin{minipage}{.85\linewidth}
    Under certain conditions, a binomial distribution with parameters $(n,p)$ can be approximated by a normal
    distribution with $\sigma = \sqrt{n(1-p)p}$.
  \end{minipage}  
  \smallskip

  \item \hfill
  \begin{minipage}{.1\linewidth}
    \framebox(40,40){\gr  } %T
  \end{minipage}\quad
  \begin{minipage}{.85\linewidth}
    Given a normal distribution with parameters $\mu$ and $\sigma$, the variance of the distribution is $\mu^{2}$.
  \end{minipage}
  \smallskip   
  \item \hfill
    \begin{minipage}{.1\linewidth}
      \framebox(40,40){\gr } %F
    \end{minipage}\quad
    \begin{minipage}{.85\linewidth}
      The standard normal variate $Z$ has a standard deviation of 1.
    \end{minipage}
  \smallskip

    % \smallskip    
    \item \hfill
    \begin{minipage}{.1\linewidth}
      \framebox(40,40){\gr } %T
    \end{minipage}\quad
    \begin{minipage}{.85\linewidth}
      If a variable $X$ is lognormally distributed with parameters $\mu$ and $\sigma$, then the median of $X$ is given by $\exp(\mu)$.
    \end{minipage}

    
  \end{enumerate}
  \eject

\section*{Problem 2 \quad {\it Venn diagrams (6 points)}}
Write the combination of events (using set notation)  depicted in each of the figures below.
  \begin{enumerate}[\bf (a)]
\begin{minipage}{.47\linewidth}
\item  ~
  \begin{center}
      \begin{tikzpicture}[baseline={($(current bounding box.north)-(0,1.6ex)$)},scale=.4]
    \draw[thick]  (-4,-3) rectangle (7,3) node[above left] {$\bm S$};
    \draw[thick] (0,0) circle (2 cm) node[left] {$\bm{A}$};
    \draw[thick] (3,0) circle (2 cm) node[right] {$\bm{B}$};

    \begin{scope}
      %\clip (0,0) circle (2 cm);
      \fill[ pattern=north west lines, opacity=.75](3,0) circle (2 cm);
     % \fill[ pattern=north west lines](0,0) circle (2 cm);      
    \end{scope}
  \end{tikzpicture} 
\end{center}
  \vspace{2ex}
  \framebox(200,30){\Huge\phantom{t}}    

\item ~
  \begin{center}
      \begin{tikzpicture}[baseline={($(current bounding box.north)-(0,1.6ex)$)},scale=.4]
   \draw[thick]  (-4,-3) rectangle (7,3) node[above left] {$\bm S$};
        \fill[ pattern=north west lines, opacity=.75](-4,-3) rectangle (7,3) ;
    \filldraw[thick,fill=white] (0,0) circle (2 cm) node[left] {$\bm{A}$};
    \draw[thick] (3,0) circle (2 cm) node[right] {$\bm{B}$};
    \begin{scope}
    \end{scope}
  \end{tikzpicture}
\end{center}
  \vspace{2ex}
     \framebox(200,30){\Huge\phantom{t}}     

\item ~
  \begin{center}
      \begin{tikzpicture}[baseline={($(current bounding box.north)-(0,1.6ex)$)},scale=.5]
    % Draw the main rectangle
    \draw[thick, black] (0,0) rectangle (9,6) node[above left] {$\bm S$};

    % Partition the rectangle into 3 regions
    % Vertical partitions creating 3 equal regions
    \draw[thick] (3,0) -- (3,6);
    \draw[thick] (6,0) -- (6,6);

    % Shade region A
    \fill[pattern=north west lines, pattern color=black, opacity=0.6] (0,0) rectangle (3,6);

    % Label the regions
    \node  at (1.5,.5) {$\bm{A}$};
    \node  at (4.5,.5) {$\bm{B}$};
    \node at (7.5,.5) {$\bm{C}$};

    % Draw the ellipse E (centered, spanning multiple regions)
    \draw[thick, blue, line width=1.5pt] (4.5,3) ellipse (3 and 2);

    % Label the ellipse
    \node[blue] at (2,5) {$\bm E$};

    % Redraw boundaries for clarity
    \draw[thick] (3,0) -- (3,6);
    \draw[thick] (6,0) -- (6,6);
    \draw[thick, black] (0,0) rectangle (9,6);
    \draw[thick, blue, line width=1.5pt] (4.5,3) ellipse (3 and 2);

    \end{tikzpicture}
\end{center}
  \vspace{2ex}
     \framebox(200,30){\Huge\phantom{t}}    
      
 \end{minipage}\hfill
\begin{minipage}
  {.47\linewidth}
    \item ~ 
  \begin{center}
      \begin{tikzpicture}[baseline={($(current bounding box.north)-(0,1.6ex)$)},scale=.5]
        % Draw the main rectangle
        \draw[thick, black] (0,0) rectangle (9,6) node[above left] {$\bm S$};

        % Partition the rectangle into 3 regions
        % Vertical partitions creating 3 equal regions
        \draw[thick] (3,0) -- (3,6);
        \draw[thick] (6,0) -- (6,6);

        % Label the regions
        \node  at (1.5,.5) {$\bm{A}$};
        \node  at (4.5,.5) {$\bm{B}$};
        \node at (7.5,.5) {$\bm{C}$};

        % Draw the ellipse E (centered, spanning multiple regions)
        \draw[thick, blue, line width=1.5pt] (4.5,3) ellipse (3 and 2);

        % Label the ellipse
        \node[blue] at (2,5) {$\bm E$};

        % Shade the entire rectangle
        \fill[pattern=north east lines, pattern color=black, opacity=0.6] (0,0) rectangle (9,6);

        % Remove shading from the ellipse E (make it white)
        \fill[white] (4.5,3) ellipse (3 and 2);

        % Redraw boundaries for clarity
        \draw[thick] (3,0) -- (3,6);
        \draw[thick] (6,0) -- (6,6);
        \draw[thick, black] (0,0) rectangle (9,6);
        \draw[thick, blue, line width=1.5pt] (4.5,3) ellipse (3 and 2);

        \end{tikzpicture}
 \end{center}
  \vspace{1ex}

     \framebox(200,30){\Huge\phantom{t}}

  
\item ~
  \begin{center}
      \begin{tikzpicture}[baseline={($(current bounding box.north)-(0,1.6ex)$)},scale=.5]
% Draw the main rectangle
\draw[thick, black] (0,0) rectangle (9,6) node[above left] {$\bm S$};

% Partition the rectangle into 3 regions
% Vertical partitions creating 3 equal regions
\draw[thick] (3,0) -- (3,6);
\draw[thick] (6,0) -- (6,6);

% Label the regions
\node  at (1.5,.5) {$\bm{A}$};
\node  at (4.5,.5) {$\bm{B}$};
\node at (7.5,.5) {$\bm{C}$};

% Draw the ellipse E (centered, spanning multiple regions)
\draw[thick, blue, line width=1.5pt] (4.5,3) ellipse (3 and 2);

% Label the ellipse
\node[blue] at (2,5) {$\bm E$};

% Shade the intersection of ellipse E with region C
\begin{scope}
\clip (6,0) rectangle (9,6);
\fill[pattern=north east lines, pattern color=black, opacity=0.6] (4.5,3) ellipse (3 and 2);
\end{scope}

% Alternative: solid fill instead of pattern
% \begin{scope}
% \clip (6,0) rectangle (9,6);
% \fill[red!40, opacity=0.8] (4.5,3) ellipse (3 and 2);
% \end{scope}

% Redraw boundaries for clarity
\draw[thick] (3,0) -- (3,6);
\draw[thick] (6,0) -- (6,6);
\draw[thick, black] (0,0) rectangle (9,6);
\draw[thick, blue, line width=1.5pt] (4.5,3) ellipse (3 and 2);

\end{tikzpicture}
\end{center}
  \vspace{1ex}
     \framebox(200,30){\Huge\phantom{t}}    
        
\item ~
  \begin{center}
      \begin{tikzpicture}[baseline={($(current bounding box.north)-(0,1.6ex)$)},scale=.5]
    % Draw the main rectangle
    \draw[thick, black] (0,0) rectangle (9,6) node[above left] {$\bm S$};

    % Partition the rectangle into 3 regions
    % Vertical partitions creating 3 equal regions
    \draw[thick] (3,0) -- (3,6);
    \draw[thick] (6,0) -- (6,6);

    % Label the regions
    \node  at (1.5,.5) {$\bm{A}$};
    \node  at (4.5,.5) {$\bm{B}$};
    \node at (7.5,.5) {$\bm{C}$};

    % Draw the ellipse E (centered, spanning multiple regions)
    \draw[thick, blue, line width=1.5pt] (4.5,3) ellipse (3 and 2);

    % Label the ellipse
    \node[blue] at (2,5) {$\bm E$};

    % Shade region B outside the intersection with E
    \begin{scope}
    \clip (3,0) rectangle (6,6);
    \fill[pattern=north east lines, pattern color=black, opacity=0.6] (0,0) rectangle (9,6);
    \end{scope}

    % Remove shading from the intersection of B and E
    \begin{scope}
    \clip (3,0) rectangle (6,6);
    \fill[white] (4.5,3) ellipse (3 and 2);
    \end{scope}

    % Redraw boundaries for clarity
    \draw[thick] (3,0) -- (3,6);
    \draw[thick] (6,0) -- (6,6);
    \draw[thick, black] (0,0) rectangle (9,6);
    \draw[thick, blue, line width=1.5pt] (4.5,3) ellipse (3 and 2);

    \end{tikzpicture}
 \end{center}
   
  \vspace{1ex}
     \framebox(200,30){\Huge\phantom{t}}     
 
 

\end{minipage}
\end{enumerate} 

\eject
\section*{Problem 3 \quad {\it Discrete distribution (4 points)}}

 The PMF of a random variable $X$ is given in the figure below. 
  Use the figure to answer the following questions.
    
  \begin{figure}[h!]
    \centering
    \begin{tikzpicture}
      %\pgfplotsset{minor grid style = {line width = 0.1pt}}
      %\pgfplotsset{major grid style = {line width = 0.4pt}}
      \begin{axis}[
        axis x line=center,
        axis y line=center,
        xtick={0,2,...,19},
        ytick={0.02,0.04,...,0.18},
        domain = 0:18,
        samples = 19,
        xlabel={$x$},
        ylabel={$P(X=x)$},
        xlabel style={right},
        ylabel style={above},
        ymax=0.19,
        xmax=20,
        x post scale=1.4,
        grid=both,
        grid style={line width=.1pt, draw=gray!50!black},
        yticklabel style={
            /pgf/number format/fixed,
            /pgf/number format/fixed zerofill,
            /pgf/number format/precision=2
          }
       % minor tick num =2
        ]
        \addplot+[ycomb,black,ultra thick,opacity=.5,mark options={scale=1.5,opacity=.8}] {poiss(6))}; %\addlegendentry{$\lambda = 1$}}
      \end{axis}
    \end{tikzpicture}
  \end{figure}

  \begin{enumerate}[\bf (a)]
    \item Estimate the probability $P(X = 6)$. \pts{1} % Answer: 0.24

    \begin{minipage}[]{.8\linewidth}
      \framebox(300,40){\Huge\phantom{t}}     
    \end{minipage}
  
    \bigskip
    \bigskip

  \item Estimate the probability $P(4 < X \le 7)$. \pts{2}  
  
  % \begin{minipage}[]{.1\linewidth}
  %   {\bf Answer:}
  % \end{minipage}\qquad
  \begin{minipage}[]{.8\linewidth}
    \framebox(300,60){\Huge\phantom{t}}     
  \end{minipage}

  \bigskip
  \bigskip

  \item If the distribution shown has a single parameter $\lambda$, what distribution is it? \pts{1}
  
      \begin{minipage}[]{.8\linewidth}
      \framebox(300,40){\Huge\phantom{t}}     
    \end{minipage}
  
  \item Which of the following would be your best guess for $\lambda$? (Circle or underline the correct answer.)\pts{1}
    \begin{center}
    ~ \hfill {\bf i.} 4 \hfill {\bf ii.} 6 \hfill {\bf iii.} 8 \hfill {\bf iv.} 10 \hfill~
    \end{center}
  \end{enumerate}
   \eject
% \item Shade the area under the curve that represents the probability $P( X > 8)$. \marginpar{\it[1]}

%   \medskip

%   \begin{figure}[h!]
%     \centering
%       \begin{tikzpicture}
%     \begin{axis}[no markers, domain=0:10, samples=100,
%       axis lines*=left, xlabel=$x$, ylabel=$f_X(x)$,,
%       height=6cm, width=10cm,
%       %xtick={-3, -2, -1, 0, 1, 2, 3},
%       %ytick=\empty,
%       enlargelimits=false, clip=false, axis on top,
%       grid style={line width=.1pt, draw=gray},
%       yticklabel style={
%             /pgf/number format/fixed,
%             /pgf/number format/fixed zerofill,
%             /pgf/number format/precision=2
%           },        
%       grid = major]
%       \addplot [ultra thick, domain=-2:12] {gauss(5,2)};

%     \end{axis}
%   \end{tikzpicture}

% \end{figure}
  

% \item Shade the area under the curve that represents the probability $P( 4 \le X < 6)$. \marginpar{\it[1]}

%   \medskip

%   \begin{figure}[h!]
%     \centering
%       \begin{tikzpicture}
%     \begin{axis}[no markers, domain=0:10, samples=100,
%       axis lines*=left, xlabel=$x$, ylabel=$f_X(x)$,,
%       height=6cm, width=10cm,
%       %xtick={-3, -2, -1, 0, 1, 2, 3},
%       %ytick=\empty,
%       enlargelimits=false, clip=false, axis on top,
%       grid style={line width=.1pt, draw=gray},
%       yticklabel style={
%             /pgf/number format/fixed,
%             /pgf/number format/fixed zerofill,
%             /pgf/number format/precision=2
%           },        
%       grid = major]
%       \addplot [ultra thick, domain=-2:12] {gauss(5,2)};

%     \end{axis}
%   \end{tikzpicture}

% \end{figure}



 

 \section*{Problem 4 \quad {\it CDF (5 points)}}
   Below is the CDF of a given  distribution. Use the graph to answer the following.

  \begin{figure}[h!]
    \centering
      \begin{tikzpicture}[scale=.9]
    \begin{axis}[no markers, domain=1:11, samples=100,
      axis lines*=left, xlabel=$x$, ylabel=$F_X(x)$,,
      height=8cm, width=16cm,ymax=1,
      xtick={1, 1.5, ..., 11},
      ytick={0,0.125,.25,...,1},
      enlargelimits=false, clip=false, %axis on top,
      grid style={line width=.1pt, draw=gray!40},
      major grid style={line width=.7pt, draw=gray},
      yticklabel style={
            /pgf/number format/fixed,
            /pgf/number format/fixed zerofill,
            /pgf/number format/precision=3
          },        
      grid = both,minor tick num=4]
      % addplot [ultra thick, domain=-4:12] {gauss(4,3)};
      \addplot+[blue, ultra thick, smooth] {normcdf(x, 6,2)};
     % \addplot [dashed, ultra thick, green!50!black] coordinates {(-1, 0.25) (.65,.25)};
     % \addplot [dashed, ultra thick, green!50!black] coordinates {(.65, 0.25) (.65,0)};
  
%      \addplot [thick,draw=none, pattern=north west lines,  domain=6:12] {gauss(4,3)} \closedcycle;
%      \addplot [dashed, ultra thick,] coordinates {(6,0) (6,{gauss(5,2)})};
    \end{axis}
  \end{tikzpicture}

\end{figure}
\begin{enumerate}[\bf (a)]

\item Estimate the first quartile (to 1 decimal place). \pts{1}

\begin{minipage}[]{.1\linewidth}
    {\bf Answer:}
  \end{minipage}\qquad
  \begin{minipage}[]{.8\linewidth}
    \framebox(300,40){\phantom{\Huge t} }     
  \end{minipage}
  \bigskip
  
\item What is the second quartile? \pts{1}
  
\begin{minipage}[]{.1\linewidth}
    {\bf Answer:}
  \end{minipage}\qquad
  \begin{minipage}[]{.8\linewidth}
    \framebox(300,40){\phantom{\Huge t} }     
  \end{minipage}
  \bigskip
  
\item Estimate the third quartile (to 1 decimal place). \pts{1}
  
 \begin{minipage}[]{.1\linewidth}
    {\bf Answer:}
  \end{minipage}\qquad
  \begin{minipage}[]{.8\linewidth}
    \framebox(300,40){\Huge\phantom{t}}     
  \end{minipage}

  \bigskip
\item Compute the interquartile range (IQR). \pts{1}
  
 \begin{minipage}[]{.1\linewidth}
    {\bf Answer:}
  \end{minipage}\qquad
  \begin{minipage}[]{.8\linewidth}
    \framebox(300,40){\Huge\phantom{t}}     
  \end{minipage}

  \bigskip
\item If you are told that the above CDF is that of a normal distribution, what is its mean? \pts{1}
  
 \begin{minipage}[]{.1\linewidth}
    {\bf Answer:}
  \end{minipage}\qquad
  \begin{minipage}[]{.8\linewidth}
    \framebox(300,40){\Huge\phantom{t}}     
  \end{minipage}
\end{enumerate}
  \eject
% \item The median of a lognormal variate is $6$. What is the mean of the associated normal variate? \pts{2}

%   \vspace{2ex}
%  \begin{minipage}[]{.1\linewidth}
%     {\bf Answer:}
%   \end{minipage}\qquad
%   \begin{minipage}[]{.8\linewidth}
%     \framebox(300,80){\Huge\phantom{t}}     
%   \end{minipage}

  

%\item%  The number of ... that ... is binomially distributed. The probability of ... is  ...
  % Use your table to find the probability that at most 3 out of 7 cars will be defective...

% \item The figure below shows the CDF of a discrete random variable $X$. \marginpar{\it[2pts]}
%   To aid you in properly interpreting this plot, you are given that $P(X\le 2) = 0.2$.
%   What is the probability $P(X >3)$? %answer: 0.6
  
%   \begin{figure}[h!]
%     \centering
%     \begin{tikzpicture}
%       \begin{axis}[
%         xlabel=$x$, ylabel=$F_X(x)$,
%         clip=false,
%         jump mark left,
%         ymin=0,ymax=1,
%         xmin=0, xmax=5,
%         every axis plot/.style={very thick},
%         discontinuous,
%         grid = both,
%         grid style={line width=.2pt, draw=gray},
%         table/create on use/cumulative distribution/.style={
%           create col/expr={\pgfmathaccuma + \thisrow{f(x)}}   
%         }
%         ]
%         \addplot [black] table [y=cumulative distribution]{
%           x f(x)
%           0 1/15
%           1 2/15
%           2 1/5
%           3 4/15
%           4 1/3
%           5 0
%         };
% \end{axis}
% \end{tikzpicture}
% \end{figure}

%  \begin{minipage}[]{.1\linewidth}
%     {\bf Answer:}
%   \end{minipage}\qquad
%   \begin{minipage}[]{.8\linewidth}
%     \framebox(300,40){\Huge\phantom{t}}     
%   \end{minipage}
  

\eject



\section*{Problem 5 \quad {\it Probabilities I (6 points)}}
The  table below summarizes the results of a survey we did in class. 
Use it to answer the following questions.
\begin{table}[h!]\small
\centering
\begin{tabular}{|l|l|c|c|c|}
\hline
&  & \multicolumn{2}{c|}{\textbf{Do you drink coffee?}} &  \\\cline{3-5}
&  & \textbf{Yes} & \textbf{No} & \textbf{Grand Total} \\
\hline
\multirow{2}{*}{\textbf{Where do you typically study?}} & \textbf{Library} & 14 & 22 & 36 \\
\cline{2-5}
& \textbf{Elsewhere} & 52 & 67 & 119 \\
\hline
\multicolumn{2}{|c|}{\textbf{Grand Total}} &  \textbf{66} & \textbf{89} & \textbf{155} \\
\hline
\end{tabular}
\end{table}

\begin{enumerate}[\bf (a)]
  \item What is the probability that a randomly selected student studies in the library? \pts{2} \vspace{20ex}
  \item What is the probability that a randomly selected student does not drink coffee? \pts{2} \vspace{20ex}
  \item What is the \pts{2} probability that a randomly selected student studies in the library and does not drink coffee?
  \vspace{20ex}

\end{enumerate}
\eject

\section*{Problem 6 \quad {\it Probabilities II (8 points)}}
The  table below summarizes the results of a survey we did in class. 
Answer the following questions (drawing on results from the previous problem).
\begin{table}[h!]\small
\centering
\begin{tabular}{|l|l|c|c|c|}
\hline
&  & \multicolumn{2}{c|}{\textbf{Do you drink coffee?}} &  \\\cline{3-5}
&  & \textbf{Yes} & \textbf{No} & \textbf{Grand Total} \\
\hline
\multirow{2}{*}{\textbf{Where do you typically study?}} & \textbf{Library} & 14 & 22 & 36 \\
\cline{2-5}
& \textbf{Elsewhere} & 52 & 67 & 119 \\
\hline
\multicolumn{2}{|c|}{\textbf{Grand Total}} &  \textbf{66} & \textbf{89} & \textbf{155} \\
\hline
\end{tabular}
\end{table}

\begin{enumerate}[\bf (a)]


\item What \pts{2} is the probability that a randomly selected student studies in the library, given that they do not drink coffee?
\vspace{25ex}

\item Are the events that a student studies in the library and does not drink coffee independent? Justify your answer. \pts{3}
\vspace{25ex}

     \item What is the \pts{3} probability that a randomly selected student either studies in the library or does not drink coffee?
\vspace{20ex}

\end{enumerate}

% \section*{Problem 4 \quad {\it Counting (7 points)}}



% \begin{enumerate}[\bf (a)]
% \item License plates in a certain state consist of \pts{3} 1 digit, followed by 3 letters and then 2 digits. How many different
% license plates can be manufactured?
% \vspace{40ex}

% \item In how many ways can you arrange 6 items? \pts{2}
% \vspace{20ex}

% \item How many distinct groups of 3 can you obtain from a larger group of 5 items? \pts{2}
% \vspace{40ex}
  
% \end{enumerate}



% \section*{Problem 6 \quad {\it Binomial distribution (9 points; 5 points EC)}}
% %%% Page 187 deVore #72

% 80\% of all vehicles inspected  at a certain facility station pass the inspection.
% Assuming that successive vehicles pass or fail independently of one another, find the following:
% \begin{enumerate}[\bf (a)]
% \item The standard deviation \pts{2} of the binomial distribution governing the probability of the next 10 vehicles passing inspection.
%   \vspace{20ex}
  
% \item The probability that half of the next 10 vehicles inspected pass. \pts{3}
%   \vspace{30ex}
  
% \item The probability that at least 4 of the next 10 vehicles inspected fail. \pts{4} %% 1 - none failing
%   \vspace{30ex}

% \end{enumerate}

% \eject

% \section*{Problem 7 \quad {\it Poisson distribution (8 points)}}

% Hourly arrivals at a small coffee shop can be model using a Poisson distribution with a variance of 4.

% \begin{enumerate}[\bf (a)]
% \item What is the mean of the Poisson distribution?\pts{2}
%   \vspace{20ex}

% \item What is the probability that \pts{2} 5 customers will arrive in a given hour?
%   \vspace{30ex}

% \item Find the \pts{4} probability that no more than 8 customers will arrive over a duration of 2 consecutive hours.
  
% \end{enumerate}

 
% \eject

% \begin{quote}
%   \gr
%   \begin{align*}
%     P(X \ge 3) &= 1 - P(X\le 2) \\
%                &= 1 - \sum_{k=0}^2 {8 \choose k} (0.6)^2(0.4)^6 \\
%                &= 1 - 0.0498 \quad \text{(from table)} \\
%                &= \boxed{0.9502}
%   \end{align*}
% \end{quote}

\section*{Problem 7  \quad {\it Normal distribution (9 points)}}
The drainage from a community during a storm is a normal random variable
estimated to have a mean of 1.2 million gallons per day (mgd) and an SD of 0.4
mgd. If the storm drain system is designed with a maximum drainage capacity of
1.5 mgd, answer the following questions (and show the code you use where appropriate).

\begin{enumerate}[\bf (a)]
\item Sketch the normal distribution, showing both the mean and the maximum drainage capacity. \pts{3}
  \vspace{20ex}

\item  What is the underlying probability of flooding during a storm that is assumed \pts{2}
in the design of the drainage system? (Note: flooding occurs when the maximum drainage capacity is exceeded.)
\vspace{20ex}

\item What is the probability that the drainage is less than 1 mgd? \pts{2}
  \vspace{25ex}


  \item  Find the 95th-percentile drainage load from the community during a storm. \pts{2}
  \vspace{30ex}


\end{enumerate}


%\vspace{40ex}


\eject
\section*{Problem 8 \quad {\it Logormal distribution (8 points)}}
The distribution of the number of words (sentence length) from random sentences collected in class 
fitted a lognormal distribution with parameters $\mu = 2.59, \sigma = 0.65$. 
Now, use this distribution to answer the following questions.

\begin{enumerate}[(a)]
  \item  What is the median sentence length? \pts{2}
  \vspace{15ex}


\item What is the probability that the sentence length  is at least 10 words? \pts{3}
  \vspace{30ex}

\item Sketch the distribution, indicating the median and shading the portion of 
the PDF corresponding to the probability in part (b). \pts{3}
  
\end{enumerate}

\eject

\section*{Problem 9 \quad {\it Short answer (8 points)}}
In each case, show your work/code.

\begin{enumerate}[\bf (a)]
\item A random variable $X$ is binomially distributed with $n=50$ and $p=0.3$. Find the probability that $X < 10$. \pts{3}
\vspace{30ex}

\item A random variable $T$ is exponentially distributed with a rate parameter of $\lambda = 20$ occurrences/minute.\pts{2}
What is the mean of $T$? 
  \vspace{20ex}

  
\item In how many ways can you have 3 successes out of 10 trials in which the outcome of each trial is either success or failure? \pts{3}
\end{enumerate}

% \eject

% \section*{Problem 11 \quad {\it Joint distributions (11 points; 3 points EC)}}
% The joint PMF of two random variables $X$ and $Y$ is shown in the table below. \\

% \begin{center}
% \begin{tabular}{l l | l l l}
%   & & \multicolumn{3}{c}{$y$} \\
%   \multicolumn{2}{c|}{$p_{X,Y}(x,y)$}
%   & \bf 0 & \bf 1 &\bf  2 \\ \hline
%   \multirow{3}{*}{~~~~$x$~~~~}
%   &\bf 0 & 0.10 & 0.06 & 0.02 \\ 
%   &\bf 1 & 0.12 & 0.04 & 0.01 \\
%   &\bf 2 & 0.20 & 0.30 & 0.15 \\
% \end{tabular}
% \end{center}
% \medskip

% \begin{enumerate}[\bf (i)]
  
% \item Find $P(X = 0 \cap Y = 1)$.\pts{2}
%   \vspace{20ex}
  
% \item Compute $P(X < 1)$. \pts{3}
%    \vspace{20ex}

%  \item Compute $P(Y \ge 1)$. \pts{3}
%    \vspace{20ex}

%    \eject
% \item Compute $P(Y = 0 | X = 0)$. \pts{3}
% \vspace{40ex}

% \item (Extra Credit) Find the marginal distribution $p_{X}(x)$. \pts{3}
  
% \end{enumerate}

% \eject

% \section*{Problem 7 \quad {\it Functions of multiple random variables (10 points)}}
% %%% Page 160, dVore #21
% The differential settlement between two piles in a foundation is given by:
% \begin{equation}
%   \label{eq:2}
%   D  = S_1  - S_2
% \end{equation}
% where $S_1$ and $S_2$ are normally distributed.

% You are given the following:
% \begin{align}
%   \mu_{S_1} = \mu_{S_2} &= 3.0 \\
%   \sigma_{S_1} = \sigma_{S_2} &= 0.7 \\
%   \rho &= 0.8 
% \end{align}
% First, convince yourself that the mean of $D$ is given by
% \begin{equation}
%   \label{eq:3}
%   \mu_D = \mu_{S_1} - \mu_{S_2} = 3 - 3 = 0
% \end{equation}

% \begin{enumerate}[\bf (i)]
% \item Now, show that the variance of $D$ is equal to 0.196. \marginpar{\it [4pts]}
%   % \begin{quote}
%   %   \gr
%   %   \begin{align*}
%   %     Var(D) &= Var(S_1) + Var(S_2) - 2Cov(S_1,S_2) \\
%   %     Cov(S_1,S_2)
%   %            &= \rho\sigma_{S_1}\sigma_{S_2} \\
%   %            &= 0.8(0.7)(0.7) = 0.392 \\
%   %     \therefore Var(D) &= \sigma_{S_1}^2  + \sigma_{S_2}^2 - 2(0.392) \\
%   %            &= 0.7^2 + 0.7^2 - 0.784 \\
%   %            &= 0.98 - 0.784 \\
%   %            &= 0.196
%   %   \end{align*}
%   % \end{quote}
% \eject
% \item Find the probability that the magnitude of the differential settlement $|D|$\marginpar{\it [6pts]}
%    is no more than 0.5 inch. 
%   % \begin{quote}
%   %   \gr
%   %   \begin{align*}
%   %     P(|D| \le 0.5)  &= P(-0.5 < D \le 0.5) \\
%   %     &= \nmfr{0.5}{0}{\sqrt{0.196}} - \nmfr{-0.5}{0}{\sqrt{0.196}} \\
%   %     &= \Phi(1.129) - \Phi(-1.129)\\
%   %     &= \Phi(1.129) - (1 - \Phi(1.129)) \\
%   %     &= 0.871 - 1 + 0.871
%   %     &= \boxed{0.741}
%   %   \end{align*}
%   % \end{quote}
% \end{enumerate}


% \newpage

% \vfill
% \begin{center}
%   BLANK PAGE
% \end{center}
% \newpage

% \vfill
% \begin{center}
%   BLANK PAGE
% \end{center}

% \newpage
% \vfill
% \begin{center}
%   BLANK PAGE
% \end{center}

\newpage
~
\thispagestyle{empty}
\vfill
\begin{center}
  \includegraphics[width=1in]{umass-seal}
  
  {\sc
    CEE 260/MIE 273 $|$ Jimi Oke $|$ Fall 2025\\
    Department of Civil and Environmental Engineering \\
    University of Massachusetts Amherst
  }
\end{center}
\end{document}

%%% Local Variables:
%%% mode: latex
%%% TeX-master: t
%%% End:
