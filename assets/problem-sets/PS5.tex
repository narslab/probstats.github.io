\documentclass[11pt,twoside]{article}
\usepackage{etex}

\raggedbottom

%geometry (sets margin) and other useful packages
\usepackage{geometry}
\geometry{top=1in, left=1in,right=1in,bottom=1in}
 \usepackage{graphicx,booktabs,calc}
 
\usepackage{listings}


% Marginpar width
%Marginpar width
\newcommand{\pts}[1]{\marginpar{ \small\hspace{0pt} \textit{[#1]} } } 
\setlength{\marginparwidth}{.5in}
%\reversemarginpar
%\setlength{\marginparsep}{.02in}

 
%\usepackage{cmbright}lstinputlisting
%\usepackage[T1]{pbsi}


\usepackage{chngcntr,mathtools}
\counterwithin{figure}{section}
\numberwithin{equation}{section}

%\usepackage{listings}

%AMS-TeX packages
\usepackage{amssymb,amsmath,amsthm} 
\usepackage{bm}
\usepackage[mathscr]{eucal}
\usepackage{colortbl}
\usepackage{color}


\usepackage{subfigure,hyperref,enumerate,polynom,polynomial}
\usepackage{multirow,minitoc,fancybox,array,multicol}

\definecolor{slblue}{rgb}{0,.3,.62}
\hypersetup{
    colorlinks,%
    citecolor=blue,%
    filecolor=blue,%
    linkcolor=blue,
    urlcolor=slblue
}

%%%TIKZ
\usepackage{tikz}

\usepackage{pgfplots}
\pgfplotsset{compat=newest}

\usetikzlibrary{arrows,shapes,positioning}
\usetikzlibrary{decorations.markings}
\usetikzlibrary{shadows}
\usetikzlibrary{patterns}
%\usetikzlibrary{circuits.ee.IEC}
\usetikzlibrary{decorations.text}
% For Sagnac Picture
\usetikzlibrary{%
    decorations.pathreplacing,%
    decorations.pathmorphing%
}

\tikzstyle arrowstyle=[black,scale=2]
\tikzstyle directed=[postaction={decorate,decoration={markings,
    mark=at position .65 with {\arrow[arrowstyle]{stealth}}}}]
\tikzstyle reverse directed=[postaction={decorate,decoration={markings,
    mark=at position .65 with {\arrowreversed[arrowstyle]{stealth};}}}]
\tikzstyle dir=[postaction={decorate,decoration={markings,
    mark=at position .98 with {\arrow[arrowstyle]{latex}}}}]
\tikzstyle rev dir=[postaction={decorate,decoration={markings,
    mark=at position .98 with {\arrowreversed[arrowstyle]{latex};}}}]

\usepackage{ctable}

%
%Redefining sections as problems
%
\makeatletter
\newenvironment{exercise}{\@startsection 
	{section}
	{1}
	{-.2em}
	{-3.5ex plus -1ex minus -.2ex}
    	{1.3ex plus .2ex}
    	{\pagebreak[3]%forces pagebreak when space is small; use \eject for better results
	\large\bf\noindent{Exercise 1.\hspace{-1.5ex} }
	}
	}
	%{\vspace{1ex}\begin{center} \rule{0.3\linewidth}{.3pt}\end{center}}
	%\begin{center}\large\bf \ldots\ldots\ldots\end{center}}
\makeatother

%
%Fancy-header package to modify header/page numbering 
%
\usepackage{fancyhdr}
\pagestyle{fancy}
%\addtolength{\headwidth}{\marginparsep} %these change header-rule width
%\addtolength{\headwidth}{\marginparwidth}
%\fancyheadoffset{30pt}
%\fancyfootoffset{30pt}
\fancyhead[LO,RE]{\small Oke}
\fancyhead[RO,LE]{\small Page \thepage} 
\fancyfoot[RO,LE]{\small PS 5  } 
\fancyfoot[LO,RE]{\small \scshape CEE 260/MIE 273} 
\cfoot{} 
\renewcommand{\headrulewidth}{0.1pt} 
\renewcommand{\footrulewidth}{0.1pt}
%\setlength\voffset{-0.25in}
%\setlength\textheight{648pt}


\usepackage{paralist}

\newcommand{\osn}{\oldstylenums}
\newcommand{\lt}{\left}
\newcommand{\rt}{\right}
\newcommand{\pt}{\phantom}
\newcommand{\tf}{\therefore}
\newcommand{\?}{\stackrel{?}{=}}
\newcommand{\fr}{\frac}
\newcommand{\dfr}{\dfrac}
\newcommand{\ul}{\underline}
\newcommand{\tn}{\tabularnewline}
\newcommand{\nl}{\newline}
\newcommand\relph[1]{\mathrel{\phantom{#1}}}
\newcommand{\cm}{\checkmark}
\newcommand{\ol}{\overline}
\newcommand{\rd}{\color{red}}
\newcommand{\bl}{\color{blue}}
\newcommand{\pl}{\color{purple}}
\newcommand{\og}{\color{orange!90!black}}
\newcommand{\gr}{\color{green!40!black}}
\newcommand{\nin}{\noindent}
\newcommand{\la}{\lambda}
\renewcommand{\th}{\theta}
\newcommand*\circled[1]{\tikz[baseline=(char.base)]{
            \node[shape=circle,draw,thick,inner sep=1pt] (char) {\small #1};}}

\newcommand{\bc}{\begin{compactenum}[\quad--]}
\newcommand{\ec}{\end{compactenum}}

\newcommand{\n}{\\[2mm]}
%% GREEK LETTERS
\newcommand{\al}{\alpha}
\newcommand{\gam}{\gamma}
\newcommand{\eps}{\epsilon}
\newcommand{\sig}{\sigma}

\newcommand{\p}{\partial}
\newcommand{\pd}[2]{\frac{\partial{#1}}{\partial{#2}}}
\newcommand{\dpd}[2]{\dfrac{\partial{#1}}{\partial{#2}}}
\newcommand{\pdd}[2]{\frac{\partial^2{#1}}{\partial{#2}^2}}
\newcommand{\mr}{\mathbb{R}}
\newcommand{\xs}{x^{*}}
\newenvironment{solution}
{\medskip\par\quad\quad\begin{minipage}[c]{.8\textwidth}\gr}{\medskip\end{minipage}}


\pgfmathdeclarefunction{poiss}{1}{%
  \pgfmathparse{(#1^x)*exp(-#1)/(x!)}%
  }

\pgfmathdeclarefunction{gauss}{2}{%
  \pgfmathparse{1/(#2*sqrt(2*pi))*exp(-((x-#1)^2)/(2*#2^2))}%
}

\pgfmathdeclarefunction{expo}{2}{%
  \pgfmathparse{#1*exp(-#1*#2)}%
}

\pgfmathdeclarefunction{binom}{2}{%
  \pgfmathparse{(#1!)/(x!*(#1-x)!) * (#2^x) * ((1-#2)^(#1-x))}%
}

\pgfmathdeclarefunction{lognormal}{2}{%
  \pgfmathparse{1/(x*#2*sqrt(2*pi))*exp(-((ln(x)-#1)^2)/(2*#2^2))}%
}

\usetikzlibrary{math}

% https://tex.stackexchange.com/questions/461758/asymmetric-distribution-gauss-curve
\tikzmath{%
  function h1(\x, \lx) { return (9*\lx + 3*((\lx)^2) + ((\lx)^3)/3 + 9); };
  function h2(\x, \lx) { return (3*\lx - ((\lx)^3)/3 + 4); };
  function h3(\x, \lx) { return (9*\lx - 3*((\lx)^2) + ((\lx)^3)/3 + 7); };
  function skewnorm(\x, \l) {
    \x = (\l < 0) ? -\x : \x;
    \l = abs(\l);
    \e = exp(-(\x^2)/2);
    return (\l == 0) ? 1 / sqrt(2 * pi) * \e: (
      (\x < -3/\l) ? 0 : (
      (\x < -1/\l) ? \e / (8 * sqrt(2 * pi)) * h1(\x, \x*\l) : (
      (\x <  1/\l) ? \e / (4 * sqrt(2 * pi)) * h2(\x, \x*\l) : (
      (\x <  3/\l) ? \e / (8 * sqrt(2 * pi)) * h3(\x, \x*\l) : (
      sqrt(2/pi) * \e)))));
  };
}

\def\cdf(#1)(#2)(#3){0.5*(1+(erf((#1-#2)/(#3*sqrt(2)))))}%
% to be used: \cdf(x)(mean)(variance)

\DeclareMathOperator{\CDF}{cdf}
\tikzset{
    declare function={
        normcdf(\x,\m,\s)=1/(1 + exp(-0.07056*((\x-\m)/\s)^3 - 1.5976*(\x-\m)/\s));
    }
}
%%%%%%%%%%%%%%%%%%%%%%%%%%%%%%%%%%%%%%%%%%%%%%%%%%%
%%%%%%%%%%%%%%%%%%%%%%%%%%%%%%%%%%%%%%%%%%%%%%%%%%%

\begin{document}

\lstset{language=C++,
                basicstyle=\tiny\ttfamily,
                keywordstyle=\color{blue}\ttfamily,
                stringstyle=\color{red}\ttfamily,
                commentstyle=\color{gray}\ttfamily,
                morecomment=[l][\color{gray}]{\#}
}


\thispagestyle{empty}


\nin{\LARGE Problem Set 5 {\gr  }}\hfill{\bf Prof. Oke}

\medskip\hrule\medskip

\nin {\small CEE 260/MIE 273: Probability \& Statistics in Civil Engineering
\hfill\textit{ 9.29.2025}}

\bigskip

\nin{\it \textbf{Due Tuesday, October 7, 2025 at 1:00 PM as PDF uploaded on Canvas.}
  Use this document as your template.
    \textbf{Show as much work as possible in order to get FULL credit.}
    There are 7 problems with a total of 41 points available.
    \textbf{Important:} If you use Python for any probability computations, briefly write/include the statements you
    used to arrive at your answers. If instead you use probability tables or a calculator, note this in the respective solution, as well.
}\\


\section*{Problem 1 \textit{(4 points)}}

Respond ``T'' ({\it True})  or  ``F'' (\textit{False}) to the following statements. Use the boxes provided. Each response is worth 1 point.

\begin{enumerate}[\bf (a)]


\item \hfill
  \begin{minipage}{.1\linewidth}
    \framebox(40,40){\gr }
  \end{minipage}\quad
  \begin{minipage}{.85\linewidth}
    If the natural logarithm of a random variable is normally distributed with parameters $\mu$ and $\sigma$, then the
    variable is lognormally distributed with the same parameters.
   \end{minipage}
  
  \smallskip

  \item \hfill
    \begin{minipage}{.1\linewidth}
      \framebox(40,40){\gr  }
    \end{minipage}\quad
    \begin{minipage}{.85\linewidth}
      The mean of a lognormal distribution is equal to or greater than its median.
    \end{minipage}

    \smallskip

  \item \hfill
  \begin{minipage}{.1\linewidth}
    \framebox(40,40){\gr  }
  \end{minipage}\quad
  \begin{minipage}{.85\linewidth}
    The mean of a exponentially-distributed random variable is equal to the inverse of its variance.
   \end{minipage}

  \smallskip
  
\item \hfill
  \begin{minipage}{.1\linewidth}
    \framebox(40,40){\gr  }
  \end{minipage}\quad
  \begin{minipage}{.85\linewidth}
    A binomial distribution with sufficiently large $n$ can be approximated by a normal distribution with $\mu = np$.
  \end{minipage}



\end{enumerate}



 \eject  

 \section*{Problem 2: Normal and Lognormal Distributions \textit{(6 points)}}
{\it Choose the option that best fills in the blank.}
   \begin{enumerate}[\bf(a)]

   \item The figure below depicts the PDF of a standard normal distribution. What is the value of $z^{*}$ in the figure? \pts{1}

 
        \begin{minipage}{.7\linewidth}
                  \begin{tikzpicture}
      \begin{axis}[no markers, domain=0:10, samples=100,
      axis x line=center,
      axis y line=none,
      xlabel=$z$, ylabel=$f_X(x)$,,
      height=3cm, width=10cm,
      xtick={-1.96,0,1.96},
      xticklabels={$-z^{*}$,$0$,$z^{*}$},
      ymax=.15,
      ytick=\empty,
      x label style={anchor=west},
      y label style={anchor=south},
      enlargelimits=true, clip=false, axis on top]

      \addplot [blue, domain=-5:5] {gauss(0,1)};
      \addplot [gray, fill=gray!50, domain=-1.96:1.96] {gauss(0,1)} \closedcycle;
      \addplot [orange,fill=orange,  domain=1.96:5] {gauss(0,1)} \closedcycle;
      \addplot [orange,fill=orange,  domain=-5:-1.96] {gauss(0,1)} \closedcycle;
      \node (d) at (axis cs: 0,.15) {Area: $0.95$};
      %\draw[thick, ->] (d) -- (axis cs: -.6,.05);
      \node (c) at (axis cs: 3.15,.1) {\og Area: $0.025$};
      \draw[thick,->] (c) -- (axis cs: 2.2, 0.005);
      \node (e) at (axis cs: -3.15,.1) {\og Area: $0.025$};
      \draw[thick,->] (e) -- (axis cs: -2.2, 0.005);
      %\draw[thick, |->] (axis cs: 6,-0.025) -- (axis cs: 10,-0.025) node[below,pos=.5] {\small\og Reject $H_0$};
      %\draw[thick, |->] (axis cs: -6,-0.025) -- (axis cs: -10,-0.025) node[below,pos=.5] {\small\og Reject $H_0$};
    \end{axis}
  \end{tikzpicture}

        \end{minipage}
        \begin{minipage}{.25\linewidth}
               \begin{enumerate}[\bf(i)]
     \item 0
     \item 1.65
     \item 1.96
     \item 2.58
     \end{enumerate}


        \end{minipage} 
      \item The figure below depicts the PDF of a standard normal distribution. What is the value of $z_{\alpha/2}$ in the figure? \pts{1}
        
        \begin{minipage}{.7\linewidth}
                  \begin{tikzpicture}
      \begin{axis}[no markers, domain=0:10, samples=100,
      axis x line=center,
      axis y line=none,
      xlabel=$z$, ylabel=$f_X(x)$,,
      height=3cm, width=10cm,
      xtick={-1.44,0,1.44},
      xticklabels={$z_{\alpha/2}$,$0$,$z_{(1-\alpha/2)}$},
      ymax=.15,
      ytick=\empty,
      x label style={anchor=west},
      y label style={anchor=south},
      enlargelimits=true, clip=false, axis on top
      %grid style={line width=.1pt, draw=gray},
      % yticklabel style={
      %   /pgf/number format/fixed,
      %   /pgf/number format/fixed zerofill,
      %   /pgf/number format/precision=2
      % },        
      %   grid = major
      ]
      \addplot [blue, domain=-5:5] {gauss(0,1)};
      \addplot [gray, fill=gray!50, domain=-1.44:1.44] {gauss(0,1)} \closedcycle;
      \addplot [orange,fill=orange,  domain=1.44:5] {gauss(0,1)} \closedcycle;
      \addplot [orange,fill=orange,  domain=-5:-1.44] {gauss(0,1)} \closedcycle;
      \node (d) at (axis cs: 0,.15) {Area: $0.85$};
      %\draw[thick, ->] (d) -- (axis cs: -.6,.05);
      \node (c) at (axis cs: 3.15,.1) {\og Area: $0.075$};
      \draw[thick,->] (c) -- (axis cs: 1.8, 0.03);
      \node (e) at (axis cs: -3.15,.1) {\og Area: $0.075$};
      \draw[thick,->] (e) -- (axis cs: -1.8, 0.03);
      %\draw[thick, |->] (axis cs: 6,-0.025) -- (axis cs: 10,-0.025) node[below,pos=.5] {\small\og Reject $H_0$};
      %\draw[thick, |->] (axis cs: -6,-0.025) -- (axis cs: -10,-0.025) node[below,pos=.5] {\small\og Reject $H_0$};
    \end{axis}
  \end{tikzpicture}
        \end{minipage}
        \begin{minipage}{.25\linewidth}
          
          \begin{enumerate}[\bf(i)]
            \item 0
     \item $-1.04$
     \item $-1.28$
     \item $-1.44$
     \end{enumerate} 
        \end{minipage}        
        
  \bigskip

\item  Find the area of the shaded portion in the figure below. \pts{2}
\begin{figure}[h!]
  \centering
  \begin{tikzpicture}
    \begin{axis}[no marks,
      samples = 200,
      axis x line=center,
      axis y line=center,
      xtick={0,1,2,...,16},
      ytick={0,0.02,0.04,0.06,0.08,0.10,0.12,0.14},
      domain = 0.1:16.5,
      xlabel={$x$},
      ylabel={$f(x)$},
      xlabel style={right},
      ylabel style={above},
      ymax=0.13,
      xmax=16.5,
      x post scale=1.5, 
      enlargelimits=false,
      grid=both,
      yticklabel style={
        /pgf/number format/fixed,
        /pgf/number format/fixed zerofill,
        /pgf/number format/precision=2
      }
      ]
      \addplot+[draw=blue, ultra thick, opacity=1] {lognormal(2,1)};
      \addplot+[draw=blue, thick, pattern=north east lines, opacity=0.7, domain=0.1:3] {lognormal(2,1)} \closedcycle;
      \node at (axis cs: 6,0.105) {\small\bf Lognormal($\mu=2$, $\sigma=1$)};
      %\node at (axis cs: 8,0.08) {\small Shaded: $P(X \le 3)$};
    \end{axis}
  \end{tikzpicture}
  %\caption{Lognormal distribution with $\mu=2$, $\sigma=1$, showing area where $x \le 3$}
\end{figure}

  \begin{minipage}[]{.1\linewidth}
    {\bf Answer:}
  \end{minipage}\qquad
  \begin{minipage}[]{.8\linewidth}
    \framebox(390,40){\phantom{\Huge t}  }     
  \end{minipage}
 
   \end{enumerate}


\eject

\section*{Problem 3: Lognormal Distribution \textit{(6 points)}}
Given that the lifetime in days of an electronic component is lognormally distributed with $\mu = 1.1$ and $\sigma = 0.5$.

\begin{enumerate}[\bf (a)]
 \item Find the median lifetime of the component. \pts{1}
 \vspace{20ex}
  
\item Find the mean lifetime of the component. \pts{2}
 \vspace{30ex}

\item Find the probability that a component lasts between 3 and 5 days. \pts{3}
  
\end{enumerate}

\eject


\section*{Problem 4: Exponential Distribution I \textit{(4 points)}}
\begin{enumerate}[\bf (a)]

\item The \pts{1} graph below is the PDF of an exponentially distributed random variable $T$, given by $f_T(t) = \lambda
  e^{-\la t}$. What is the value of the parameter $\la$?
 
  \begin{figure}[h!]
    \centering
    \begin{tikzpicture}
      \begin{axis}[no marks,
        samples = 100,
        axis x line=center,
        axis y line=center,
        xtick={0,1,2,...,8},
        ytick={0,0.25,...,1.75},
        domain = 0:8,
        xlabel={$t$},
        ylabel={$f_T(t)$},
        xlabel style={right},
        ylabel style={above },
        ymax=1.8,
        xmax=8.2,
        x post scale=1.7, enlargelimits=false,
        grid=both,
        yticklabel style={
          /pgf/number format/fixed,
          /pgf/number format/fixed zerofill,
          /pgf/number format/precision=2
        }
        ]
        \addplot+[draw=black, ultra thick,opacity=1] {expo(1.5,x)}; %\addlegendentry{\large $\bm{f_{T_X}}$}
        \addplot+[draw=black,thick, pattern=north east lines, opacity=1, domain=0:1] {expo(1.5,x)} \closedcycle; 
      \end{axis}
    \end{tikzpicture}
  \end{figure}

  \vspace{2ex}
  \begin{minipage}[]{.1\linewidth}
    {\bf Answer:}
  \end{minipage}\qquad
  \begin{minipage}[]{.8\linewidth}
    \framebox(390,40){\phantom{\Huge t}  }     
  \end{minipage}

  \bigskip

  \bigskip

  \item What is the mean of $T$? \pts{1} 
  
  \vspace{2ex}
    \begin{minipage}[]{.1\linewidth}
    {\bf Answer:}
  \end{minipage}\qquad
  \begin{minipage}[]{.8\linewidth}
    \framebox(390,40){\phantom{\Huge t}   }     
  \end{minipage}

  \bigskip
  \bigskip
  
\item What is the probability represented by the shaded area in the figure in part \textbf{(i)}? \pts{2} (A numeric value
  is expected here, not just a symbolic expression.)

  \vspace{2ex}
    \begin{minipage}[]{.1\linewidth}
    {\bf Answer:}
  \end{minipage}\qquad
  \begin{minipage}[]{.8\linewidth}
    \framebox(390,80){\phantom{\Huge t}   }     
  \end{minipage}

  
    \end{enumerate}
\eject

\section*{Problem 5: Exponential Distribution II \quad {\it  (7 points)}}
The delay time $T$ of a flight is exponentially distributed with $\la = 4$ (mean rate of occurrence per hour).


\begin{enumerate}[(a)]
\item What is the expectation of $T$? \pts{1}
  \vspace{10ex}

\item What is the standard deviation of $T$? \pts{1}
  \vspace{10ex}

  
\item What is the probability that a flight is delayed by no more than half an hour? \pts{2}
  \vspace{25ex}
  
\item Given that a family member has already waited for half an hour, \pts{3}
 what is the probability that a certain flight will be further delayed by over an hour?

\end{enumerate}

\eject
\section*{Problem 6: Binomial Distribution I \textit{(6 points)}}

\textit{Show brief amount of work for partial credit if answer is wrong. Not required however for full credit. }

\medskip

 The PMF of a random variable $X$ is given in the figure below.
  
  \begin{figure}[h!]
    \centering
    \begin{tikzpicture}
      %\pgfplotsset{minor grid style = {line width = 0.1pt}}
      %\pgfplotsset{major grid style = {line width = 0.4pt}}
      \begin{axis}[
        axis x line=center,
        axis y line=center,
        xtick={0,2,...,20},
        ytick={0.02,0.04,...,.2},
        domain = 0:20,
        samples = 21,
        xlabel={$x$},
        ylabel={$P(X=x)$},
        xlabel style={right},
        ylabel style={above},
        ymax=0.2,
        xmax=20,
        x post scale=1.4,
        grid=both,
        grid style={line width=.1pt, draw=gray!50!black},
        yticklabel style={
            /pgf/number format/fixed,
            /pgf/number format/fixed zerofill,
            /pgf/number format/precision=2
          }
       % minor tick num =2
        ]
        \addplot+[ycomb,black,ultra thick,opacity=.5,mark options={fill=black,scale=1.5,opacity=.8}] {binom(20,.4)}; %\addlegendentry{$\lambda = 1$}}
      \end{axis}
    \end{tikzpicture}
  \end{figure}

  \begin{enumerate}[\bf (a)]

      \item   Use the figure to estimate the probability $P(X = 8)$. \pts{1}

      \vspace{2ex}
  \begin{minipage}[]{.1\linewidth}
    {\bf Answer:}
  \end{minipage}\qquad
  \begin{minipage}[]{.8\linewidth}
    \framebox(390,40){\phantom{\Huge t}  }     
  \end{minipage}

  \bigskip
  
  \item   Use the figure to estimate the probability $P(X = 8 \cup X = 10)$.\pts{2}  % Answer: 0.15

      \vspace{2ex}
  \begin{minipage}[]{.1\linewidth}
    {\bf Answer:}
  \end{minipage}\qquad
  \begin{minipage}[]{.8\linewidth}
    \framebox(390,40){\phantom{\Huge t}   }     
  \end{minipage}

  \bigskip

    \item   Use the figure to estimate the probability $P(5 < X \le 8)$.\pts{2} 

      \vspace{2ex}
  \begin{minipage}[]{.1\linewidth}
    {\bf Answer:}
  \end{minipage}\qquad
  \begin{minipage}[]{.8\linewidth}
    \framebox(390,40){\phantom{\Huge t} }     
  \end{minipage}

  \bigskip

\item If the PMF in the figure above is that of a Binomial distribution with $p=0.4$, what is $\mathbb{E}(X)$? \pts{1}

  \vspace{2ex}
  \begin{minipage}[]{.1\linewidth}
    {\bf Answer:}
  \end{minipage}\qquad
  \begin{minipage}[]{.8\linewidth}
    \framebox(390,40){\phantom{\Huge t}   }     
  \end{minipage}

  
  
  \end{enumerate}

 
 \eject

 \section*{Problem 7: Binomial Distribution II \textit{(8 points)}}
75\% of all vehicles examined at an emissions inspection station pass.
Successive vehicles pass or fail independently of one another. Let $X$ be the number of vehicles that pass the inspection out of the next $n=6$  vehicles inspected.

\begin{enumerate}[\bf (a)]
  \item What is the expectation of $X$, i.e. $\mathbb{E}[X]$? \pts{1}
 \vspace{10ex}

\item What is the standard deviation of $X$? \pts{1}
 
  \vspace{10ex}

\item Find the probability that all of the next six vehicles inspected pass, i.e. $P(X=6)$.\pts{1}

  \vspace{20ex}

  
\item Find the probability that only two of the next six vehicles inspected pass, i.e. $P(X=2)$.\pts{2}
 \vspace{25ex}
  
\item Find the probability that at least four of the next six vehicles inspected pass.\pts{3}
  \vspace{35ex}

  

  
\end{enumerate}

\eject

\end{document}

%%% Local Variables:
%%% mode: latex
%%% TeX-master: t
%%% End:
