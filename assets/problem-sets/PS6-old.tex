\documentclass[11pt,twoside]{article}
\usepackage{etex}

\raggedbottom

%geometry (sets margin) and other useful packages
\usepackage{geometry}
\geometry{top=1in, left=1in,right=1in,bottom=1in}
 \usepackage{graphicx,booktabs,calc}
 
\usepackage{listings}


% Marginpar width
%Marginpar width
\newcommand{\pts}[1]{\marginpar{ \small\hspace{0pt} \textit{[#1]} } } 
\setlength{\marginparwidth}{.5in}
%\reversemarginpar
%\setlength{\marginparsep}{.02in}

 
%\usepackage{cmbright}lstinputlisting
%\usepackage[T1]{pbsi}


\usepackage{chngcntr,mathtools}
\counterwithin{figure}{section}
%\numberwithin{equation}{section}

%\usepackage{listings}

%AMS-TeX packages
\usepackage{amssymb,amsmath,amsthm} 
\usepackage{bm}
\usepackage[mathscr]{eucal}
\usepackage{colortbl}
\usepackage{color}


\usepackage{subfigure,hyperref,enumerate,polynom,polynomial}
\usepackage{multirow,minitoc,fancybox,array,multicol}

\definecolor{slblue}{rgb}{0,.3,.62}
\hypersetup{
    colorlinks,%
    citecolor=blue,%
    filecolor=blue,%
    linkcolor=blue,
    urlcolor=slblue
}

%%%TIKZ
\usepackage{tikz}

\usepackage{pgfplots}
\pgfplotsset{compat=newest}

\usetikzlibrary{arrows,shapes,positioning}
\usetikzlibrary{decorations.markings}
\usetikzlibrary{shadows}
\usetikzlibrary{patterns}
%\usetikzlibrary{circuits.ee.IEC}
\usetikzlibrary{decorations.text}
% For Sagnac Picture
\usetikzlibrary{%
    decorations.pathreplacing,%
    decorations.pathmorphing%
}

\tikzstyle arrowstyle=[black,scale=2]
\tikzstyle directed=[postaction={decorate,decoration={markings,
    mark=at position .65 with {\arrow[arrowstyle]{stealth}}}}]
\tikzstyle reverse directed=[postaction={decorate,decoration={markings,
    mark=at position .65 with {\arrowreversed[arrowstyle]{stealth};}}}]
\tikzstyle dir=[postaction={decorate,decoration={markings,
    mark=at position .98 with {\arrow[arrowstyle]{latex}}}}]
\tikzstyle rev dir=[postaction={decorate,decoration={markings,
    mark=at position .98 with {\arrowreversed[arrowstyle]{latex};}}}]

\usepackage{ctable}

%
%Redefining sections as problems
%
\makeatletter
\newenvironment{exercise}{\@startsection 
	{section}
	{1}
	{-.2em}
	{-3.5ex plus -1ex minus -.2ex}
    	{1.3ex plus .2ex}
    	{\pagebreak[3]%forces pagebreak when space is small; use \eject for better results
	\large\bf\noindent{Exercise 1.\hspace{-1.5ex} }
	}
	}
	%{\vspace{1ex}\begin{center} \rule{0.3\linewidth}{.3pt}\end{center}}
	%\begin{center}\large\bf \ldots\ldots\ldots\end{center}}
\makeatother

%
%Fancy-header package to modify header/page numbering 
%
\usepackage{fancyhdr}
\pagestyle{fancy}
%\addtolength{\headwidth}{\marginparsep} %these change header-rule width
%\addtolength{\headwidth}{\marginparwidth}
%\fancyheadoffset{30pt}
%\fancyfootoffset{30pt}
\fancyhead[LO,RE]{\small Oke}
\fancyhead[RO,LE]{\small Page \thepage} 
\fancyfoot[RO,LE]{\small PS 6  } 
\fancyfoot[LO,RE]{\small \scshape CEE 260/MIE 273} 
\cfoot{} 
\renewcommand{\headrulewidth}{0.1pt} 
\renewcommand{\footrulewidth}{0.1pt}
%\setlength\voffset{-0.25in}
%\setlength\textheight{648pt}


\usepackage{paralist}

\newcommand{\osn}{\oldstylenums}
\newcommand{\lt}{\left}
\newcommand{\rt}{\right}
\newcommand{\pt}{\phantom}
\newcommand{\tf}{\therefore}
\newcommand{\?}{\stackrel{?}{=}}
\newcommand{\fr}{\frac}
\newcommand{\dfr}{\dfrac}
\newcommand{\ul}{\underline}
\newcommand{\tn}{\tabularnewline}
\newcommand{\nl}{\newline}
\newcommand\relph[1]{\mathrel{\phantom{#1}}}
\newcommand{\cm}{\checkmark}
\newcommand{\ol}{\overline}
\newcommand{\rd}{\color{red}}
\newcommand{\bl}{\color{blue}}
\newcommand{\pl}{\color{purple}}
\newcommand{\og}{\color{orange!90!black}}
\newcommand{\gr}{\color{green!40!black}}
\newcommand{\nin}{\noindent}
\newcommand{\la}{\lambda}
\renewcommand{\th}{\theta}
\newcommand*\circled[1]{\tikz[baseline=(char.base)]{
            \node[shape=circle,draw,thick,inner sep=1pt] (char) {\small #1};}}

\newcommand{\bc}{\begin{compactenum}[\quad--]}
\newcommand{\ec}{\end{compactenum}}

\newcommand{\n}{\\[2mm]}
%% GREEK LETTERS
\newcommand{\al}{\alpha}
\newcommand{\gam}{\gamma}
\newcommand{\eps}{\epsilon}
\newcommand{\sig}{\sigma}

\newcommand{\p}{\partial}
\newcommand{\pd}[2]{\frac{\partial{#1}}{\partial{#2}}}
\newcommand{\dpd}[2]{\dfrac{\partial{#1}}{\partial{#2}}}
\newcommand{\pdd}[2]{\frac{\partial^2{#1}}{\partial{#2}^2}}
\newcommand{\mr}{\mathbb{R}}
\newcommand{\xs}{x^{*}}
\newenvironment{solution}
{\medskip\par\quad\quad\begin{minipage}[c]{.8\textwidth}\gr}{\medskip\end{minipage}}


\pgfmathdeclarefunction{poiss}{1}{%
  \pgfmathparse{(#1^x)*exp(-#1)/(x!)}%
  }

\pgfmathdeclarefunction{gauss}{2}{%
  \pgfmathparse{1/(#2*sqrt(2*pi))*exp(-((x-#1)^2)/(2*#2^2))}%
}

\pgfmathdeclarefunction{expo}{2}{%
  \pgfmathparse{#1*exp(-#1*#2)}%
}

\usetikzlibrary{math}

% https://tex.stackexchange.com/questions/461758/asymmetric-distribution-gauss-curve
\tikzmath{%
  function h1(\x, \lx) { return (9*\lx + 3*((\lx)^2) + ((\lx)^3)/3 + 9); };
  function h2(\x, \lx) { return (3*\lx - ((\lx)^3)/3 + 4); };
  function h3(\x, \lx) { return (9*\lx - 3*((\lx)^2) + ((\lx)^3)/3 + 7); };
  function skewnorm(\x, \l) {
    \x = (\l < 0) ? -\x : \x;
    \l = abs(\l);
    \e = exp(-(\x^2)/2);
    return (\l == 0) ? 1 / sqrt(2 * pi) * \e: (
      (\x < -3/\l) ? 0 : (
      (\x < -1/\l) ? \e / (8 * sqrt(2 * pi)) * h1(\x, \x*\l) : (
      (\x <  1/\l) ? \e / (4 * sqrt(2 * pi)) * h2(\x, \x*\l) : (
      (\x <  3/\l) ? \e / (8 * sqrt(2 * pi)) * h3(\x, \x*\l) : (
      sqrt(2/pi) * \e)))));
  };
}

\def\cdf(#1)(#2)(#3){0.5*(1+(erf((#1-#2)/(#3*sqrt(2)))))}%
% to be used: \cdf(x)(mean)(variance)

\DeclareMathOperator{\CDF}{cdf}
\tikzset{
    declare function={
        normcdf(\x,\m,\s)=1/(1 + exp(-0.07056*((\x-\m)/\s)^3 - 1.5976*(\x-\m)/\s));
    }
}
%%%%%%%%%%%%%%%%%%%%%%%%%%%%%%%%%%%%%%%%%%%%%%%%%%%
%%%%%%%%%%%%%%%%%%%%%%%%%%%%%%%%%%%%%%%%%%%%%%%%%%%

\begin{document}

\lstset{language=C++,
                basicstyle=\tiny\ttfamily,
                keywordstyle=\color{blue}\ttfamily,
                stringstyle=\color{red}\ttfamily,
                commentstyle=\color{gray}\ttfamily,
                morecomment=[l][\color{gray}]{\#}
}


\thispagestyle{empty}


\nin{\LARGE Problem Set 6 {\gr  }}\hfill{\bf Prof. Oke}

\medskip\hrule\medskip

\nin {\small CEE 260/MIE 273: Probability \& Statistics in Civil Engineering
\hfill\textit{ 10.29.2024}}

\bigskip

\nin{\it \textbf{Due \textbf{Tuesday}, November 12, 2024 at 11:59 PM as PDF uploaded on Moodle.}
  I strongly encourage you to write/type your responses directly on this document and upload it.
    \textbf{Show as much work as possible in order to get FULL credit.}
    There are 7 problems with a total of 67 points available.
    \textbf{Important:} If you use MATLAB/Python for any probability computations, briefly write/include the statements you
    used to arrive at your answers. If instead you use probability tables, note this in the respective solution, as well.
}\\


 

  % \smallskip
  
% \item \hfill
%   \begin{minipage}{.1\linewidth}
%     \framebox(40,40){}
%   \end{minipage}\quad
%   \begin{minipage}{.85\linewidth}
%     A good clustering solution should maximize the total within-cluster variation. %False
%   \end{minipage}

%\eject  
% \section*{Problem 2 \textit{(10 points)}}

% \textit{Show brief amount of work for partial credit if answer is wrong. Not required however for full credit. }

% \medskip

   

%   \begin{enumerate}[\bf (a)]

%   \item   Write down the expression of the probability represented by the shaded portion of the normal PDF below. \pts{1} For example, $P(X \le
%     2)$. Note that a dashed vertical boundary indicates ``$>$'' or ``$<$,'' while a solid vertical boundary indicates ``$\ge$'' or
%     ``$\le$.''

%           \begin{figure}[h!]
%     \centering
%       \begin{tikzpicture}
%     \begin{axis}[no markers, domain=0:10, samples=100,
%       axis lines*=left, xlabel=$x$, ylabel=$f_X(x)$,,
%       height=6cm, width=10cm,
%       %xtick={-3, -2, -1, 0, 1, 2, 3},
%       %ytick=\empty,
%       enlargelimits=false, clip=false, axis on top,
%       grid style={line width=.1pt, draw=gray},
%       yticklabel style={
%             /pgf/number format/fixed,
%             /pgf/number format/fixed zerofill,
%             /pgf/number format/precision=2
%           },        
%       grid = major]
%       \addplot [ultra thick, domain=-2:12] {gauss(5,2)};
%       \addplot [thick,draw=none, pattern=north west lines,  domain=6:12] {gauss(5,2)} \closedcycle;
%       \addplot [dashed, ultra thick,] coordinates {(6,0) (6,{gauss(5,2)})};
%     \end{axis}
%   \end{tikzpicture}

% \end{figure}

%       \vspace{2ex}
%   \begin{minipage}[]{.1\linewidth}
%     {\bf Answer:}
%   \end{minipage}\qquad
%   \begin{minipage}[]{.8\linewidth}
%     \framebox(390,40){\phantom{\Huge t}  \gr $P(X > 6)$ }     
%   \end{minipage}

%   \bigskip
  
%   \item   Write down the expression of the probability represented by the shaded portion of the normal PDF below. \pts{1} For example, $P(X \le
%     2)$. Note that a dashed vertical boundary indicates ``$>$'' or ``$<$,'' while a solid vertical boundary indicates ``$\ge$'' or
%     ``$\le$.''

%     \begin{figure}[h!]
%       \centering
%       \begin{tikzpicture}
%         \begin{axis}[no markers, domain=0:10, samples=100,
%           axis lines*=left, xlabel=$x$, ylabel=$f_X(x)$,,
%           height=6cm, width=10cm,
%           % xtick={-3, -2, -1, 0, 1, 2, 3},
%           % ytick=\empty,
%           enlargelimits=false, clip=false, axis on top,
%           grid style={line width=.1pt, draw=gray},
%           yticklabel style={
%             /pgf/number format/fixed,
%             /pgf/number format/fixed zerofill,
%             /pgf/number format/precision=2
%           },        
%           grid = major]
%           \addplot [ultra thick, domain=-3:3] {gauss(0,1)};
%           \addplot [thick,draw=none, pattern=north west lines,  domain=-3:0] {gauss(0,1)} \closedcycle;
%           \addplot [ultra thick] coordinates {(0,0) (0,{gauss(0,1)})};
%         \end{axis}
%       \end{tikzpicture}

%     \end{figure}


%       \vspace{2ex}
%   \begin{minipage}[]{.1\linewidth}
%     {\bf Answer:}
%   \end{minipage}\qquad
%   \begin{minipage}[]{.8\linewidth}
%     \framebox(390,40){\phantom{\Huge t}  \gr $P(X\le 0)$}     
%   \end{minipage}

 

  
  
%   \end{enumerate}


%   \eject

\section*{Problem 1 \textit{(8 points)}}

Respond ``T'' ({\it True})  or  ``F'' (\textit{False}) to the following statements. Use the boxes provided. Each response is worth 1 point.

\begin{enumerate}[\bf (i)]
\item \hfill
  \begin{minipage}{.1\linewidth}
    \framebox(40,40){ \gr }
  \end{minipage}\quad
  \begin{minipage}{.85\linewidth}
   A sample of size $n=100$ has a proportion parameter estimate $\hat{p} = 0.3$. If the sample observations are independent, then the sample satisfies the success-failure condition. 
   \end{minipage}
  
  \smallskip

  \item \hfill
  \begin{minipage}{.1\linewidth}
    \framebox(40,40){\gr  }
  \end{minipage}\quad
  \begin{minipage}{.85\linewidth}
    For a given confidence level, the margin of error decreases as the sample size increases.
  \end{minipage}

\smallskip

\item \hfill
  \begin{minipage}{.1\linewidth}
    \framebox(40,40){\gr  }
  \end{minipage}\quad
  \begin{minipage}{.85\linewidth}
    If the 95\% confidence interval for a proportion $p$ is $(0.669, 0.731)$, then the margin of error is 0.062. 
  \end{minipage}

  \smallskip

  \item \hfill
  \begin{minipage}{.1\linewidth}
    \framebox(40,40){\gr  }
  \end{minipage}\quad
  \begin{minipage}{.85\linewidth}
   In computing a confidence interval, if the critical $Z$-score is 2.58, then the $\alpha$ corresponding to the confidence level of interest is $0.01$.
   \end{minipage}

  \smallskip


\item \hfill
  \begin{minipage}{.1\linewidth}
    \framebox(40,40){\gr  }
  \end{minipage}\quad
  \begin{minipage}{.85\linewidth}
    In a hypothesis test, the significance level $\alpha$ can also be understood as the probability of making a Type II error.
  \end{minipage}

  \smallskip
  
\item \hfill
  \begin{minipage}{.1\linewidth}
    \framebox(40,40){\gr  }
  \end{minipage}\quad
  \begin{minipage}{.85\linewidth}
    In hypothesis testing, failure to reject the null hypothesis $H_0$ does not mean the null hypothesis is necessarily true.
  \end{minipage}

  \smallskip
  

\item \hfill
  \begin{minipage}{.1\linewidth}
    \framebox(40,40){\gr  }
  \end{minipage}\quad
  \begin{minipage}{.85\linewidth}
    In a hypothesis test, if the $p$-value for the test statistic is 0.084 and $\alpha = 0.05$, then we would reject the null hypothesis.
  \end{minipage}

  \item \hfill
  \begin{minipage}{.1\linewidth}
    \framebox(40,40){\gr  }
  \end{minipage}\quad
  \begin{minipage}{.85\linewidth}
    In a hypothesis test, if the $p$-value for the test statistic is $3.2\times 10^{-5}$ and $\alpha = 0.01$, then we would reject the null hypothesis.
  \end{minipage}  

\end{enumerate}

\eject
 
  \section*{Problem 2 \quad {\it Confidence interval of a proportion (10 points)}      }
  A poll conducted in 2013 found that 52\% of U.S. adult X (formerly Twitter) users get at least some news on X. The standard error for this estimate was 2.4\%, and a normal distribution may be used to model the sample proportion $\hat p$. Now answer the following step-by-step questions in order to   construct a 95\% confidence interval for the fraction of U.S. adult X users who get some news on X.

  
  \begin{enumerate}[\bf (a)]
    \item State the value of $\hat p$. \pts{1}
    \vspace{2ex}
    
    \item Find the corresponding critical $Z$-score for a 95\% confidence level. Show how you obtained your answer (wehther its via Matlab/Python or another source.)\pts{1}
    \vspace{14ex}

    \item Compute the margin of error. \pts{2}
    \vspace{16ex}

    \item Write the 95\% confidence interval of $p$. \pts{2}
    \vspace{14ex}

    \item Briefly interpret the interval in the context of this question. \pts{2}
    \vspace{14ex}

    \item Would you expect the 99\% confidence interval to be narrower (smaller) or wider (larger)? Provide a reason for your answer. \pts{2}
  \end{enumerate}
  

 
\eject


\section*{Problem 3 \quad {\it CI of proportion and sample size (10 points)}      }
An article reports that when each football helmet in a random sample of 37 suspension-type helmets was subjected to a certain impact test, 24 showed damage. Let $p$ denote the proportion of all helmets of this type that would show damage when tested in the prescribed manner.

\begin{enumerate}[\bf (a)]

\item  Calculate a 99\% CI (confidence interval) for $p$. \pts{6}
\vspace{60ex}

\item What sample size would be required for the width of a 99\% CI to be  0.1? \pts{4}
\end{enumerate}

\eject

\section*{Problem 4\quad {\it Hypothesis testing (5 points)}}
In the following hypothesis tests, decide whether to (i) ``Reject $H_{0}$'' or (ii) ``Fail to reject $H_{0}$'' by comparing the $Z$-scores ($z$) to the critical values ($z_{\fr\alpha2}$, $z_{(1-\fr\alpha2)}$, etc.) at the boundaries of the critical regions (in orange).  Circle the correct decision in each case.

\begin{enumerate}[\bf (a)]

    
\item  $H_{0}: p = p_{0}; H_{1}: p > p_{0}$
 
    \begin{figure}[h!]
      \centering
      \begin{tikzpicture}[
        declare function={gamma(\z)=
          2.506628274631*sqrt(1/\z)+ 0.20888568*(1/\z)^(1.5)+ 0.00870357*(1/\z)^(2.5)
          - (174.2106599*(1/\z)^(3.5))/25920- (715.6423511*(1/\z)^(4.5))/1244160)*exp((-ln(1/\z)-1)*\z;},
        declare function={student(\x,\n)= gamma((\n+1)/2.)/(sqrt(\n*pi) *gamma(\n/2.)) *((1+(\x*\x)/\n)^(-(\n+1)/2.));}
        ]
        \begin{axis}[no markers, domain=-5:5, samples=100,
          axis x line=center,
          axis y line=none,
          every x tick/.style={color=black, thick},          
          xlabel=$Z$, ylabel=$f_X(x)$,
          height=3cm, width=14cm,
          every x tick/.style={color=black, thick},
          xtick={-1, 0, 2.326},
          xticklabels={$z$,0,$z_{(1-\alpha)}$},
          ymax=.15,
          ytick=\empty,
          x label style={anchor=west},
          y label style={anchor=south},
          enlargelimits=true, clip=false, axis on top
          ]
          \addplot [blue, domain=-5:5] {gauss(0,1)};
          \addplot [gray, fill=gray!50, domain=-5:2.326] {gauss(0,1)} \closedcycle;
          \addplot [orange,fill=orange,  domain=2.326:5] {gauss(0,1)} \closedcycle;
          %\node (d) at (axis cs: 0,.1) {Area: .90};
          %\node (c) at (axis cs: 3.2,.08) {\og Area: .05};
          %\draw[thick,->] (c) -- (axis cs: 2, 0.03);
          %\node (c) at (axis cs: -3.2,.08) {\og Area: .05};
          %\draw[thick,->] (c) -- (axis cs: -2, 0.03);
        \end{axis}
      \end{tikzpicture}
      % \caption{Hypothesis test for Problem 6.1}
    \end{figure}

    {\bf i.} Reject $H_0$ \qquad {\bf ii.} Fail to reject $H_0$
    
    \bigskip
    \bigskip

    \item  $H_{0}: p = p_{0}; H_{1}: p > p_{0}$
 
    \begin{figure}[h!]
      \centering
      \begin{tikzpicture}[
        declare function={gamma(\z)=
          2.506628274631*sqrt(1/\z)+ 0.20888568*(1/\z)^(1.5)+ 0.00870357*(1/\z)^(2.5)
          - (174.2106599*(1/\z)^(3.5))/25920- (715.6423511*(1/\z)^(4.5))/1244160)*exp((-ln(1/\z)-1)*\z;},
        declare function={student(\x,\n)= gamma((\n+1)/2.)/(sqrt(\n*pi) *gamma(\n/2.)) *((1+(\x*\x)/\n)^(-(\n+1)/2.));}
        ]
        \begin{axis}[no markers, domain=-5:5, samples=100,
          axis x line=center,
          axis y line=none,
          every x tick/.style={color=black, thick},          
          xlabel=$Z$, ylabel=$f_X(x)$,
          height=3cm, width=14cm,
          every x tick/.style={color=black, thick},
          xtick={-1.9, -1, 0},
          xticklabels={$z$, $z_{\alpha}$, 0 },
          ymax=.15,
          ytick=\empty,
          x label style={anchor=west},
          y label style={anchor=south},
          enlargelimits=true, clip=false, axis on top
          ]
          \addplot [blue, domain=-5:5] {gauss(0,1)};
          \addplot [gray, fill=gray!50, domain=-1.65:5] {gauss(0,1)} \closedcycle;
          \addplot [orange,fill=orange,  domain=-5:-1.65] {gauss(0,1)} \closedcycle;
          %\node (d) at (axis cs: 0,.1) {Area: .90};
          %\node (c) at (axis cs: 3.2,.08) {\og Area: .05};
          %\draw[thick,->] (c) -- (axis cs: 2, 0.03);
          %\node (c) at (axis cs: -3.2,.08) {\og Area: .05};
          %\draw[thick,->] (c) -- (axis cs: -2, 0.03);
        \end{axis}
      \end{tikzpicture}
    \end{figure}

    {\bf i.} Reject $H_0$ \qquad {\bf ii.} Fail to reject $H_0$
    
    \bigskip
    \bigskip

\item  $H_{0}: p = p_{0}; H_{1}: p \ne p_{0}$
 
    \begin{figure}[h!]
      \centering
      \begin{tikzpicture}[
        declare function={gamma(\z)=
          2.506628274631*sqrt(1/\z)+ 0.20888568*(1/\z)^(1.5)+ 0.00870357*(1/\z)^(2.5)
          - (174.2106599*(1/\z)^(3.5))/25920- (715.6423511*(1/\z)^(4.5))/1244160)*exp((-ln(1/\z)-1)*\z;},
        declare function={student(\x,\n)= gamma((\n+1)/2.)/(sqrt(\n*pi) *gamma(\n/2.)) *((1+(\x*\x)/\n)^(-(\n+1)/2.));}
        ]
        \begin{axis}[no markers, domain=-5:5, samples=100,
          axis x line=center,
          axis y line=none,
          xlabel=$Z$, ylabel=$f_X(x)$,
          height=3cm, width=14cm,
          every x tick/.style={color=black, thick},
          xtick={-1.96,-0.5, 0,1.96},
          xticklabels={$z_{\fr\alpha2}$,$z$,0,$z_{(1-\fr\alpha2)}$},
          ymax=.15,
          ytick=\empty,
          x label style={anchor=west},
          y label style={anchor=south},
          enlargelimits=true, clip=false, axis on top
          ]
          \addplot [blue, domain=-5:5] {gauss(0,1)};
          \addplot [gray, fill=gray!50, domain=-1.96:1.96] {gauss(0,1)} \closedcycle;
          \addplot [orange,fill=orange,  domain=-5:-1.96] {gauss(0,1)} \closedcycle;
          \addplot [orange,fill=orange,  domain=1.96:5] {gauss(0,1)} \closedcycle;
          %\node (d) at (axis cs: 0,.1) {Area: .90};
          %\node (c) at (axis cs: 3.2,.08) {\og Area: .05};
          %\draw[thick,->] (c) -- (axis cs: 2, 0.03);
          %\node (c) at (axis cs: -3.2,.08) {\og Area: .05};
          %\draw[thick,->] (c) -- (axis cs: -2, 0.03);
        \end{axis}
      \end{tikzpicture}
      % \caption{Hypothesis test for Problem 6.1}
    \end{figure}

    {\bf i.} Reject $H_0$ \qquad {\bf ii.} Fail to reject $H_0$
    
    \bigskip
    \bigskip

    \eject
  \item  $H_{0}:p = p_{0}; H_{1}: p \ne p_{0}$
 
    \begin{figure}[h!]
      \centering
      \begin{tikzpicture}[
        declare function={gamma(\z)=
          2.506628274631*sqrt(1/\z)+ 0.20888568*(1/\z)^(1.5)+ 0.00870357*(1/\z)^(2.5)
          - (174.2106599*(1/\z)^(3.5))/25920- (715.6423511*(1/\z)^(4.5))/1244160)*exp((-ln(1/\z)-1)*\z;},
        declare function={student(\x,\n)= gamma((\n+1)/2.)/(sqrt(\n*pi) *gamma(\n/2.)) *((1+(\x*\x)/\n)^(-(\n+1)/2.));}
        ]
        \begin{axis}[no markers, domain=-5:5, samples=100,
          axis x line=center,
          axis y line=none,
          xlabel=$Z$, ylabel=$f_X(x)$,
          height=3cm, width=14cm,
          every x tick/.style={color=black, thick},
          xtick={-1.645, 0,1.645, 3},
          xticklabels={$z_{\fr\alpha2}$,0,$z_{(1-\fr\alpha2)}$, $z$},
          ymax=.15,
          ytick=\empty,
          x label style={anchor=west},
          y label style={anchor=south},
          enlargelimits=true, clip=false, axis on top
          ]
          \addplot [blue, domain=-5:5] {gauss(0,1)};
          \addplot [gray, fill=gray!50, domain=-1.645:1.645] {gauss(0,1)} \closedcycle;
          \addplot [orange,fill=orange,  domain=-5:-1.645] {gauss(0,1)} \closedcycle;
          \addplot [orange,fill=orange,  domain=1.645:5] {gauss(0,1)} \closedcycle;
          %\node (d) at (axis cs: 0,.1) {Area: .90};
          %\node (c) at (axis cs: 3.2,.08) {\og Area: .05};
          %\draw[thick,->] (c) -- (axis cs: 2, 0.03);
          %\node (c) at (axis cs: -3.2,.08) {\og Area: .05};
          %\draw[thick,->] (c) -- (axis cs: -2, 0.03);
        \end{axis}
      \end{tikzpicture}
      % \caption{Hypothesis test for Problem 6.1}
    \end{figure}
    {\bf i.} Reject $H_0$ \qquad {\bf ii.} Fail to reject $H_0$
    
    \bigskip
    \bigskip
 % \end{quote}
\end{enumerate}
 





 \section*{Problem 5 \quad {\it Two-tailed hypothesis test using critical values (10 points)}      }
 400 students were randomly sampled from a large university, and 289 said they did not get enough sleep. Conduct a hypothesis test to check whether this represents a statistically significant difference from 50\%, and use a significance level of 0.05.
 In this problem, you are required to use the critical value appreoach (so no need to compute p-values). Your response will be graded on the following steps:
 \begin{compactitem}
  \item State the hypotheses (there are two) \pts{2}
  \item Compute test statistic \pts{2}
  \item Compute the critical values \pts{2}
  \item Compare test statistic to critical values \pts{2}
  \item State outcome of test and write concluding statement \pts{2}
 \end{compactitem}


 \eject 
 ~
 \eject

 \section*{Problem 6 \quad {\it Two-tailed hypothesis test using p-value (12 points)}      }
 It is believed that nearsightedness affects about 8\% of all children. In a random sample of 194 children, 21 are nearsighted. Conduct a hypothesis test ($\alpha=.05$) for the following question: do these data provide evidence that the 8\% value is inaccurate?
 Your response will be graded on the following steps:
 \begin{compactitem}
   \item State the hypotheses (there are two) \pts{2}
   \item Find the standard error \pts{2}
   \item Find the test statistic \pts{2}
   \item Find the p-value \pts{2}
   \item Compare the appropriate values \pts{1}
   \item Clearly state the outcome from your hypothesis test \pts{1}
   \item Write a final concluding statement in response to the question \pts{2}
   \end{compactitem}

   \eject~
   \eject

\section*{Problem 7 \quad {\it Difference of two proportions hypothesis testing using p-values (12 points) }      }
A quadcopter company is considering a new supplier for rotor blades. The new supplier would be more expensive, but they claim their higher-quality blades are more reliable,  with 3\% more blades passing inspection than their competitor.
The company's quality control engineer collects a sample of blades, examining 1000 blades from each supplier, and she finds that 899 blades pass inspection from the current supplier and 958 pass inspection from the prospective supplier.
Using these data, set up and evaluate the hypotheses with a significance level $\alpha = .01$.
Your response will be graded on the following steps:
\begin{compactitem}
  \item State the hypotheses (there are two) \pts{2}
  \item Find the standard error \pts{2}
  \item Find the test statistic \pts{2}
  \item Find the p-value \pts{2}
  \item Compare the appropriate values \pts{1}
  \item Clearly state the outcome from your hypothesis test \pts{1}
  \item Write a final concluding statement in response to the question \pts{2}
  \end{compactitem}


\eject  
~
\end{document}

%%% Local Variables:
%%% mode: latex
%%% TeX-master: t
%%% End:
