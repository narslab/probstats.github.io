\documentclass[11pt,twoside]{article}
\usepackage{etex}

\raggedbottom

%geometry (sets margin) and other useful packages
\usepackage{geometry}
\geometry{top=1in, left=1in,right=1in,bottom=1in}
 \usepackage{graphicx,booktabs,calc}
 
\usepackage{listings}


% Marginpar width
%Marginpar width
\newcommand{\pts}[1]{\marginpar{ \small\hspace{0pt} \textit{[#1]} } } 
\setlength{\marginparwidth}{.5in}
%\reversemarginpar
%\setlength{\marginparsep}{.02in}

 
%\usepackage{cmbright}lstinputlisting
%\usepackage[T1]{pbsi}


\usepackage{chngcntr,mathtools}
\counterwithin{figure}{section}
\numberwithin{equation}{section}

%\usepackage{listings}

%AMS-TeX packages
\usepackage{amssymb,amsmath,amsthm} 
\usepackage{bm}
\usepackage[mathscr]{eucal}
\usepackage{colortbl}
\usepackage{color}


\usepackage{subfigure,hyperref,enumerate,polynom,polynomial}
\usepackage{multirow,minitoc,fancybox,array,multicol}

\definecolor{slblue}{rgb}{0,.3,.62}
\hypersetup{
    colorlinks,%
    citecolor=blue,%
    filecolor=blue,%
    linkcolor=blue,
    urlcolor=slblue
}

%%%TIKZ
\usepackage{tikz}

\usepackage{pgfplots}
\pgfplotsset{compat=newest}

\usetikzlibrary{arrows,shapes,positioning}
\usetikzlibrary{decorations.markings}
\usetikzlibrary{shadows}
\usetikzlibrary{patterns}
%\usetikzlibrary{circuits.ee.IEC}
\usetikzlibrary{decorations.text}
% For Sagnac Picture
\usetikzlibrary{%
    decorations.pathreplacing,%
    decorations.pathmorphing%
}

\tikzstyle arrowstyle=[black,scale=2]
\tikzstyle directed=[postaction={decorate,decoration={markings,
    mark=at position .65 with {\arrow[arrowstyle]{stealth}}}}]
\tikzstyle reverse directed=[postaction={decorate,decoration={markings,
    mark=at position .65 with {\arrowreversed[arrowstyle]{stealth};}}}]
\tikzstyle dir=[postaction={decorate,decoration={markings,
    mark=at position .98 with {\arrow[arrowstyle]{latex}}}}]
\tikzstyle rev dir=[postaction={decorate,decoration={markings,
    mark=at position .98 with {\arrowreversed[arrowstyle]{latex};}}}]

\usepackage{ctable}

%
%Redefining sections as problems
%
\makeatletter
\newenvironment{exercise}{\@startsection 
	{section}
	{1}
	{-.2em}
	{-3.5ex plus -1ex minus -.2ex}
    	{1.3ex plus .2ex}
    	{\pagebreak[3]%forces pagebreak when space is small; use \eject for better results
	\large\bf\noindent{Exercise 1.\hspace{-1.5ex} }
	}
	}
	%{\vspace{1ex}\begin{center} \rule{0.3\linewidth}{.3pt}\end{center}}
	%\begin{center}\large\bf \ldots\ldots\ldots\end{center}}
\makeatother

%
%Fancy-header package to modify header/page numbering 
%
\usepackage{fancyhdr}
\pagestyle{fancy}
%\addtolength{\headwidth}{\marginparsep} %these change header-rule width
%\addtolength{\headwidth}{\marginparwidth}
%\fancyheadoffset{30pt}
%\fancyfootoffset{30pt}
\fancyhead[LO,RE]{\small Oke}
\fancyhead[RO,LE]{\small Page \thepage} 
\fancyfoot[RO,LE]{\small PS 6} 
\fancyfoot[LO,RE]{\small \scshape CEE 260/MIE 273} 
\cfoot{} 
\renewcommand{\headrulewidth}{0.1pt} 
\renewcommand{\footrulewidth}{0.1pt}
%\setlength\voffset{-0.25in}
%\setlength\textheight{648pt}


\usepackage{paralist}

\newcommand{\osn}{\oldstylenums}
\newcommand{\lt}{\left}
\newcommand{\rt}{\right}
\newcommand{\pt}{\phantom}
\newcommand{\tf}{\therefore}
\newcommand{\?}{\stackrel{?}{=}}
\newcommand{\fr}{\frac}
\newcommand{\dfr}{\dfrac}
\newcommand{\ul}{\underline}
\newcommand{\tn}{\tabularnewline}
\newcommand{\nl}{\newline}
\newcommand\relph[1]{\mathrel{\phantom{#1}}}
\newcommand{\cm}{\checkmark}
\newcommand{\ol}{\overline}
\newcommand{\rd}{\color{red}}
\newcommand{\bl}{\color{blue}}
\newcommand{\pl}{\color{purple}}
\newcommand{\og}{\color{orange!90!black}}
\newcommand{\gr}{\color{green!40!black}}
\newcommand{\nin}{\noindent}
\newcommand{\la}{\lambda}
\renewcommand{\th}{\theta}
\newcommand*\circled[1]{\tikz[baseline=(char.base)]{
            \node[shape=circle,draw,thick,inner sep=1pt] (char) {\small #1};}}

\newcommand{\bc}{\begin{compactenum}[\quad--]}
\newcommand{\ec}{\end{compactenum}}

\newcommand{\n}{\\[2mm]}
%% GREEK LETTERS
\newcommand{\al}{\alpha}
\newcommand{\gam}{\gamma}
\newcommand{\eps}{\epsilon}
\newcommand{\sig}{\sigma}

\newcommand{\p}{\partial}
\newcommand{\pd}[2]{\frac{\partial{#1}}{\partial{#2}}}
\newcommand{\dpd}[2]{\dfrac{\partial{#1}}{\partial{#2}}}
\newcommand{\pdd}[2]{\frac{\partial^2{#1}}{\partial{#2}^2}}
\newcommand{\mr}{\mathbb{R}}
\newcommand{\xs}{x^{*}}
\newenvironment{solution}
{\medskip\par\quad\quad\begin{minipage}[c]{.8\textwidth}\gr}{\medskip\end{minipage}}


\pgfmathdeclarefunction{poiss}{1}{%
  \pgfmathparse{(#1^x)*exp(-#1)/(x!)}%
  }

\pgfmathdeclarefunction{gauss}{2}{%
  \pgfmathparse{1/(#2*sqrt(2*pi))*exp(-((x-#1)^2)/(2*#2^2))}%
}

\pgfmathdeclarefunction{expo}{2}{%
  \pgfmathparse{#1*exp(-#1*#2)}%
}

\usetikzlibrary{math}

% https://tex.stackexchange.com/questions/461758/asymmetric-distribution-gauss-curve
\tikzmath{%
  function h1(\x, \lx) { return (9*\lx + 3*((\lx)^2) + ((\lx)^3)/3 + 9); };
  function h2(\x, \lx) { return (3*\lx - ((\lx)^3)/3 + 4); };
  function h3(\x, \lx) { return (9*\lx - 3*((\lx)^2) + ((\lx)^3)/3 + 7); };
  function skewnorm(\x, \l) {
    \x = (\l < 0) ? -\x : \x;
    \l = abs(\l);
    \e = exp(-(\x^2)/2);
    return (\l == 0) ? 1 / sqrt(2 * pi) * \e: (
      (\x < -3/\l) ? 0 : (
      (\x < -1/\l) ? \e / (8 * sqrt(2 * pi)) * h1(\x, \x*\l) : (
      (\x <  1/\l) ? \e / (4 * sqrt(2 * pi)) * h2(\x, \x*\l) : (
      (\x <  3/\l) ? \e / (8 * sqrt(2 * pi)) * h3(\x, \x*\l) : (
      sqrt(2/pi) * \e)))));
  };
}

\def\cdf(#1)(#2)(#3){0.5*(1+(erf((#1-#2)/(#3*sqrt(2)))))}%
% to be used: \cdf(x)(mean)(variance)

\DeclareMathOperator{\CDF}{cdf}
\tikzset{
    declare function={
        normcdf(\x,\m,\s)=1/(1 + exp(-0.07056*((\x-\m)/\s)^3 - 1.5976*(\x-\m)/\s));
    }
}
%%%%%%%%%%%%%%%%%%%%%%%%%%%%%%%%%%%%%%%%%%%%%%%%%%%
%%%%%%%%%%%%%%%%%%%%%%%%%%%%%%%%%%%%%%%%%%%%%%%%%%%

\begin{document}

\lstset{language=C++,
                basicstyle=\tiny\ttfamily,
                keywordstyle=\color{blue}\ttfamily,
                stringstyle=\color{red}\ttfamily,
                commentstyle=\color{gray}\ttfamily,
                morecomment=[l][\color{gray}]{\#}
}


\thispagestyle{empty}


\nin{\LARGE Problem Set 6 {\gr  }}\hfill{\bf Prof. Oke}

\medskip\hrule\medskip

\nin {\small CEE 260/MIE 273: Probability \& Statistics in Civil Engineering
\hfill\textit{ 10.21.2025}}

\bigskip

\nin{\it \textbf{Due Tuesday, October 28, 2025 at 1:00 PM as PDF uploaded on Canvas.}
  If it helps and if possible, you can write your responses directly on this document and upload it instead.
    \textbf{Show as much work as possible in order to get FULL credit.}
    There are 6 problems with a total of 38 points available.
    \textbf{Important:} If you use Python for any probabilistic or statistical computations, briefly write/include the statements you
    used to arrive at your answers. If instead you use probability tables, note this in the respective solution, as well.
}\\


\section*{Problem 1 \textit{(3 points)}}

Respond ``T'' ({\it True})  or  ``F'' (\textit{False}) to the following statements. Use the boxes provided. Each response is worth 1 point.

\begin{enumerate}[\bf (i)]
% \item \hfill
%   \begin{minipage}{.1\linewidth}
%     \framebox(40,40){ \gr }
%   \end{minipage}\quad
%   \begin{minipage}{.85\linewidth}
%     The mean of a Poisson-distributed random variable is equal to its variance.
%    \end{minipage}
  
%   \smallskip
  
% \item \hfill
%   \begin{minipage}{.1\linewidth}
%     \framebox(40,40){\gr  }
%   \end{minipage}\quad
%   \begin{minipage}{.85\linewidth}
%     The mean of a exponentially-distributed random variable is equal to the inverse of its variance.
%    \end{minipage}

%   \smallskip
  
\item \hfill
  \begin{minipage}{.1\linewidth}
    \framebox(40,40){\gr  }
  \end{minipage}\quad
  \begin{minipage}{.85\linewidth}
    The result that a binomial distribution with $n$ large enough (and $p$ reasonably less than 1 or greater than 0) can
    be approximated by a normal distribution with parameters $\mu = np$ and $\sigma^{2} = np(1-p) = npq$ can be derived
    via the Central Limit Theorem.  
  \end{minipage}

  \smallskip
  
\item \hfill
  \begin{minipage}{.1\linewidth}
    \framebox(40,40){\gr  }
  \end{minipage}\quad
  \begin{minipage}{.85\linewidth}
    The household income of the full population of a certain country follows the lognormal distribution. The mean
    household income from a large random sample ($n = 10,000$) will also be lognormally distributed.
   \end{minipage}

  \smallskip

\item \hfill
  \begin{minipage}{.1\linewidth}
    \framebox(40,40){\gr  }
  \end{minipage}\quad
  \begin{minipage}{.85\linewidth}
    The standard error of the mean decreases as the sample size increases.
   \end{minipage}

\end{enumerate}



\eject  

 \section*{Problem 2 \textit{(4 points)}}
{\it Choose the option that best fills in the blank.}
   \begin{enumerate}[\bf(a)]
   \item If an estimator is unbiased, then \underline{\hspace{60ex}}.\pts{1}

     \begin{enumerate}[\bf(i)]
     \item the estimator is equal to the true value.
     \item the estimator is usually close to the true value.
     \item the mean of the estimator is equal to the true value.
     \item the mean of the estimator is usually close to the true value.
     \end{enumerate}

   \item The property \pts{1} of an estimator associated with the bias converging to 0 as the sample size increases (or tends to infinity) is called \underline{\hspace{20ex}}.
     \begin{enumerate}[\bf(i)]
     \item consistency
     \item unbiasedness
     \item efficiency
     \item sufficiency
     \end{enumerate}

   \item The figure below depicts the PDF of a standard normal distribution. What is the value of $z^{*}$ in the figure? \pts{1}

 
        \begin{minipage}{.7\linewidth}
                    \begin{tikzpicture}
              \begin{axis}[no markers, domain=0:10, samples=100,
              axis x line=center,
              axis y line=none,
              xlabel=$z$, ylabel=$f_X(x)$,,
              height=3cm, width=10cm,
              xtick={-1.96,0,1.96},
              xticklabels={$-z^{*}$,$0$,$z^{*}$},
              ymax=.15,
              ytick=\empty,
              x label style={anchor=west},
              y label style={anchor=south},
              enlargelimits=true, clip=false, axis on top,
              tick style={very thick, major tick length=7pt}
              %grid style={line width=.1pt, draw=gray},
              % yticklabel style={
              %   /pgf/number format/fixed,
              %   /pgf/number format/fixed zerofill,
              %   /pgf/number format/precision=2
              % },        
              %   grid = major
              ]
              \addplot [blue, domain=-5:5] {gauss(0,1)};
              \addplot [gray, fill=gray!50, domain=-1.96:1.96] {gauss(0,1)} \closedcycle;
              \addplot [orange,fill=orange,  domain=1.96:5] {gauss(0,1)} \closedcycle;
              \addplot [orange,fill=orange,  domain=-5:-1.96] {gauss(0,1)} \closedcycle;
              \node (d) at (axis cs: 0,.15) {Area: $0.95$};
              %\draw[thick, ->] (d) -- (axis cs: -.6,.05);
              \node (c) at (axis cs: 3.15,.1) {\og Area: $0.025$};
              \draw[thick,->] (c) -- (axis cs: 2.2, 0.005);
              \node (e) at (axis cs: -3.15,.1) {\og Area: $0.025$};
              \draw[thick,->] (e) -- (axis cs: -2.2, 0.005);
              %\draw[thick, |->] (axis cs: 6,-0.025) -- (axis cs: 10,-0.025) node[below,pos=.5] {\small\og Reject $H_0$};
              %\draw[thick, |->] (axis cs: -6,-0.025) -- (axis cs: -10,-0.025) node[below,pos=.5] {\small\og Reject $H_0$};
            \end{axis}
            \end{tikzpicture}


        \end{minipage}
        \begin{minipage}{.25\linewidth}
               \begin{enumerate}[\bf(i)]
     \item 0
     \item 1.65
     \item 1.96
     \item 2.58
     \end{enumerate}


        \end{minipage} 
      \item The figure below depicts the PDF of a standard normal distribution. What is the value of $z_{\alpha/2}$ in the figure? \pts{1}
        
        \begin{minipage}{.7\linewidth}
                  \begin{tikzpicture}
      \begin{axis}[no markers, domain=0:10, samples=100,
      axis x line=center,
      axis y line=none,
      xlabel=$z$, ylabel=$f_X(x)$,,
      height=3cm, width=10cm,
      xtick={-1.44,0,1.44},
      xticklabels={$z_{\alpha/2}$,$0$,$z_{(1-\alpha/2)}$},
      ymax=.15,
      ytick=\empty,
      tick style={very thick, major tick length=7pt},
      x label style={anchor=west},
      y label style={anchor=south},
      enlargelimits=true, clip=false, axis on top
      %grid style={line width=.1pt, draw=gray},
      % yticklabel style={
      %   /pgf/number format/fixed,
      %   /pgf/number format/fixed zerofill,
      %   /pgf/number format/precision=2
      % },        
      %   grid = major
      ]
      \addplot [blue, domain=-5:5] {gauss(0,1)};
      \addplot [gray, fill=gray!50, domain=-1.44:1.44] {gauss(0,1)} \closedcycle;
      \addplot [orange,fill=orange,  domain=1.44:5] {gauss(0,1)} \closedcycle;
      \addplot [orange,fill=orange,  domain=-5:-1.44] {gauss(0,1)} \closedcycle;
      \node (d) at (axis cs: 0,.15) {Area: $0.85$};
      %\draw[thick, ->] (d) -- (axis cs: -.6,.05);
      \node (c) at (axis cs: 3.15,.1) {\og Area: $0.075$};
      \draw[thick,->] (c) -- (axis cs: 1.8, 0.03);
      \node (e) at (axis cs: -3.15,.1) {\og Area: $0.075$};
      \draw[thick,->] (e) -- (axis cs: -1.8, 0.03);
      %\draw[thick, |->] (axis cs: 6,-0.025) -- (axis cs: 10,-0.025) node[below,pos=.5] {\small\og Reject $H_0$};
      %\draw[thick, |->] (axis cs: -6,-0.025) -- (axis cs: -10,-0.025) node[below,pos=.5] {\small\og Reject $H_0$};
    \end{axis}
  \end{tikzpicture}
        \end{minipage}
        \begin{minipage}{.25\linewidth}
          
          \begin{enumerate}[\bf(i)]
            \item 0
     \item $-1.04$
     \item $-1.28$
     \item $-1.44$
     \end{enumerate} 
        \end{minipage}        

 
   \end{enumerate}


\eject

\section*{Problem 3: Standard error of a proportion  \textit{(5 points)}}
An article reports that when each football helmet in a random sample of 37 suspension-type helmets was subjected to a
certain impact test, 24 showed damage. Let $p$ denote the proportion of all helmets of this type that would show damage
when tested in the prescribed manner.

\begin{enumerate}[\bf (a)]
\item  Estimate the value of the proportion $p$. \pts{1}
  \vspace{30ex}
 
\item Compute the standard error of the estimate $\hat p$.\pts{2}
  \vspace{30ex}

\item Using the 68–95–99.7 (empirical) rule, estimate the interval that would contain the sample proportion
  $\hat p$ in 95\% of all possible samples of a similar size. \pts{2}
%  \vspace{30ex}
 \end{enumerate}

\eject

\section*{Problem 4: Point estimate and sampling variability  \textit{(12 points)}}
(Adapted from Ex 5.3 in OS) As part of a quality control process for computer chips, an engineer at a factory
randomly samples 212 chips during a week of production to test the current rate of chips with severe defects.
She finds that 27 of the chips are defective. Match the labels corresponding to the rows in the table below that correctly answers  questions \textbf{(a)} through \textbf{(d)}. Then work out the questions that follow.

\begin{center}
  \small
\begin{tabular}{c l}\toprule
  \bf Label & \bf Answer \\\midrule
  \bf (i)  & All computer chips produced during the week \\
  \bf (ii) & Sample proportion of defective chips, $\hat p$ \\
  \bf (iii)& Standard error of the sample proportion, $SE_{\hat p}$ \\
  \bf (iv)  & Population proportion of defective chips\\\bottomrule
\end{tabular}
\end{center}

\begin{enumerate}[\bf (a)]
\item What population is under consideration in the data set? \boxed{\huge\phantom{BB}}\pts{1}
\item What parameter is being estimated? \boxed{\huge\phantom{BB}} \pts{1}
\item What is the name of point estimate of the parameter? \boxed{\huge\phantom{BB}} \pts{1}
\item What is the name of the statistic that measures the uncertainty of the point estimate? \pts{1} \boxed{\huge\phantom{BB}} 

\item Compute the value from part \textbf{(c)} for this context. \pts{2}
\vspace{10ex}

\item Compute the value from part \textbf{(d)} for this context. \pts{2}
\vspace{10ex}

\item The historical rate of defects is 10\%. \pts{2} Should the engineer be surprised by the observed rate of defects during the current week? Explain briefly.
\vspace{12ex}

\item Suppose the true population value was found to be 10\%. \pts{2} If we use this proportion to recompute the value in part \textbf{(f)} using $p = 0.1$ instead of $\hat{p}$, does the resulting value change much? (Show your work.)

\end{enumerate}

\eject


\section*{Problem 5: Estimating probability from a random sample \textit{(8 points)}}
Among the adults in a large city, 30\% have a college degree. A simple random sample of $n=100$ adults is chosen.
Let $p$ be the proportion of the full adult population that have a college degree. Thus, $p = 0.30$. Now, this proportion
  (of college degree holders) can also be estimated from a random sample. Let the estimate be denoted by $\hat
  p$.

  \begin{enumerate}[\bf (a)]
  \item Show that $\hat p$ satisfies the success-failure condition. \pts{2}
       \vspace{20ex}

       
\item  According \pts{3} to the Central Limit Theorem, $\hat p$ follows a normal distribution. What are the parameters $\mu$ and
  $\sigma$ of this distribution? [1 point for mean, 2 points for variance]

   \vspace{40ex}



\item Estimate the \pts{3} probability that over 40\% of the full adult population have a college degree?

 
\end{enumerate}

\eject


\section*{Problem 6: Confidence interval of a proportion \textit{(6 points)}}
Recall the example of the solar energy poll we examined in class, with $p=0.88$ and $n=1000$.

  \begin{enumerate}[\bf (a)]
  \item Compute the standard error $SE_{\hat p}$. \pts{2}
       \vspace{20ex}

       
\item Find the 95\% confidence interval of $\hat p$. (Express your final answer as an interval.) \pts{2}

   \vspace{40ex}


  
\item Sketch the distribution of $\hat p$ and indicate the confidence interval found in part \textbf{(b)} on the plot. \pts{2}

   \vspace{60ex}
 
\end{enumerate}

 
\end{document}

%%% Local Variables:
%%% mode: latex
%%% TeX-master: t
%%% End:
