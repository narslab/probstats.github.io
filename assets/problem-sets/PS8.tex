\documentclass[11pt,twoside]{article}
\usepackage{etex}

\raggedbottom

%geometry (sets margin) and other useful packages
\usepackage{geometry}
\geometry{top=1in, left=1in,right=1in,bottom=1in}
 \usepackage{graphicx,booktabs,calc}
 
\usepackage{listings}
\usepackage{float}


% Marginpar width
%Marginpar width
\newcommand{\pts}[1]{\marginpar{ \small\hspace{0pt} \textit{[#1]} } } 
\setlength{\marginparwidth}{.5in}
%\reversemarginpar
%\setlength{\marginparsep}{.02in}

 
%\usepackage{cmbright}lstinputlisting
%\usepackage[T1]{pbsi}


\usepackage{chngcntr,mathtools}
\counterwithin{figure}{section}
\numberwithin{equation}{section}

%\usepackage{listings}

%AMS-TeX packages
\usepackage{amssymb,amsmath,amsthm} 
\usepackage{bm}
\usepackage[mathscr]{eucal}
\usepackage{colortbl}
\usepackage{color}


\usepackage{subfig,hyperref,enumerate,polynom,polynomial}
\usepackage{multirow,minitoc,fancybox,array,multicol}

\definecolor{slblue}{rgb}{0,.3,.62}
\hypersetup{
    colorlinks,%
    citecolor=blue,%
    filecolor=blue,%
    linkcolor=blue,
    urlcolor=slblue
}

%%%TIKZ
\usepackage{tikz}

\usepackage{pgfplots}
\pgfplotsset{compat=newest}

\usetikzlibrary{arrows,shapes,positioning}
\usetikzlibrary{decorations.markings}
\usetikzlibrary{shadows}
\usetikzlibrary{patterns}
%\usetikzlibrary{circuits.ee.IEC}
\usetikzlibrary{decorations.text}
% For Sagnac Picture
\usetikzlibrary{%
    decorations.pathreplacing,%
    decorations.pathmorphing%
}

\tikzstyle arrowstyle=[black,scale=2]
\tikzstyle directed=[postaction={decorate,decoration={markings,
    mark=at position .65 with {\arrow[arrowstyle]{stealth}}}}]
\tikzstyle reverse directed=[postaction={decorate,decoration={markings,
    mark=at position .65 with {\arrowreversed[arrowstyle]{stealth};}}}]
\tikzstyle dir=[postaction={decorate,decoration={markings,
    mark=at position .98 with {\arrow[arrowstyle]{latex}}}}]
\tikzstyle rev dir=[postaction={decorate,decoration={markings,
    mark=at position .98 with {\arrowreversed[arrowstyle]{latex};}}}]

\usepackage{ctable}

%
%Redefining sections as problems
%
\makeatletter
\newenvironment{exercise}{\@startsection 
	{section}
	{1}
	{-.2em}
	{-3.5ex plus -1ex minus -.2ex}
    	{1.3ex plus .2ex}
    	{\pagebreak[3]%forces pagebreak when space is small; use \eject for better results
	\large\bf\noindent{Exercise 1.\hspace{-1.5ex} }
	}
	}
	%{\vspace{1ex}\begin{center} \rule{0.3\linewidth}{.3pt}\end{center}}
	%\begin{center}\large\bf \ldots\ldots\ldots\end{center}}
\makeatother

%
%Fancy-header package to modify header/page numbering 
%
\usepackage{fancyhdr}
\pagestyle{fancy}
%\addtolength{\headwidth}{\marginparsep} %these change header-rule width
%\addtolength{\headwidth}{\marginparwidth}
%\fancyheadoffset{30pt}
%\fancyfootoffset{30pt}
\fancyhead[LO,RE]{\small Oke}
\fancyhead[RO,LE]{\small Page \thepage} 
\fancyfoot[RO,LE]{\small PS 8} 
\fancyfoot[LO,RE]{\small \scshape CEE 260/MIE 273} 
\cfoot{} 
\renewcommand{\headrulewidth}{0.1pt} 
\renewcommand{\footrulewidth}{0.1pt}
%\setlength\voffset{-0.25in}
%\setlength\textheight{648pt}


\usepackage{paralist}

\newcommand{\osn}{\oldstylenums}
\newcommand{\lt}{\left}
\newcommand{\rt}{\right}
\newcommand{\pt}{\phantom}
\newcommand{\tf}{\therefore}
\newcommand{\?}{\stackrel{?}{=}}
\newcommand{\fr}{\frac}
\newcommand{\dfr}{\dfrac}
\newcommand{\ul}{\underline}
\newcommand{\tn}{\tabularnewline}
\newcommand{\nl}{\newline}
\newcommand\relph[1]{\mathrel{\phantom{#1}}}
\newcommand{\cm}{\checkmark}
\newcommand{\ol}{\overline}
\newcommand{\rd}{\color{red}}
\newcommand{\bl}{\color{blue}}
\newcommand{\pl}{\color{purple}}
\newcommand{\og}{\color{orange!90!black}}
\newcommand{\gr}{\color{green!40!black}}
\newcommand{\nin}{\noindent}
\newcommand{\la}{\lambda}
\renewcommand{\th}{\theta}
\newcommand*\circled[1]{\tikz[baseline=(char.base)]{
            \node[shape=circle,draw,thick,inner sep=1pt] (char) {\small #1};}}

\newcommand{\bc}{\begin{compactenum}[\quad--]}
\newcommand{\ec}{\end{compactenum}}

\newcommand{\n}{\\[2mm]}
%% GREEK LETTERS
\newcommand{\al}{\alpha}
\newcommand{\gam}{\gamma}
\newcommand{\eps}{\epsilon}
\newcommand{\sig}{\sigma}

\newcommand{\p}{\partial}
\newcommand{\pd}[2]{\frac{\partial{#1}}{\partial{#2}}}
\newcommand{\dpd}[2]{\dfrac{\partial{#1}}{\partial{#2}}}
\newcommand{\pdd}[2]{\frac{\partial^2{#1}}{\partial{#2}^2}}
\newcommand{\mr}{\mathbb{R}}
\newcommand{\xs}{x^{*}}
\newenvironment{solution}
{\medskip\par\quad\quad\begin{minipage}[c]{.8\textwidth}\gr}{\medskip\end{minipage}}


\pgfmathdeclarefunction{poiss}{1}{%
  \pgfmathparse{(#1^x)*exp(-#1)/(x!)}%
  }

\pgfmathdeclarefunction{gauss}{2}{%
  \pgfmathparse{1/(#2*sqrt(2*pi))*exp(-((x-#1)^2)/(2*#2^2))}%
}

\pgfmathdeclarefunction{expo}{2}{%
  \pgfmathparse{#1*exp(-#1*#2)}%
}

\usetikzlibrary{math}

% https://tex.stackexchange.com/questions/461758/asymmetric-distribution-gauss-curve
\tikzmath{%
  function h1(\x, \lx) { return (9*\lx + 3*((\lx)^2) + ((\lx)^3)/3 + 9); };
  function h2(\x, \lx) { return (3*\lx - ((\lx)^3)/3 + 4); };
  function h3(\x, \lx) { return (9*\lx - 3*((\lx)^2) + ((\lx)^3)/3 + 7); };
  function skewnorm(\x, \l) {
    \x = (\l < 0) ? -\x : \x;
    \l = abs(\l);
    \e = exp(-(\x^2)/2);
    return (\l == 0) ? 1 / sqrt(2 * pi) * \e: (
      (\x < -3/\l) ? 0 : (
      (\x < -1/\l) ? \e / (8 * sqrt(2 * pi)) * h1(\x, \x*\l) : (
      (\x <  1/\l) ? \e / (4 * sqrt(2 * pi)) * h2(\x, \x*\l) : (
      (\x <  3/\l) ? \e / (8 * sqrt(2 * pi)) * h3(\x, \x*\l) : (
      sqrt(2/pi) * \e)))));
  };
}

\def\cdf(#1)(#2)(#3){0.5*(1+(erf((#1-#2)/(#3*sqrt(2)))))}%
% to be used: \cdf(x)(mean)(variance)

\DeclareMathOperator{\CDF}{cdf}
\tikzset{
    declare function={
        normcdf(\x,\m,\s)=1/(1 + exp(-0.07056*((\x-\m)/\s)^3 - 1.5976*(\x-\m)/\s));
    }
}
%%%%%%%%%%%%%%%%%%%%%%%%%%%%%%%%%%%%%%%%%%%%%%%%%%%
%%%%%%%%%%%%%%%%%%%%%%%%%%%%%%%%%%%%%%%%%%%%%%%%%%%
\newcommand{\nmfr}[3]{\Phi\left(\frac{{#1} - {#2}}{#3}\right)}


\begin{document}

\lstset{language=C++,
                basicstyle=\tiny\ttfamily,
                keywordstyle=\color{blue}\ttfamily,
                stringstyle=\color{red}\ttfamily,
                commentstyle=\color{gray}\ttfamily,
                morecomment=[l][\color{gray}]{\#}
}


\thispagestyle{empty}


\nin{\LARGE Problem Set 8 {\gr }}\hfill{\bf Prof. Oke}

\medskip\hrule\medskip

\nin {\small CEE 260/MIE 273: Probability \& Statistics in Civil Engineering
\hfill\textit{ 11.19.2025}}

\bigskip

\nin{\it {Due Friday, November 26, 2025 at 11:59 PM as PDF uploaded on Canvas.}
  %If it helps and if possible, you can write your responses directly on this document and upload it instead.
    \textbf{Show as much work as possible in order to get FULL credit.}
    There are 10 problems with a total of 63 points available.
 %   \textbf{Important:} If you use MATLAB for any probability computations, briefly write/include the statements you
   % used to arrive at your answers. If instead you use probability tables, note this in the respective solution, as well.
}

 




 


\section*{Problem 1 \quad {\it Chi-square hypothesis test conclusions (9 points)}      }
What conclusions would be appropriate for an upper-tailed chi-square test in each of the following situations (where $\chi^2$ is the test statistic)? (In each case, show explicitly how you compute and compare the critical value $\chi^2_{1-\alpha, \nu}$. Then circle the correct option ({\bf i.}) or ({\bf ii.}).)

\begin{enumerate}[\bf (a)]
	\item $\alpha = 0.01$, $k = 3$, $\chi^2 = 8.54$ \pts{3}
	\vspace{20ex}
	\begin{enumerate}[\bf i.]
		\item Fail to reject $H_0$
		\item Reject $H_0$
	\end{enumerate}
	\item $\alpha = 0.10$, $k = 2$, $\chi^2 = 4.36$ \pts{3}
		\vspace{20ex}
	\begin{enumerate}[\bf i.]
		\item Fail to reject $H_0$
		\item Reject $H_0$
	\end{enumerate}
	\item $\alpha = 0.01$, $k = 6$, $\chi^2 = 10.20$ \pts{3}
		\vspace{20ex}
	\begin{enumerate}[\bf i.]
		\item Fail to reject $H_0$
		\item Reject $H_0$
	\end{enumerate}
\end{enumerate}


\section*{Problem 2\quad {\it $p$-value of chi-square statistic (6 points)}      }
Calculate the $p$-value for an upper-tailed chi-square test in each of the following situations (show the Python/calculator function you use in each case):
\begin{enumerate}[\bf (a)]
	\item $\chi^2 = 13.0$, $k = 6$\pts{2}
	\vspace{30ex}
	\item  $\chi^2 = 18.0$, $k = 9$\pts{2}
	\vspace{30ex}
	\item $\chi^2 = 5.0$, $k =4$\pts{2}
	\vspace{30ex}
\end{enumerate}

\eject
 \section*{Problem 3 \quad {\it  Upper confidence bound (3 points)}      }%
\begin{minipage}{.6\textwidth}
The charge-to-tap time (min) for a carbon steel in one type of open hearth furnace was determined for each heat\footnote{A ``heat'' describes each batch in an open-hearth process for steel production.
  Read this article for more information: \url{https://en.wikipedia.org/wiki/Open_hearth_furnace}} in a sample of size 46,
resulting in a sample mean time of 382.1 and a population standard deviation of 31.5.
Calculate a 95\% upper confidence bound for true average charge-to-tap time.

{\small \it (RIGHT:) An open hearth furnace being tapped at a Swedish steel mill. Source: \url{https://en.wikipedia.org/wiki/File:Tappning_av_martinugn.jpg}}
\end{minipage}\quad\quad
\begin{minipage}{.35\textwidth}
    \includegraphics[width=.7\textwidth]{tapping.jpg}
\end{minipage}

\vfill


\vspace{55ex}

\eject

\section*{Problem 4 \quad {\it Confidence Intervals and Sample Size (7 points)}      }

\begin{minipage}{.5\textwidth}
The article ``Evaluating Tunnel Kiln Performance'' ({\it Amer.\ Ceramic Soc.\ Bull.}, Aug 1997: 59--63) gave the following summary information for fracture
strengths (MPa) of $n=169$ ceramic bars fired in a particular kiln: $\ol{x} = 89.10, s = 3.73$.\\

\noindent {\small\it  (RIGHT:) A tunnel kiln. Source:  \url{https://blog.therseruk.com/hubfs/Tunnel%20Kiln%20for%20Refractories%20in%20UK%2017-2.jpg}}
\end{minipage}\quad \quad
\begin{minipage}{.4\textwidth}
	\begin{figure}[H]
		\includegraphics[width=.9\textwidth]{tunnel-kiln.jpg}

	\end{figure}

\end{minipage}

\begin{enumerate}[\bf (a)]
\item Calculate a [two-sided] confidence interval for true average \pts{4} fracture strength using a confidence level of 95\%.
  Does it appear that true average fracture strength has been precisely estimated?
  \vspace{40ex}
  
  
\item Suppose the \pts{3} investigators had believed a priori that the population standard deviation was about 4 MPa.
  Based on this supposition, how large a sample would have been required to estimate $\mu$ to within 0.5 MPa with 95\% confidence?
  \vspace{50ex}
 
\end{enumerate}

\eject
\section*{Problem 5 \quad {\it  Confidence intervals (5 points)}      }
A 95\% confidence interval for a population mean, $\mu$, is given as $(18.985,
21.015)$. This confidence interval is based on a simple random sample of 36 observations.
Assume that all conditions necessary for inference are satisfied. Using the $t$-distribution, calculate the
\begin{enumerate}[(a)]

  \item Margin of error \pts{1}
  \vspace{15ex}
\item Sample mean \pts{1}
\vspace{15ex}
 
\item Sample standard deviation \pts{3}
\vspace{40ex}
 
\end{enumerate}
\eject

\section*{Problem 6 \quad {\it Hypothesis testing (4 points)}}
In the following hypothesis tests, decide whether to ``Reject $H_{0}$'' or ``Fail to reject $H_{0}$'' by comparing the $Z$ or $T$ scores ($z$ or $t$, respectively) to the critical values. (Critical regions in orange.)  
\begin{enumerate}[(a)]
\item  $H_{0}:\mu = \mu_{0}; H_{1}: \mu > \mu_{0}$
 
    % \gr $t_{(1 - \alpha)} = 3.733 \implies  1 - \alpha = F_T(3.733,\nu = 15) = 0.999$.\\
    % The confidence level is therefore $\boxed{99.9\%.}$

    \begin{figure}[h!]
      \centering
      \begin{tikzpicture}[
        declare function={gamma(\z)=
          2.506628274631*sqrt(1/\z)+ 0.20888568*(1/\z)^(1.5)+ 0.00870357*(1/\z)^(2.5)
          - (174.2106599*(1/\z)^(3.5))/25920- (715.6423511*(1/\z)^(4.5))/1244160)*exp((-ln(1/\z)-1)*\z;},
        declare function={student(\x,\n)= gamma((\n+1)/2.)/(sqrt(\n*pi) *gamma(\n/2.)) *((1+(\x*\x)/\n)^(-(\n+1)/2.));}
        ]
        \begin{axis}[no markers, domain=-5:5, samples=100,
          axis x line=center,
          axis y line=none,
          xlabel=$T$, ylabel=$f_X(x)$,
          every x tick/.style={color=black, thick},          
          height=3cm, width=15cm,
          xtick={0,2.249, 3},
          xticklabels={0,$t_{(1-\alpha)}$, $t$},
          ymax=.15,
          ytick=\empty,
          x label style={anchor=west},
          y label style={anchor=south},
          enlargelimits=true, clip=false, axis on top
          % grid style={line width=.1pt, draw=gray},
          % yticklabel style={
          % /pgf/number format/fixed,
          % /pgf/number format/fixed zerofill,
          % /pgf/number format/precision=2
          % },        
          %   grid = major
          ]
          \addplot [blue, domain=-5:5] {student(x,15)};
          \addplot [gray, fill=gray!50, domain=-5:2.249] {student(x,15)} \closedcycle;
          \addplot [orange,fill=orange,  domain=2.249:5] {student(x,15)} \closedcycle;
          % \addplot [orange,fill=orange,  domain=1.697:5] {student(x,8)} \closedcycle;
         % \node (d) at (axis cs: 0,.1) {Area: .999};
         % \node (c) at (axis cs: 3.5,.08) {\og Area: .001};
         % \draw[thick,->] (c) -- (axis cs: 3.5, 0.01);
          % \node (c) at (axis cs: -3.2,.08) {\og Area: .05};
          % \draw[thick,->] (c) -- (axis cs: -2, 0.03);
        \end{axis}
      \end{tikzpicture}
      % \caption{Hypothesis test for Problem 6.1}
      \label{fig:6.1}
    \end{figure}
 % \end{quote}


 
    
\item  $H_{0}:\mu = \mu_{0}; H_{1}: \mu < \mu_{0}$
    %  \begin{quote} 
    % \gr $t_{\alpha} = -2.5 \implies  1 - \alpha = F_T(-2.5, \nu = 24) = 0.9902$.\\
    % The confidence level is therefore $\boxed{99\%.}$

    \begin{figure}[h!]
      \centering
      \begin{tikzpicture}[
        declare function={gamma(\z)=
          2.506628274631*sqrt(1/\z)+ 0.20888568*(1/\z)^(1.5)+ 0.00870357*(1/\z)^(2.5)
          - (174.2106599*(1/\z)^(3.5))/25920- (715.6423511*(1/\z)^(4.5))/1244160)*exp((-ln(1/\z)-1)*\z;},
        declare function={student(\x,\n)= gamma((\n+1)/2.)/(sqrt(\n*pi) *gamma(\n/2.)) *((1+(\x*\x)/\n)^(-(\n+1)/2.));}
        ]
        \begin{axis}[no markers, domain=-5:5, samples=100,
          axis x line=center,
          axis y line=none,
          xlabel=$T$, ylabel=$f_X(x)$,
          height=3cm, width=15cm,
          xtick={-2.5, -2.2, 0},
          xticklabels={$t_{\alpha}$,$t$, 0},
          every x tick/.style={color=black, thick},         
          ymax=.15,
          ytick=\empty,
          x label style={anchor=west},
          y label style={anchor=south},
          enlargelimits=true, clip=false, axis on top
          % grid style={line width=.1pt, draw=gray},
          % yticklabel style={
          % /pgf/number format/fixed,
          % /pgf/number format/fixed zerofill,
          % /pgf/number format/precision=2
          % },        
          %   grid = major
          ]
          \addplot [blue, domain=-5:5] {student(x,24)};
          \addplot [gray, fill=gray!50, domain=-2.5:5] {student(x,24)} \closedcycle;
          \addplot [orange,fill=orange,  domain=-5:-2.5] {student(x,24)} \closedcycle;
          %\addplot [orange,fill=orange,  domain=1.697:5] {student(x,8)} \closedcycle;
          %\node (d) at (axis cs: 0,.1) {Area: .99};
       %   \node (c) at (axis cs: -3.2,.08) {\og Area: .01};
          %\draw[thick,->] (c) -- (axis cs: -2.7, 0.005);
          % \node (c) at (axis cs: -3.2,.08) {\og Area: .05};
          % \draw[thick,->] (c) -- (axis cs: -2, 0.03);
        \end{axis}
      \end{tikzpicture}
      % \caption{Hypothesis test for Problem 6.1}
      \label{fig:6.1}
    \end{figure}
 % \end{quote}

    %\eject
    
 
     
\item  $H_{0}:\mu = \mu_{0}; H_{1}: \mu \ne \mu_{0}$
 
    \begin{figure}[h!]
      \centering
      \begin{tikzpicture}[
        declare function={gamma(\z)=
          2.506628274631*sqrt(1/\z)+ 0.20888568*(1/\z)^(1.5)+ 0.00870357*(1/\z)^(2.5)
          - (174.2106599*(1/\z)^(3.5))/25920- (715.6423511*(1/\z)^(4.5))/1244160)*exp((-ln(1/\z)-1)*\z;},
        declare function={student(\x,\n)= gamma((\n+1)/2.)/(sqrt(\n*pi) *gamma(\n/2.)) *((1+(\x*\x)/\n)^(-(\n+1)/2.));}
        ]
        \begin{axis}[no markers, domain=-5:5, samples=100,
          axis x line=center,
          axis y line=none,
          xlabel=$T$, ylabel=$f_X(x)$,
          height=3cm, width=14cm,
          every x tick/.style={color=black, thick},
          xtick={-1.96,-0.5, 0,1.96},
          xticklabels={$t_{\fr\alpha2}$,$t$,0,$t_{(1-\fr\alpha2)}$},
          ymax=.15,
          %ytick=\empty,
          x label style={anchor=west},
          y label style={anchor=south},
          enlargelimits=true, clip=false, axis on top
          ]
          \addplot [blue, domain=-5:5] {student(x,20)};
          \addplot [gray, fill=gray!50, domain=-1.96:1.96] {student(x,20)} \closedcycle;
          \addplot [orange,fill=orange,  domain=-5:-1.96] {student(x,20)} \closedcycle;
          \addplot [orange,fill=orange,  domain=1.96:5] {student(x,20)} \closedcycle;
          %\node (d) at (axis cs: 0,.1) {Area: .90};
          %\node (c) at (axis cs: 3.2,.08) {\og Area: .05};
          %\draw[thick,->] (c) -- (axis cs: 2, 0.03);
          %\node (c) at (axis cs: -3.2,.08) {\og Area: .05};
          %\draw[thick,->] (c) -- (axis cs: -2, 0.03);
        \end{axis}
      \end{tikzpicture}
      % \caption{Hypothesis test for Problem 6.1}
      \label{fig:6.1}
    \end{figure}


   
    
  \item  $H_{0}:\mu = \mu_{0}; H_{1}: \mu \ne \mu_{0}$
 
    \begin{figure}[h!]
      \centering
      \begin{tikzpicture}[
        declare function={gamma(\z)=
          2.506628274631*sqrt(1/\z)+ 0.20888568*(1/\z)^(1.5)+ 0.00870357*(1/\z)^(2.5)
          - (174.2106599*(1/\z)^(3.5))/25920- (715.6423511*(1/\z)^(4.5))/1244160)*exp((-ln(1/\z)-1)*\z;},
        declare function={student(\x,\n)= gamma((\n+1)/2.)/(sqrt(\n*pi) *gamma(\n/2.)) *((1+(\x*\x)/\n)^(-(\n+1)/2.));}
        ]
        \begin{axis}[no markers, domain=-5:5, samples=100,
          axis x line=center,
          axis y line=none,
          xlabel=$Z$, ylabel=$f_X(x)$,
          height=3cm, width=14cm,
          every x tick/.style={color=black, thick},
          xtick={-1.645, 0,1.645, 3},
          xticklabels={$z_{\fr\alpha2}$,0,$z_{(1-\fr\alpha2)}$, $z$},
          ymax=.15,
          ytick=\empty,
          x label style={anchor=west},
          y label style={anchor=south},
          enlargelimits=true, clip=false, axis on top
          ]
          \addplot [blue, domain=-5:5] {gauss(0,1)};
          \addplot [gray, fill=gray!50, domain=-1.645:1.645] {gauss(0,1)} \closedcycle;
          \addplot [orange,fill=orange,  domain=-5:-1.645] {gauss(0,1)} \closedcycle;
          \addplot [orange,fill=orange,  domain=1.645:5] {gauss(0,1)} \closedcycle;
          %\node (d) at (axis cs: 0,.1) {Area: .90};
          %\node (c) at (axis cs: 3.2,.08) {\og Area: .05};
          %\draw[thick,->] (c) -- (axis cs: 2, 0.03);
          %\node (c) at (axis cs: -3.2,.08) {\og Area: .05};
          %\draw[thick,->] (c) -- (axis cs: -2, 0.03);
        \end{axis}
      \end{tikzpicture}
      % \caption{Hypothesis test for Problem 6.1}
      \label{fig:6.1}
    \end{figure}
    
 % \end{quote}
\end{enumerate}
 


\eject


 \section*{Problem 7 \quad {\it Identifying significance levels with $Z$-score (6 points)}      }
Let the test statistic $Z$ have a standard normal distribution when $H_0$ is true.
Find the significance level for each of the following situations (show your work and/or calculator/Matlab/Python functions):

\begin{enumerate}[\bf (a)]
\item $H_1: \mu > \mu_0$, critical region: $z \ge 1.88$.
  \vspace{36ex}

\item $H_1: \mu < \mu_0$, critical region: $z \le -2.75$.
  \vspace{36ex}

\item $H_1: \mu \ne \mu_0$, critical region: $z \ge 2.88$ or $z \le -2.88$.
 \vspace{36ex}

\end{enumerate}

\eject
\section*{Problem 8 \quad {\it Identifying significance levels with $T$-score (6 points) }      }
Let the test statistic $T$ have a $t$ distribution when $H_0$ is true.
Find the significance level for each of the following situations:
\begin{enumerate}[\bf (a)]
\item $H_1: \mu > \mu_0$, d.o.f. $=15$, rejection region: $t \ge 3.733$.
\vspace{36ex}

\item $H_1: \mu < \mu_0$, d.o.f. $=24$, rejection region: $t \le -2.500$.
\vspace{36ex}

\item $H_1: \mu \ne \mu_0$, d.o.f. $=31$, rejection region: $t \ge 1.697$ or $t \le -1.697$.
\vspace{36ex}

\end{enumerate}



\eject
\section*{Problem 9 \quad {\it p-values (7 points)}      }
\begin{enumerate}[\bf (a)]
\item Pairs of $p$-values and significance levels $\alpha$ are given.
For each pair, state whether the observed $p$-value would lead to rejection of $H_0$ at the given significance level:
\begin{enumerate}[ \bf (i)]
\item $p$-value = 0.084; $\alpha = 0.05$ \pts{1}
  \vspace{10ex}
  
\item  $p$-value $= 3.2 \times 10^{-5}$; $\alpha = 0.001$  \pts{1}
  \vspace{10ex}
  
\item $p$-value = 0.039; $\alpha = 0.01$  \pts{1}
  \vspace{10ex}
  
\end{enumerate}



\item Let $\mu$  denote the mean reaction time to a certain stimulus. For a large-sample test of $H_0: \mu = 5$, 
  versus $H_1: \mu > 5$, find the $p$-value associated with the following test statistic values
  \begin{enumerate}[\bf (i)]
  \item $z = 1.42$ \pts{2}
    \vspace{30ex}
    
    \item $z = -0.11$ \pts{2}
  \end{enumerate}
 
    \vspace{25ex}
\end{enumerate}


\eject

\section*{Problem 10 \quad {\it Two-tailed hypothesis test (10 points)}      }
The melting point of each of 16 samples of a certain brand of hydrogenated vegetable oil was determined, resulting in $\ol{x} = 94.32$F and $s = 1.20$.
Test $H_0: \mu = 95$ versus $H_1: \mu \ne 95$ using a two-tailed level 0.01 $t$ test.
  \begin{enumerate}[\bf (a)]
    \item  State the hypotheses \pts{2}
  \vspace{10ex}
  \item Compute the test statistic and critical value(s) OR $p$-value \pts{3}
  \vspace{35ex}
  \item Explicitly compare test statistic to the critical value (or p-value to $\alpha$); sketch a supporting diagram of the distribution \pts{3}
  \vspace{40ex}
  \item State the outcome of the hypothesis test and write a concluding statement \pts{2}
  \end{enumerate}
 

% \eject

% \section*{Problem 10 \quad {\it One-tailed hypothesis test (8 points)}      }
% The times of first sprinkler activation for a series of tests with fire prevention sprinkler systems using an aqueous film-forming foam were (in sec):
% \[27, 41, 22, 27, 23, 35, 30, 33, 24, 27, 28, 22, 24\] The system has been designed so that true average activation time
% is at most 25 sec under such conditions.  Test the relevant hypotheses at significance level 0.05 using the $p$-value
% approach to determine if the data strongly contradict the validity of this design specification.
% \begin{quote}
% {\it NOTE: To get full credit for this problem, make sure you do the following:
%   \begin{compactitem}
%   \item State the hypotheses (there are two) \pts{1}
%   \item Find the sample mean and standard deviation \pts{2}
%   \item Find the test statistic (hint: $T$-score) \pts{1}
%   \item Find the p-value \pts{1}
%   \item Compare the appropriate values \pts{1}
%   \item Clearly state the outcome from your hypothesis test \pts{1}
%   \item Write a final concluding statement in response to the question \pts{1}
%   \end{compactitem}
%   }
% \end{quote}

\end{document}

%%% Local Variables:
%%% mode: latex
%%% TeX-master: t
%%% End:
