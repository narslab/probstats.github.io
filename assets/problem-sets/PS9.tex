\documentclass[11pt,twoside]{article}
\usepackage{etex}

\raggedbottom

%geometry (sets margin) and other useful packages
\usepackage{geometry}
\geometry{top=1in, left=1in,right=1in,bottom=1in}
 \usepackage{graphicx,booktabs,calc}
 
\usepackage{listings}


% Marginpar width
%Marginpar width
\newcommand{\pts}[1]{\marginpar{ \small\hspace{0pt} \textit{[#1]} } } 
\setlength{\marginparwidth}{.5in}
%\reversemarginpar
%\setlength{\marginparsep}{.02in}

 
%\usepackage{cmbright}lstinputlisting
%\usepackage[T1]{pbsi}


\usepackage{chngcntr,mathtools}
\counterwithin{figure}{section}
\numberwithin{equation}{section}

%\usepackage{listings}

%AMS-TeX packages
\usepackage{amssymb,amsmath,amsthm} 
\usepackage{bm}
\usepackage[mathscr]{eucal}
\usepackage{colortbl}
\usepackage{color}


\usepackage{subfig,hyperref,enumerate,polynom,polynomial}
\usepackage{multirow,minitoc,fancybox,array,multicol}

\definecolor{slblue}{rgb}{0,.3,.62}
\hypersetup{
    colorlinks,%
    citecolor=blue,%
    filecolor=blue,%
    linkcolor=blue,
    urlcolor=slblue
}

%%%TIKZ
\usepackage{tikz}

\usepackage{pgfplots}
\pgfplotsset{compat=newest}

\usetikzlibrary{arrows,shapes,positioning}
\usetikzlibrary{decorations.markings}
\usetikzlibrary{shadows}
\usetikzlibrary{patterns}
%\usetikzlibrary{circuits.ee.IEC}
\usetikzlibrary{decorations.text}
% For Sagnac Picture
\usetikzlibrary{%
    decorations.pathreplacing,%
    decorations.pathmorphing%
}

\tikzstyle arrowstyle=[black,scale=2]
\tikzstyle directed=[postaction={decorate,decoration={markings,
    mark=at position .65 with {\arrow[arrowstyle]{stealth}}}}]
\tikzstyle reverse directed=[postaction={decorate,decoration={markings,
    mark=at position .65 with {\arrowreversed[arrowstyle]{stealth};}}}]
\tikzstyle dir=[postaction={decorate,decoration={markings,
    mark=at position .98 with {\arrow[arrowstyle]{latex}}}}]
\tikzstyle rev dir=[postaction={decorate,decoration={markings,
    mark=at position .98 with {\arrowreversed[arrowstyle]{latex};}}}]

\usepackage{ctable}

%
%Redefining sections as problems
%
\makeatletter
\newenvironment{exercise}{\@startsection 
	{section}
	{1}
	{-.2em}
	{-3.5ex plus -1ex minus -.2ex}
    	{1.3ex plus .2ex}
    	{\pagebreak[3]%forces pagebreak when space is small; use \eject for better results
	\large\bf\noindent{Exercise 1.\hspace{-1.5ex} }
	}
	}
	%{\vspace{1ex}\begin{center} \rule{0.3\linewidth}{.3pt}\end{center}}
	%\begin{center}\large\bf \ldots\ldots\ldots\end{center}}
\makeatother

%
%Fancy-header package to modify header/page numbering 
%
\usepackage{fancyhdr}
\pagestyle{fancy}
%\addtolength{\headwidth}{\marginparsep} %these change header-rule width
%\addtolength{\headwidth}{\marginparwidth}
%\fancyheadoffset{30pt}
%\fancyfootoffset{30pt}
\fancyhead[LO,RE]{\small Oke}
\fancyhead[RO,LE]{\small Page \thepage} 
\fancyfoot[RO,LE]{\small PS 8} 
\fancyfoot[LO,RE]{\small \scshape CEE 260/MIE 273} 
\cfoot{} 
\renewcommand{\headrulewidth}{0.1pt} 
\renewcommand{\footrulewidth}{0.1pt}
%\setlength\voffset{-0.25in}
%\setlength\textheight{648pt}


\usepackage{paralist}

\newcommand{\osn}{\oldstylenums}
\newcommand{\lt}{\left}
\newcommand{\rt}{\right}
\newcommand{\pt}{\phantom}
\newcommand{\tf}{\therefore}
\newcommand{\?}{\stackrel{?}{=}}
\newcommand{\fr}{\frac}
\newcommand{\dfr}{\dfrac}
\newcommand{\ul}{\underline}
\newcommand{\tn}{\tabularnewline}
\newcommand{\nl}{\newline}
\newcommand\relph[1]{\mathrel{\phantom{#1}}}
\newcommand{\cm}{\checkmark}
\newcommand{\ol}{\overline}
\newcommand{\rd}{\color{red}}
\newcommand{\bl}{\color{blue}}
\newcommand{\pl}{\color{purple}}
\newcommand{\og}{\color{orange!90!black}}
\newcommand{\gr}{\color{green!40!black}}
\newcommand{\nin}{\noindent}
\newcommand{\la}{\lambda}
\renewcommand{\th}{\theta}
\newcommand*\circled[1]{\tikz[baseline=(char.base)]{
            \node[shape=circle,draw,thick,inner sep=1pt] (char) {\small #1};}}

\newcommand{\bc}{\begin{compactenum}[\quad--]}
\newcommand{\ec}{\end{compactenum}}

\newcommand{\n}{\\[2mm]}
%% GREEK LETTERS
\newcommand{\al}{\alpha}
\newcommand{\gam}{\gamma}
\newcommand{\eps}{\epsilon}
\newcommand{\sig}{\sigma}

\newcommand{\p}{\partial}
\newcommand{\pd}[2]{\frac{\partial{#1}}{\partial{#2}}}
\newcommand{\dpd}[2]{\dfrac{\partial{#1}}{\partial{#2}}}
\newcommand{\pdd}[2]{\frac{\partial^2{#1}}{\partial{#2}^2}}
\newcommand{\mr}{\mathbb{R}}
\newcommand{\xs}{x^{*}}
\newenvironment{solution}
{\medskip\par\quad\quad\begin{minipage}[c]{.8\textwidth}\gr}{\medskip\end{minipage}}


\pgfmathdeclarefunction{poiss}{1}{%
  \pgfmathparse{(#1^x)*exp(-#1)/(x!)}%
  }

\pgfmathdeclarefunction{gauss}{2}{%
  \pgfmathparse{1/(#2*sqrt(2*pi))*exp(-((x-#1)^2)/(2*#2^2))}%
}

\pgfmathdeclarefunction{expo}{2}{%
  \pgfmathparse{#1*exp(-#1*#2)}%
}

\usetikzlibrary{math}

% https://tex.stackexchange.com/questions/461758/asymmetric-distribution-gauss-curve
\tikzmath{%
  function h1(\x, \lx) { return (9*\lx + 3*((\lx)^2) + ((\lx)^3)/3 + 9); };
  function h2(\x, \lx) { return (3*\lx - ((\lx)^3)/3 + 4); };
  function h3(\x, \lx) { return (9*\lx - 3*((\lx)^2) + ((\lx)^3)/3 + 7); };
  function skewnorm(\x, \l) {
    \x = (\l < 0) ? -\x : \x;
    \l = abs(\l);
    \e = exp(-(\x^2)/2);
    return (\l == 0) ? 1 / sqrt(2 * pi) * \e: (
      (\x < -3/\l) ? 0 : (
      (\x < -1/\l) ? \e / (8 * sqrt(2 * pi)) * h1(\x, \x*\l) : (
      (\x <  1/\l) ? \e / (4 * sqrt(2 * pi)) * h2(\x, \x*\l) : (
      (\x <  3/\l) ? \e / (8 * sqrt(2 * pi)) * h3(\x, \x*\l) : (
      sqrt(2/pi) * \e)))));
  };
}

\def\cdf(#1)(#2)(#3){0.5*(1+(erf((#1-#2)/(#3*sqrt(2)))))}%
% to be used: \cdf(x)(mean)(variance)

\DeclareMathOperator{\CDF}{cdf}
\tikzset{
    declare function={
        normcdf(\x,\m,\s)=1/(1 + exp(-0.07056*((\x-\m)/\s)^3 - 1.5976*(\x-\m)/\s));
    }
}
%%%%%%%%%%%%%%%%%%%%%%%%%%%%%%%%%%%%%%%%%%%%%%%%%%%
%%%%%%%%%%%%%%%%%%%%%%%%%%%%%%%%%%%%%%%%%%%%%%%%%%%
\newcommand{\nmfr}[3]{\Phi\left(\frac{{#1} - {#2}}{#3}\right)}


\begin{document}

\lstset{language=C++,
                basicstyle=\tiny\ttfamily,
                keywordstyle=\color{blue}\ttfamily,
                stringstyle=\color{red}\ttfamily,
                commentstyle=\color{gray}\ttfamily,
                morecomment=[l][\color{gray}]{\#}
}


\thispagestyle{empty}


\nin{\LARGE Problem Set 8 {\gr }}\hfill{\bf Prof. Oke}

\medskip\hrule\medskip

\nin {\small CEE 260/MIE 273: Probability \& Statistics in Civil Engineering
\hfill\textit{ 12.06.2024}}

\bigskip

\nin{\it {Due Friday, December 13, 2024 at 11:59 PM as PDF uploaded on Canvas.}
  %If it helps and if possible, you can write your responses directly on this document and upload it instead.
    \textbf{Show as much work as possible in order to get FULL credit.}
    There are 8 problems with a total of 55 points available.
 %   \textbf{Important:} If you use MATLAB for any probability computations, briefly write/include the statements you
   % used to arrive at your answers. If instead you use probability tables, note this in the respective solution, as well.
}

 




 


\section*{Problem 1 \quad {\it Chi-square hypothesis test conclusions (9 points)}      }
What conclusions would be appropriate for an upper-tailed chi-square test in each of the following situations (where $\chi^2$ is the test statistic)? (In each case, show explicitly how you compute and compare the critical value $\chi^2_{1-\alpha, \nu}$. Then circle the correct option ({\bf i.}) or ({\bf ii.}).)

\begin{enumerate}[\bf (a)]
	\item $\alpha = 0.01$, $\nu = 3$, $\chi^2 = 8.54$ \pts{3}
	\vspace{20ex}
	\begin{enumerate}[\bf i.]
		\item Fail to reject $H_0$
		\item Reject $H_0$
	\end{enumerate}
	\item $\alpha = 0.10$, $\nu = 2$, $\chi^2 = 4.36$ \pts{3}
		\vspace{20ex}
	\begin{enumerate}[\bf i.]
		\item Fail to reject $H_0$
		\item Reject $H_0$
	\end{enumerate}
	\item $\alpha = 0.01$, $k = 6$, $\chi^2 = 10.20$ \pts{3}
		\vspace{20ex}
	\begin{enumerate}[\bf i.]
		\item Fail to reject $H_0$
		\item Reject $H_0$
	\end{enumerate}
\end{enumerate}


\section*{Problem 2\quad {\it $p$-value of chi-square statistic (6 points)}      }
Calculate the $p$-value for an upper-tailed chi-square test in each of the following situations (show the Matlab/Python/calculator function you use in each case):
\begin{enumerate}[\bf (a)]
	\item $\chi^2 = 13.0$, $\nu = 6$\pts{2}
	\vspace{30ex}
	\item  $\chi^2 = 18.0$, $\nu = 9$\pts{2}
	\vspace{30ex}
	\item $\chi^2 = 5.0$, $k =4$\pts{2}
	\vspace{30ex}
\end{enumerate}

\eject
\section*{Problem 3 \quad {\it ANOVA (8 points)}      }
The China Health and Nutrition Survey aims to examine the effects of the health,
nutrition, and family planning policies and programs implemented by national and
local governments\footnote{UNC Carolina Population Center,
	\url{https://www.cpc.unc.edu/projects/china}}. It, for example, collects
information on number of hours  parents spend taking care of their
children under age 6.  Provided below is
the partial ANOVA output for comparing average hours across educational attainment
categories.

\begin{table}[h!]
	\centering
	\begin{tabular}{l r r r r r}\hline
		& d.o.f. & $SS$ & $MS$ & $f$ & $P(>F)$ \\ \hline
		Education & 4 & 4142.09 & 1035.52 & ?  & ?  \\
		Residuals & 794 & 653047.83 & 822.48 & &  \\\hline
	\end{tabular}
\end{table}
\begin{enumerate}[(a)]
	\item Write the hypotheses for testing for a difference between the average number of hours spent on child
	care across educational attainment levels. \pts{2}
	\vspace{20ex}
	\item Compute the test statistic. \pts{2}
		\vspace{20ex}
	\item Compute the $p$-value. \pts{2}
		\vspace{20ex}
	\item What is the conclusion of the hypothesis test and why? \pts{2}
		\vspace{20ex}
\end{enumerate}

\section*{Problem 4 \quad {\it Identifying relationships (12 points)}      }
For each of the six plots, identify the strength of the \textbf{relationship} (whether linear or not) (e.g.\
weak, moderate, or strong) in the data \textbf{and} whether fitting a linear model would be reasonable.
\begin{figure}[h!]
  \centering
    \subfloat{
    
        \includegraphics[width=.3\textwidth]{identify_relationships_u}
  }
  \subfloat{
    \includegraphics[width=.3\textwidth]{identify_relationships_lin_pos_strong}
  }
  \subfloat{
    \includegraphics[width=.3\textwidth]{identify_relationships_lin_pos_weak}
  }


  \subfloat{
    \includegraphics[width=.3\textwidth]{identify_relationships_n}
  }
  \subfloat{
    \includegraphics[width=.3\textwidth]{identify_relationships_lin_neg_strong}
  }
  \subfloat{
    \includegraphics[width=.3\textwidth]{identify_relationships_none}
  }

\end{figure}
   \begin{enumerate}[\bf (a)]
   \item
     \bigskip
\bigskip
   \item
     \bigskip
\bigskip
        \item
     \bigskip
\bigskip
        \item
     \bigskip
\bigskip

        \item
     \bigskip
     \bigskip
     
   \item
     \bigskip
\bigskip
  \end{enumerate}
 % \begin{enumerate}[(a)]
% \item Write the hypotheses for testing for a difference between the average number of hours spent on child
%   care across educational attainment levels.
% \item Compute the test statistic.
% \item Compute the $p$-value.
% \item What is the conclusion of the hypothesis test?
% \end{enumerate}

\eject

\section*{Problem 5 \quad {\it Understanding correlation: True/False (4 points)}}
Determine if the following statements are absolutely true or false. Briefly explain your reasoning.

\begin{enumerate}[\bf (a)]
\item  A correlation coefficient of $-0.90$ indicates a stronger linear relationship than a correlation of 0.5.
  \vspace{15ex}
  
\item  Correlation is a measure of the relationship (linear or otherwise) between any two variables.
 
\vspace{15ex}
%  \item In a paired analysis we first take the difference of each pair of observations, and then we do inference
%  on these differences.
%\vspace{15ex}  
%\item Two data sets of different sizes cannot be analyzed as paired data.
%
%\vspace{15ex}
\end{enumerate}


\eject
\section*{Problem 6 \quad {\it Working with a regression equation (7 points)}}
Suppose that in a certain chemical process the reaction time $y$ (hr) is related to the temperature ($^\circ$F) in the chamber in which
the reaction takes place according to the simple linear regression model with equation:
\begin{equation*}
  y = 5.00 - 0.01x
\end{equation*}
and $\sigma = 0.075$.
\begin{enumerate}[\bf (a)]
\item What is the expected change in reaction time for a 1$^\circ$F increase in temperature? \pts{1}
  \vspace{15ex}
  
\item What is the expected reaction time when the temperature is 250$^\circ$F? \pts{2}
\vspace{25ex}

\item Suppose five observations are made independently \pts{4}
  on reaction time, each one for a temperature of 250$^\circ$F, what is the probability that all five times are between 2.4 and 2.6 hr?
 
\end{enumerate}

\eject
\section*{Problem 7 \quad {\it Interpreting least-squares linear regression output in MATLAB  (9 points)}}

The table below shows MATLAB output from fitting the least squares regression line to the number of fatal accidents in
each U.S. state to the population of the state.\footnote{I would strongly encourage you run the script
  \texttt{least\_squares\_regression.m} in the M6a folder on Moodle to gain further perspective. More will be said on
  inference in linear regression during M6b.}

\begin{table}[h!]
  \centering
  \begin{tabular}{c l l l l}
    &  \bf Estimate & \bf SE & \bf tStat & \bf pValue \\\midrule \\
    \bf (Intercept) & 142.71 & 64.88  & 2.200  & 0.0326 \\
    \bf x1 & 0.00013 & 7.8854 & 15.933 & 5.1357 $\times 10^{-21}$ \\ 
  \end{tabular}
  \caption{Output from least-squares regression of accidents on population in MATLAB}
  \label{tab:p5}
\end{table}
\noindent Let $y$ be the number of accidents in a state and $x$ its population. The ``Estimate'' column of the table shows the values of the regression coefficients.

\begin{enumerate}[\bf (a)]
\item   Write the regression equation in the form: $y = \hat\beta_0 + \hat\beta_1 x$ \pts{2}
  \vspace{15ex}
  
\end{enumerate}

\noindent  The ``SE'' column shows the standard errors of the points estimates, $\hat\beta_0$ and $\hat\beta_1$, respectively. The ``tStat'' column indicates the respective test statistics ($T$-scores). The ``pValue'' column shows the $p$-values of the test statistics. The hypotheses for $\beta_0$ (first row of table) are:
 \begin{align*}
   H_0: \beta_0 &= 0 \quad \text{(The true intercept is zero)} \\
   H_1 : \beta_0 &\ne 0 \quad \text{(The true intercept is not zero)}
 \end{align*}
 
 \begin{enumerate}[\bf (a)]\setcounter{enumi}{1}
 \item Based on the $p$-value \pts{2} of the intercept, determine whether the intercept is significant (i.e.\ non-zero) at $\alpha = 0.05$. Would your conclusion change if $\alpha = 0.01$?

   \vfill
 \end{enumerate}
 \eject
 
The hypotheses for $\beta_1$ (second row of table) are:
 \begin{align*}
   H_0: \beta_1 &= 0 \quad \text{(The true slope is zero)} \\
   H_1 : \beta_1 &\ne 0 \quad \text{(The true slope is not zero)}
 \end{align*}
 \begin{enumerate}[(a)]\setcounter{enumi}{2}
 \item Based on the size of the $p$-value for $\hat\beta_1$, \pts{2} what is the conclusion of the hypothesis test for $\beta_1$? Explain your answer briefly.
   \vspace{25ex}
   
 \item Would you \pts{1} agree that the size of of the $p$-value is a strong indication of the linear relationship between $x$ and $y$?
   \vspace{15ex}
   
 \item If the \pts{2}  population of a state were 10,000,000, what would be the expected number of accidents?
   
 \end{enumerate}

 \eject
\section*{Problem 8\quad {\it Regression and ANOVA (8 points)}      }
The following script was run in MATLAB to regress the number of accidents on the population of a state.
\begin{verbatim}
load accidents
x = hwydata(:,14); %Population of states
y = hwydata(:,4);  %Accidents per state
mdl = fitlm(x,y);
mdl.anova('summary')
\end{verbatim}
Below is the output of running the ANOVA method on the model estimate:
\begin{verbatim}
                  SumSq       DF      MeanSq        F         pValue  
                __________    __    __________    ______    __________

    Total       3.5776e+07    50    7.1553e+05                        
    Model       2.9988e+07     1    2.9988e+07    253.86    5.1357e-21
    Residual    5.7883e+06    49    1.1813e+05                        
\end{verbatim}
\begin{enumerate}[(a)]
\item What are the values of $SSE$, $SSR$ and $SST$? \pts{3}
  \vspace{20ex}
  
\item Use the above output to compute the coefficient of determination $R^2$. \pts{2}
\vspace{20ex}
  
\item How many observations are in the data sample? \pts{1}
\vspace{7ex}
  
\item The null hypothesis of the ANOVA test for a \pts{2} linear regression statest that the true slope is equal to zero: $H_0: \beta_1 = 0$. Given the $p$-value in the above output, do you reject or fail to reject the null hypothesis? Briefly explain the reason behind your answer.

  
\end{enumerate}


\end{document}

%%% Local Variables:
%%% mode: latex
%%% TeX-master: t
%%% End:
